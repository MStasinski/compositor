%------- INPUT
\input trans
\input epsfx
\input ../input/bop-hax
\input map
\input split

%====================================================
%                  F O N T Y
%====================================================
\defaultfontfeatures{Ligatures=TeX,Scale=.92}
\setmainfont{Lucida Bright OT}
\setsansfont{Lucida Sans OT}
\setmonofont{Lucida Sans Typewriter OT}
\setmathfont{Lucida Bright Math OT}
\setmathfont[version=bold]{Lucida Bright Math OT Demibold}

\newfontface\LucidaBlackletter{Lucida Blackletter OT}       %% a nuż się kiedyś przydadzą
\newfontface\LucidaCalligraphy{Lucida Calligraphy OT}
\newfontface\LucidaHandwriting{Lucida Handwriting OT}

\makeatletter
  \newcommand\espe{\@setfontsize\espe{11.5pt}{15}}
  \newcommand\espemat{\@setfontsize\espemat{13pt}{12.5}}
  \newcommand\gimbaza{\@setfontsize\gimbaza{10pt}{14}}
  \newcommand\gimbazamat{\@setfontsize\gimbazamat{11.5pt}{11}}
  \newcommand\trialheader{\@setfontsize\trialheader{14pt}{12}}
  \newcommand\trialheaderT{\@setfontsize\trialheaderT{7pt}{7}}
  \newcommand\trialheaderR{\@setfontsize\trialheaderR{10pt}{10}}
  \newcommand\gimbazanorr{\@setfontsize\gimbazanorr{8pt}{10}}  %%dodałem, bo potrzebowałe mniejszych Ł.
  \newcommand\pisankaG{\@setfontsize\gimbaza{12pt}{15.5}}
\makeatother

% czerwony tekst (wyróżnienie tekstu)
\newcommand{\highlighted}[1]{{\color{red}#1}}

\def\nor{}
\def\norr{}   %%ale to coś mi nie działa :)
\let\nor\gimbaza
\let\norr\gimbazanorr

\newif \ifsklad
  \skladfalse

%------- Ułamki
\makeatletter                       %przerabianie fraca od AMStex
  \DeclareRobustCommand{\frac}{\new@ifnextchar[{\frfrac{}}{\frfrac{}[]}}
  \def\g@ra{\raise.125em}
  \def\rozp@rka{\vrule height.615em depth0pt width0pt}
  \def\frfrac#1[#2]#3#4{\def\next@{#2}%
    \ifx\next@\@empty \def\next@{\g@ra\hbox{$#1{#3\over\rozp@rka\relax#4}$}}%
    \else \def\next@{\g@ra\hbox{$#1{#3\above#2\relax\rozp@rka\relax#4}$}}%
    \fi
    \next@}
  \def\dfrac{\protect\frfrac\displaystyle[]}
  \def\tfrac{\protect\frfrac\textstyle[]}
\makeatother

\def\jednostki#1#2{\frac{\mathrm{#1}}{\mathrm{#2}}}         % km na godzine itepe

\catcode`\=\active
\def \Imat
  {\begingroup
   \def{$\egroup\endgroup}%
   \lower0.03em\hbox\bgroup\gimbazamat$%%%%%%%%%%%
  }
\let=\Imat%

\catcode`\=\active
\def \IImat
  {\begingroup
   \def{$\egroup\endgroup}%
   \lower0.06em\hbox\bgroup\espemat$%%%%%%%%%%%
  }
\let=\IImat%

%frac mały w obu, Frac duży w gimnazjum, FracII duży w SP
\def\Frac #1#2{\bgroup \frac{#1}{#2} \egroup}
\def\FracII #1#2{\bgroup \frac{#1}{#2} \egroup}

%------- Rozmiary strony definiujemy na początku main.tex
\topskip = 0mm

\parindent=0cm
\parskip0pt
\hfuzz=1pt
\thinmuskip=3mu
\medmuskip=4mu
\thickmuskip=5mu
\renewcommand{\line}[1]{\hbox to\hsize{#1}}
\def \v #1 {\vskip#1mm\relax}
\def \h #1 {\hskip#1mm\relax}
\def \vp #1 {\vskip#1pt\relax}

%------- county dimeny fontowe bzdety itepe
\def\kompozytorFont[1#1pt]{%
\ifcase #1\gimbaza%0
      \or\espe%1
      \else Błąd. Niepoprawny rozmiar czcionki.\fi%error
}

\newdimen\wysI\wysI=0pt         %uzasadnienie
\newdimen\wysII\wysII=0pt       %uzasadnienie
\newdimen\szerI\wysI=0pt         %uzasadnienie
\newdimen\szerII\wysII=0pt       %uzasadnienie
\newdimen\szerIII\wysII=0pt       %uzasadnienie
\newdimen\wysMAX\wysMAX=0pt     %uzasadnienie
\newdimen\komMAX\komMAX=0pt     %uzasadnienie
\newcount\Podpunkt \Podpunkt=97     %\pp
\newcount\Przypis \Przypis=1       %\pr
\newcount\Numer\Numer=1       %\Nr
\newdimen\pindent \pindent=4.5mm    %\pp
\newdimen\SzerOka  %\odp
\newdimen\Szeroka  %\odp
\newdimen\SPR      %\odp
\newdimen\tmpa      %kratki
\newdimen\temp      %podręczna
\newif\ifsklad
  \skladtrue
\newdimen\kol       %wcinka
\newdimen\szer      %wcinka
\newdimen\dszer     %wcinka
\newdimen\Wcinka    %wcinka
\global\Wcinka\hsize%wcinka

\def\NazwaEPSfont{\tt\tiny}

%------- Kropeczki
\def\dotfill{\xleaders\hbox to 3.219pt{\hss\tiny.\hss}\hfill}
\def\Dotfill{\xleaders\hbox to 3.219pt{\hss\tiny.\hss}\hfill}
\def\ddotfill#1{\xleaders\hbox to 3.219pt{\hss\lower#1\hbox{\tiny.}\hss}\hfill}
\def\Kr#1{\hbox to #1cm{\Dotfill}}%

\def\Krl#1{\kern0pt\lower.68mm\hbox{\Kr{#1}}}%1mm
\def\Krd#1{\kern0pt\lower1.3mm\hbox{\Kr{#1}}}%1mm
\def\Krdd#1{\kern0pt\lower1.6mm\hbox{\Kr{#1}}}%1mm
\def\kr{\hbox{$\ldots$}}%

\def\Kropky#1{\bgroup\hskip0mm\hbox to #1mm{\small
\setlength{\unitlength}{1mm}
\multido{\n=0+1}{#1}{%
\put(\n,0){.}}
\hss}\egroup}
\def\kropkiON{\rput(0,0){\phantom{\DF{  }}}}

\def\DF#1{%
\setbox1=\hbox{#1}
\tmpa=\hsize\advance\tmpa by -\wd1 \advance\tmpa by -.5em
\line{\box1\hfill\lower.68mm\hbox to\tmpa{\Dotfill}}}

\def\DFI#1{%
\setbox1=\hbox{#1}
\tmpa=\hsize\advance\tmpa by -\wd1 \advance\tmpa by -.5em
\line{\box1\hfill\lower.68mm\hbox to\tmpa{\Dotfill}}}

\def\DFII#1{%
\setbox1=\hbox{#1}
\tmpa=\hsize\advance\tmpa by -\wd1 \advance\tmpa by -.5em
\line{\box1\hfill\lower.68mm\hbox to\tmpa{\Dotfill}}}

\def\DFD#1{%
\setbox1=\hbox{#1}
\tmpa=\hsize\advance\tmpa by -\wd1 \advance\tmpa by -.5em
\line{\box1\hfil\lower1.5mm\hbox to\tmpa{\Dotfill}}}

\def\DFDD#1{%
\setbox1=\hbox{#1}
\tmpa=\hsize\advance\tmpa by -\wd1 \advance\tmpa by -.5em
\line{\box1\hfil\lower2.5mm\hbox to\tmpa{\Dotfill}}}


%------- Podmiana znaków matematycznych
\XeTeXmathchardef\xleq = 1 0 "02A7D
\XeTeXmathchardef\xgeq = 1 0 "02A7E
\let\leq\xleq
\let\geq\xgeq
\let\le\xleq
\let\ge\xgeq
\XeTeXmathchardef\kat = 1 0 "02222
\XeTeXmathchardef\permil = 1 0 "02030
\def\promil{\hbox{\kern1pt$\permil$}}
\XeTeXmathcode`\: = 1 0 "0003A
\def\proc{\kern1pt\%}
\def\Circ{\raise3pt\zscale{75}\hbox{$\circ$}}
\def\CircII{\raise3pt\zscale{90}\hbox{$\circ$}}
\catcode`\=\active
\catcode`\=\active
\gdef\POLSPACJA {\nobreak\hskip 1.55pt\relax}%
\gdef\TWARDASPACJA {\nobreak\hskip 3.1pt\relax}%
\let\POLSPACJA
\let=\TWARDASPACJA

%------- Wcinki i dwaokna

\def \lwcinka #1 #2
    {\nointerlineskip
     \vskip1.8mm
     \vglue-1.8mm
%     \global \Podpunkt=97
     \global \Wcinka=#1mm
     \kol=6mm
     \dszer=\hsize
     \advance\dszer by -\Wcinka
     \szer=\dszer
     \advance\szer by -\kol
     \def \c ##1{\hbox to#1mm{\hskip2\leftskip\hss ##1\hss}}
     \def \cwc ##1{\hskip0pt \hbox to\szer{\hss ##1\hss}}
     \rightline{\llap{\smash{\hbox{\vtop{\hsize\szer
                \overfullrule=0mm
                \vskip1.8mm
                \rm #2}}}}}
     \begingroup\rightskip\dszer
    }

\def \rwcinka #1 #2
    {\nointerlineskip
     \vskip1.8mm
     \vglue-1.8mm
%     \global\Podpunkt=97
     \global\Wcinka=#1mm
     \kol=6mm
     \dszer=\hsize
     \advance\dszer by -\Wcinka
     \szer=\dszer
     \advance\szer by -\kol
     \def \c ##1{\hskip0mm \hbox to#1mm{\hfil ##1 \hfil}}
     \def \cwc ##1{\noindent\hskip0pt \hbox to\szer{\hfil ##1 \hfil}}
     \rlap{\smash{\hbox{\vtop{\hsize\szer
           \vskip1.8mm
           \rm #2}}}}
     \begingroup\leftskip\dszer
     }

\def \wcinkaEND
    {\par
     \endgroup
     \global\Wcinka\hsize
    }

\long\def\dwaokna #1 #2 #3 #4 #5{\noindent\hbox to#1mm{%
    \vtop{\hsize=#2mm #4}\hfil\vtop{\hsize=#3mm #5}}}

\long\def\trzyokna #1 #2 #3 #4 #5 #6{\noindent\hbox to 161mm{%
    \vtop{\hsize=#1mm #4}\hfil\vtop{\hsize=#2mm #5}\hfil\vtop{\hsize=#3mm #6}\hfil}}


\long\def\Trzyokna #1 #2 #3 #4 #5 #6{\noindent\hbox to 161mm{%
    \vtop{\hsize=#1mm #4}\hfill\vtop{\hsize=#2mm #5}\hfill\vtop{\hsize=#3mm #6}}}

\def\Metka #1 {%
\h0
\setbox0\hbox{#1}
\setbox1\hbox{\psframebox[linecolor=black, linewidth=.35mm, fillstyle=none, framesep=2mm, framearc=.4]{\hbox to\wd0{#1}}}
\temp=\wd1
\divide\temp by 2
\hbox{%
\vtop{\hsize=\wd1%
\rput[b](\temp,-\dp1){\psframe[linecolor=black, linewidth=.15mm](-.25,.025)(.25,-.4975)}\hfil\copy1\hfil%
\vskip5mm}\hss}
}


\newcmykcolor{yllwV}{0 0 .018 0}
%------- format zadań, odpowiedzi, podpunktów i inne makra pozycjonujące
%\long\def\zad#1\ezad{\vskip3mm\global\Podpunkt=97{\hskip5mm\vtop{\hsize=165mm\vrule height12pt depth0mm width0pt#1}}}%
%\long\def\zad#1\ezad{\vskip2pt\global\Podpunkt=97\hskip5mm\vtop{\hsize=165mm\vrule height0pt depth0mm width0pt#1}}%
\long\def\zad#1\ezad{\vskip2pt\global\Podpunkt=97\psframebox[linecolor=yllwV,linewidth=.5pt,framesep=1mm]{\hskip5mm\vtop{\hsize=165mm\vrule height0pt depth0mm width0pt#1}}}%
%\long\def\zad#1\ezad{{\vrule width500pt}\vskip3mm\global\Podpunkt=97\psframebox[linecolor=red,linewidth=.5pt,framesep=1mm]{\hskip5mm\vtop{\hsize=165mm\vrule height12pt depth0mm width0pt#1}}}%
\long\def\ODP#1\ezad{\vskip3mm\global\Podpunkt=97\psframebox[linecolor=blue,linewidth=.5pt,framesep=1mm]{\hskip5mm\vtop{\hsize=165mm\vrule height12pt depth0mm width0pt#1}}}%
\long\def\ODPII#1\ezad{\vskip3mm\global\Podpunkt=97\psframebox[linecolor=blue,linewidth=.5pt,framesep=1mm]{\hskip5mm\vtop{\hsize=165mm\vrule height12pt depth0mm width0pt#1}}}%

%\long\def\zadII#1\ezad{\global\Podpunkt=97\psframebox[linecolor=yllwV,linewidth=.5pt,framesep=1mm]{\hskip5mm\vtop{\hsize=165mm\vrule height12pt depth0mm width0pt#1}\hskip0mm}}%
\long\def\zadII#1\ezad{\vskip3mm\global\Podpunkt=97\psframebox[linecolor=red,linewidth=.5pt,framesep=1mm]{\hskip5mm\vtop{\hsize=165mm\vrule height12pt depth0mm width0pt#1}}}%

%\def\z #1 #2{#2}
\def\z #1 #2 #3%
{\nobreak%
\Podpunkt=97%
\setbox0\hbox{#1}%
\setbox1\hbox{#2}%
\bgroup%
\ifnum\wd0>0\relax\rput[Br](-6pt,0){#1.}\relax\fi%
#3
\ifnum\wd1>0\relax\hfill (0--#2pkt)\relax\fi
\egroup}

\def \ppr
    {\leavevmode%
     \przypis%
     \global \advance \Przypis by 1\relax
     \ignorespaces}

\def \nnr
    {\leavevmode%
     \numerek%
     \global\advance\Numer by 1\relax
     \ignorespaces}

\def\przypis {\smash{\raise.2\baselineskip\hbox to 0.4em{\scriptsize\the\Przypis\hss}}}

\def \ppp
    {\leavevmode%
     \podpunkt%
     \global \advance \Podpunkt by 1\relax
     \ignorespaces}

\def\podpunkt {\hbox to \pindent{\char\Podpunkt)\hss}}

\def\numerek {\hbox to \pindent{\the\Numer.\hss}}

\def \pp
    {\ifvmode\vskip.5mm \ppp\else\ppp\fi}

\def \pr
    {\ifvmode\vskip.5mm \ppr\hskip1em\else\ppr\fi}

\def \Nr
    {\ifvmode\vskip.5mm \nnr\else\nnr\fi}

\def \st #1 #2 {\vrule width0pt height #1mm depth #2mm}
\def \stpt #1 #2 {\vrule width0pt height #1pt depth #2pt}

\def\ph#1{\phantom{#1}}

\def\doc{{\parfillskip0pt\endgraf}}

\def\hh #1 #2% -#1: od prawej, +#1: od lewej, -#2: od zera do #2 indent, +#2: po #2 linijkach indent
{\hangindent#1cm
\hangafter#2
}

%------- Kratki, \Frame i kółka
\def\koleczko #1%
    {\setbox0=\hbox{%
    \pscircle[linewidth=.3pt](.22,.12){.22}
    {\hbox to 4.4mm{\hss#1\hss}}}%
    \box0\hskip.2mm }



\def\kolkogim{}
\def\kolkosp{}

\def\pkolko{%
    \pscircle[linewidth=0pt,linecolor=Raster,fillstyle=solid,fillcolor=Raster](-.3,.15){.2}}

\def\szkratka{\hbox to 4mm{\psframe[linewidth=.25mm,linecolor=Raster,fillstyle=solid,fillcolor=Raster](0,0)(.4,.4)\hfil}}

\def\mkolko #1%
    {\setbox0=\hbox{%
    \pscircle[linewidth=.3pt](.2,.135){.25}
    {\hbox to 4mm{\hss#1\hss}}}%
    \box0\hskip.73mm }

\let\kolkogim\koleczko
\let\kolkosp\mkolko

\long\def\Frame #1#2#3#4{%
 \vbox{\hrule height#2 \hbox{\vrule width#2
 \hskip#1 \vbox{\vskip#1{}\hsize#3#4\vskip#1}\hskip#1
 \vrule width#2} \hrule height#2}}

\newcmykcolor{colBudek}{0 0 0 .35}
\def\Budka#1{\hbox{\psframe[linewidth=.18mm,framearc=.4, linecolor=colBudek](0,-.1)(.#1,.4)\vrule height4.2mm width0mm depth1.2mm\hskip#1mm}}
\def\bud#1{\hbox{\psframe[linewidth=.18mm,framearc=.4, linecolor=colBudek](0,-.15)(#1,.35)\vrule height3.2mm width0mm depth1.2mm\hskip#1cm}}

\def\graframka#1{\psframebox[framearc=.4,framesep=2mm]{\vrule height10.5pt depth3.5pt width0pt\hbox to 15mm{\hss#1\hss}}}
\def\kleks{\lower4.5pt\hbox{\XeTeXpdffile 'pdf/kleks.pdf' }}
\def\kratka{\zscale{100}\Frame{0pt}{.3pt}{4mm}{\vbox to 5.2mm{\hsize=4mm\hfil\vfil}}}
\def\kratkaSz{\zscale{100}\Frame{0pt}{.3pt}{4mm}{\vbox to 5.2mm{\hsize=6mm\hfil\vfil}}}
\def\krat{\lower2mm\hbox{\Frame{0pt}{.3pt}{4mm}{\vbox to 5.5mm{\hsize=4mm\hfil\vfil}}}}
\def\kratkw{\lower1mm\hbox{\Frame{0pt}{.3pt}{4mm}{\vbox to 4mm{\hsize=4mm\hfil\vfil}}}\kern1.5mm\ignorespaces} % kratka do makr NIE RUSZAC!!!
\def\nkratkw{\noindent\lower1mm\hbox{\Frame{0pt}{.3pt}{4mm}{\vbox to 4mm{\hsize=4mm\hfil\vfil}}}\kern1.5mm\ignorespaces} % kratka do makr NIE RUSZAC!!!
\def\kratsc{\kern.3pt\lower.5mm\hbox{\Frame{0pt}{.3pt}{2.5mm}{\vbox to 2.5mm{\hsize=2.5mm\hfil\vfil}}}}
\def\kratt{\lower1.2mm\hbox{\Frame{0pt}{.3pt}{8mm}{\vbox to 4.5mm{\hsize=4.5mm\hfil\vfil}}}}
\def\kratul{\lower2mm\hbox{\Frame{0pt}{.3pt}{4mm}{\vbox to 8.5mm{\hsize=5.5mm\hfil\vfil}}}}
\def\kratKW{\lower1mm\hbox{\Frame{0pt}{.3pt}{4mm}{\vbox to 4mm{\hsize=4mm\hfil\vfil}}}}
\def\kratKWTAB{\psframe[linecolor=black, linewidth=.3pt](-2mm,-.7mm)(2mm,3.5mm)}
\def\KR{\lower2mm\hbox{\Frame{0pt}{.3pt}{7mm}{\vbox to 6mm{\hsize=7mm\hfil\vfil}}}}
\def\KRR{\lower1.25mm\hbox{\Frame{0pt}{.3pt}{9mm}{\vbox to 4.5mm{\hsize=9mm\hfil\vfil}}}}           % 2 cyfry luzem SP
\def\frejm{\lower3pt\hbox{\Frame{0pt}{0.3pt}{4mm}{\vbox to 12pt{\hsize=10mm\hfil\vfil}}}} %221 janowicz 15
\def\arrow#1#2#3{\raise3pt\hbox to 12mm{\rput(.5,.25){$#1{#2}^{#3}$}\psline{->}(1,0)}} %214 janowicz 8
\def\strzalkado#1{\raise3pt\hbox to 14mm{\rput[B](.7,.25){#1}\psline{->}(.1,0)(1.3,0)}}
\def\Kwadiiv{\lower1.5mm\hbox{\Frame{2.5mm}{.3pt}{6mm}{}}}  %214 janowicz 8
\def\mkw{\vbox{\hrule %
               \hbox{\vrule \vrule depth1mm height1mm width0mm \hskip2mm \vrule}%
               \hrule}}

\def\kwadracik{\noindent\lower.75mm\vbox{\hrule %
               \hbox{\vrule \vrule depth3mm height2mm width0mm \hskip4mm \vrule}%
               \hrule}}

\def\Kwadracik{\noindent\lower.75mm\vbox{\hrule %
               \hbox{\vrule \vrule depth3mm height2mm width0mm \hskip6mm \vrule}%
               \hrule}}


%------- makra do zadań: odp, pf, tn, tnp
\def\forAga{{\sf X\ }}


\long\def\odp #1 #2 #3 #4
     {{\nobreak\overfullrule=0pt
      \vskip2mm
      \nobreak
      \mathsurround=0pt
      \setbox0\hbox{\nkratkw{\rm A.}~#1}
      \setbox1\hbox{\nkratkw{\rm B.}~#2}
      \setbox2\hbox{\nkratkw{\rm C.}~#3}
      \setbox3\hbox{\nkratkw{\rm D.}~#4}
      \setbox4\hbox{#1}
      \setbox5\hbox{#3}
      \ht4=0mm\ht5=0mm\wd5=0mm%
      \ifnum\wd0>\wd1\relax\SzerOka=\wd0\else\SzerOka=\wd1\relax\fi
      \ifnum\wd2>\wd3\relax\Szeroka=\wd2\else\Szeroka=\wd3\relax\fi
\SPR=\wd0\advance\SPR by\wd1\advance\SPR by\wd2\advance\SPR by\wd3\advance\SPR by30mm
\ifdim\SPR<\hsize
\hskip0pt \hbox to\hsize{%
      {\nkratkw\rm {A.}}~#1\hskip9.3mm plus.5mm minus5mm
      {\nkratkw\rm {B.}}~#2\hskip9.3mm plus.5mm minus5mm
      {\nkratkw\rm {C.}}~#3\hskip9.3mm plus.5mm minus5mm
      {\nkratkw\rm {D.}}~#4\hfil\hss}\vskip0pt
\else
\SPR=\SzerOka\advance\SPR by \Szeroka \advance\SPR by 20mm
\ifdim\SPR<\hsize
      \hskip0pt \hbox to\hsize{\leftskip=0mm%
              \vbox{\hsize=\SzerOka
      \line{{\nkratkw\rm A.}~\box4\hfil}%
      \vskip1mm
      \line{{\nkratkw\rm B.}~\hbox{#2}\hfil}}\hskip20mm plus.5mm minus1mm
                  \vbox{\hsize=\Szeroka
      \line{{\nkratkw\rm C.}~\box5\hfil}%
      \vskip1mm
      \line{{\nkratkw\rm D.}~\hbox{#4}\hfil}%
               }\hss}%
\else
\ifdim\SzerOka>\Szeroka \SPR=\SzerOka\else\SPR=\Szeroka\fi
\hangindent10.5mm\hangafter1
\nkratkw{\rm A.}~#1
\vskip1mm
\hangindent10.5mm\hangafter1
\nkratkw{\rm B.}~#2
\vskip1mm
\hangindent10.5mm\hangafter1
\nkratkw{\rm C.}~#3
\vskip1mm
\hangindent10.5mm\hangafter1
\nkratkw{\rm D.}~#4
\vskip1mm
\fi\fi
}}

\long\def\przyporzadkujlitery #1 #2 #3 #4
     {{\nobreak\overfullrule=0pt
      \vskip2mm
      \nobreak
      \mathsurround=0pt
      \setbox0\hbox{{\rm A.}~#1}
      \setbox1\hbox{{\rm B.}~#2}
      \setbox2\hbox{{\rm C.}~#3}
      \setbox3\hbox{{\rm D.}~#4}
      \setbox4\hbox{#1}
      \setbox5\hbox{#3}
      \ht4=0mm\ht5=0mm\wd5=0mm%
      \ifnum\wd0>\wd1\relax\SzerOka=\wd0\else\SzerOka=\wd1\relax\fi
      \ifnum\wd2>\wd3\relax\Szeroka=\wd2\else\Szeroka=\wd3\relax\fi
\SPR=\wd0\advance\SPR by\wd1\advance\SPR by\wd2\advance\SPR by\wd3\advance\SPR by30mm
\ifdim\SPR<\hsize
\hskip0pt \hbox to\hsize{%
      {\rm {A.}}~#1\hskip9.3mm plus.5mm minus5mm
      {\rm {B.}}~#2\hskip9.3mm plus.5mm minus5mm
      {\rm {C.}}~#3\hskip9.3mm plus.5mm minus5mm
      {\rm {D.}}~#4\hfil\hss}\vskip0pt
\else
\SPR=\SzerOka\advance\SPR by \Szeroka \advance\SPR by 20mm
\ifdim\SPR<\hsize
      \hskip0pt \hbox to\hsize{\leftskip=0mm%
              \vbox{\hsize=\SzerOka
      \line{{\rm A.}~\box4\hfil}%
      \vskip1mm
      \line{{\rm B.}~\hbox{#2}\hfil}}\hskip20mm plus.5mm minus1mm
                  \vbox{\hsize=\Szeroka
      \line{{\rm C.}~\box5\hfil}%
      \vskip1mm
      \line{{\rm D.}~\hbox{#4}\hfil}%
               }\hss}%
\else
\ifdim\SzerOka>\Szeroka \SPR=\SzerOka\else\SPR=\Szeroka\fi
\hangindent10.5mm\hangafter1
{\rm A.}~#1
\vskip1mm
\hangindent10.5mm\hangafter1
{\rm B.}~#2
\vskip1mm
\hangindent10.5mm\hangafter1
{\rm C.}~#3
\vskip1mm
\hangindent10.5mm\hangafter1
{\rm D.}~#4
\vskip1mm
\fi\fi
}}

\long\def\przyporzadkujcyfry #1 #2 #3 #4
     {{\nobreak\overfullrule=0pt
      \vskip2mm
      \nobreak
      \mathsurround=0pt
      \setbox0\hbox{{\rm 1.}~#1}
      \setbox1\hbox{{\rm 2.}~#2}
      \setbox2\hbox{{\rm 3.}~#3}
      \setbox3\hbox{{\rm 4.}~#4}
      \setbox4\hbox{#1}
      \setbox5\hbox{#3}
      \ht4=0mm\ht5=0mm\wd5=0mm%
      \ifnum\wd0>\wd1\relax\SzerOka=\wd0\else\SzerOka=\wd1\relax\fi
      \ifnum\wd2>\wd3\relax\Szeroka=\wd2\else\Szeroka=\wd3\relax\fi
\SPR=\wd0\advance\SPR by\wd1\advance\SPR by\wd2\advance\SPR by\wd3\advance\SPR by30mm
\ifdim\SPR<\hsize
\hskip0pt \hbox to\hsize{%
      {\rm {1.}}~#1\hskip9.3mm plus.5mm minus5mm
      {\rm {2.}}~#2\hskip9.3mm plus.5mm minus5mm
      {\rm {3.}}~#3\hskip9.3mm plus.5mm minus5mm
      {\rm {4.}}~#4\hfil\hss}\vskip0pt
\else
\SPR=\SzerOka\advance\SPR by \Szeroka \advance\SPR by 20mm
\ifdim\SPR<\hsize
      \hskip0pt \hbox to\hsize{\leftskip=0mm%
              \vbox{\hsize=\SzerOka
      \line{{\rm 1.}~\box4\hfil}%
      \vskip1mm
      \line{{\rm 2.}~\hbox{#2}\hfil}}\hskip20mm plus.5mm minus1mm
                  \vbox{\hsize=\Szeroka
      \line{{\rm 3.}~\box5\hfil}%
      \vskip1mm
      \line{{\rm 4.}~\hbox{#4}\hfil}%
               }\hss}%
\else
\ifdim\SzerOka>\Szeroka \SPR=\SzerOka\else\SPR=\Szeroka\fi
\hangindent10.5mm\hangafter1
{\rm 1.}~#1
\vskip1mm
\hangindent10.5mm\hangafter1
{\rm 2.}~#2
\vskip1mm
\hangindent10.5mm\hangafter1
{\rm 3.}~#3
\vskip1mm
\hangindent10.5mm\hangafter1
{\rm 4.}~#4
\vskip1mm
\fi\fi
}}

\def\przyporzadkujodp{\centerline{\rm {1.} \Krl{0.85}\hfil\rm {2.} \Krl{0.85}\hfil\rm {3.} \Krl{0.85}\hfil\rm {4.} \Krl{0.85}}}

\newcount\pfCount
\newcount\uzCount

\long\def\podpunkty #1 #2 #3 #4 #5 #6 #7 #8 #9 {
\overfullrule=0pt
      \nobreak
      \mathsurround=0pt
      \setbox0\hbox{#2}
      \setbox1\hbox{#3}
      \setbox2\hbox{#4}
      \setbox3\hbox{#5}
      \setbox4\hbox{#6}
      \setbox5\hbox{#7}
      \setbox6\hbox{#8}
      \setbox7\hbox{#9}
    \ifnum\wd0>0\relax\vskip0mm\hangindent4.5mm\advance\hangindent by #1\hangafter1\hskip#1\global\Podpunkt=97\pp \unhbox0\relax\fi%
    \ifnum\wd1>0\relax\vskip0mm\hangindent4.5mm\advance\hangindent by #1\hangafter1\hskip#1\global\Podpunkt=98\pp \unhbox1\relax\fi%
    \ifnum\wd2>0\relax\vskip0mm\hangindent4.5mm\advance\hangindent by #1\hangafter1\hskip#1\global\Podpunkt=99\pp \unhbox2\relax\fi%
    \ifnum\wd3>0\relax\vskip0mm\hangindent4.5mm\advance\hangindent by #1\hangafter1\hskip#1\global\Podpunkt=100\pp \unhbox3\relax\fi%
    \ifnum\wd4>0\relax\vskip0mm\hangindent4.5mm\advance\hangindent by #1\hangafter1\hskip#1\global\Podpunkt=101\pp \unhbox4\relax\fi%
    \ifnum\wd5>0\relax\vskip0mm\hangindent4.5mm\advance\hangindent by #1\hangafter1\hskip#1\global\Podpunkt=102\pp \unhbox5\relax\fi%
    \ifnum\wd6>0\relax\vskip0mm\hangindent4.5mm\advance\hangindent by #1\hangafter1\hskip#1\global\Podpunkt=103\pp \unhbox6\relax\fi%
    \ifnum\wd7>0\relax\vskip0mm\hangindent4.5mm\advance\hangindent by #1\hangafter1\hskip#1\global\Podpunkt=104\pp \unhbox7\relax\fi%
}

\long\def\Podpunkty #1 #2 #3 #4 #5 #6 #7 #8 #9 {
\overfullrule=0pt
      \nobreak
      \mathsurround=0pt
      \setbox0\hbox{#2}
      \setbox1\hbox{#3}
      \setbox2\hbox{#4}
      \setbox3\hbox{#5}
      \setbox4\hbox{#6}
      \setbox5\hbox{#7}
      \setbox6\hbox{#8}
      \setbox7\hbox{#9}
    \ifnum\wd0>0\relax\vskip0mm\hangindent4.5mm\advance\hangindent by #1\hangafter1\hskip#1\hbox to \pindent{A.} \unhbox0\relax\fi%
    \ifnum\wd1>0\relax\vskip0mm\hangindent4.5mm\advance\hangindent by #1\hangafter1\hskip#1\hbox to \pindent{B.} \unhbox1\relax\fi%
    \ifnum\wd2>0\relax\vskip0mm\hangindent4.5mm\advance\hangindent by #1\hangafter1\hskip#1\hbox to \pindent{C.} \unhbox2\relax\fi%
    \ifnum\wd3>0\relax\vskip0mm\hangindent4.5mm\advance\hangindent by #1\hangafter1\hskip#1\hbox to \pindent{D.} \unhbox3\relax\fi%
    \ifnum\wd4>0\relax\vskip0mm\hangindent4.5mm\advance\hangindent by #1\hangafter1\hskip#1\hbox to \pindent{E.} \unhbox4\relax\fi%
    \ifnum\wd5>0\relax\vskip0mm\hangindent4.5mm\advance\hangindent by #1\hangafter1\hskip#1\hbox to \pindent{F.} \unhbox5\relax\fi%
    \ifnum\wd6>0\relax\vskip0mm\hangindent4.5mm\advance\hangindent by #1\hangafter1\hskip#1\hbox to \pindent{G.} \unhbox6\relax\fi%
    \ifnum\wd7>0\relax\vskip0mm\hangindent4.5mm\advance\hangindent by #1\hangafter1\hskip#1\hbox to \pindent{H.} \unhbox7\relax\fi%
}

\long\def\podpunktyKratZa #1 #2 #3 #4 #5 #6 #7 #8 #9 {
\overfullrule=0pt
      \nobreak
      \mathsurround=0pt
      \setbox0\hbox{#2}
      \setbox1\hbox{#3}
      \setbox2\hbox{#4}
      \setbox3\hbox{#5}
      \setbox4\hbox{#6}
      \setbox5\hbox{#7}
      \setbox6\hbox{#8}
      \setbox7\hbox{#9}
    \ifnum\wd0>0\relax\vskip0mm\hangindent4.5mm\advance\hangindent by #1\hangafter1\hskip#1\global\Podpunkt=97\pp \unhbox0~~\kratkw\relax\fi%
    \ifnum\wd1>0\relax\vskip0mm\hangindent4.5mm\advance\hangindent by #1\hangafter1\hskip#1\global\Podpunkt=98\pp \unhbox1~~\kratkw\relax\fi%
    \ifnum\wd2>0\relax\vskip0mm\hangindent4.5mm\advance\hangindent by #1\hangafter1\hskip#1\global\Podpunkt=99\pp \unhbox2~~\kratkw\relax\fi%
    \ifnum\wd3>0\relax\vskip0mm\hangindent4.5mm\advance\hangindent by #1\hangafter1\hskip#1\global\Podpunkt=100\pp \unhbox3~~\kratkw\relax\fi%
    \ifnum\wd4>0\relax\vskip0mm\hangindent4.5mm\advance\hangindent by #1\hangafter1\hskip#1\global\Podpunkt=101\pp \unhbox4~~\kratkw\relax\fi%
    \ifnum\wd5>0\relax\vskip0mm\hangindent4.5mm\advance\hangindent by #1\hangafter1\hskip#1\global\Podpunkt=102\pp \unhbox5~~\kratkw\relax\fi%
    \ifnum\wd6>0\relax\vskip0mm\hangindent4.5mm\advance\hangindent by #1\hangafter1\hskip#1\global\Podpunkt=103\pp \unhbox6~~\kratkw\relax\fi%
    \ifnum\wd7>0\relax\vskip0mm\hangindent4.5mm\advance\hangindent by #1\hangafter1\hskip#1\global\Podpunkt=104\pp \unhbox7~~\kratkw\relax\fi%
}


\long\def\PodpunktyKrat #1 #2 #3 #4 #5 #6 #7 #8 #9 {
\overfullrule=0pt
      \nobreak
      \mathsurround=0pt
      \setbox0\hbox{#2}
      \setbox1\hbox{#3}
      \setbox2\hbox{#4}
      \setbox3\hbox{#5}
      \setbox4\hbox{#6}
      \setbox5\hbox{#7}
      \setbox6\hbox{#8}
      \setbox7\hbox{#9}
    \ifnum\wd0>0\relax\vskip0mm\hangindent4.5mm\advance\hangindent by #1\hangafter1\hskip#1\nkratkw\hbox to \pindent{A.} \unhbox0\relax\fi%
    \ifnum\wd1>0\relax\vskip0mm\hangindent4.5mm\advance\hangindent by #1\hangafter1\hskip#1\nkratkw\hbox to \pindent{B.} \unhbox1\relax\fi%
    \ifnum\wd2>0\relax\vskip0mm\hangindent4.5mm\advance\hangindent by #1\hangafter1\hskip#1\nkratkw\hbox to \pindent{C.} \unhbox2\relax\fi%
    \ifnum\wd3>0\relax\vskip0mm\hangindent4.5mm\advance\hangindent by #1\hangafter1\hskip#1\nkratkw\hbox to \pindent{D.} \unhbox3\relax\fi%
    \ifnum\wd4>0\relax\vskip0mm\hangindent4.5mm\advance\hangindent by #1\hangafter1\hskip#1\nkratkw\hbox to \pindent{E.} \unhbox4\relax\fi%
    \ifnum\wd5>0\relax\vskip0mm\hangindent4.5mm\advance\hangindent by #1\hangafter1\hskip#1\nkratkw\hbox to \pindent{F.} \unhbox5\relax\fi%
    \ifnum\wd6>0\relax\vskip0mm\hangindent4.5mm\advance\hangindent by #1\hangafter1\hskip#1\nkratkw\hbox to \pindent{G.} \unhbox6\relax\fi%
    \ifnum\wd7>0\relax\vskip0mm\hangindent4.5mm\advance\hangindent by #1\hangafter1\hskip#1\nkratkw\hbox to \pindent{H.} \unhbox7\relax\fi%
}

\long\def\numery #1 #2 #3 #4 #5 #6 #7 #8 #9 {
\overfullrule=0pt
      \nobreak
      \mathsurround=0pt
      \setbox0\hbox{#2}
      \setbox1\hbox{#3}
      \setbox2\hbox{#4}
      \setbox3\hbox{#5}
      \setbox4\hbox{#6}
      \setbox5\hbox{#7}
      \setbox6\hbox{#8}
      \setbox7\hbox{#9}
    \ifnum\wd0>0\relax\vskip0mm\hangindent4.5mm\advance\hangindent by #1\hangafter1\hskip#1\hbox to \pindent{1.} \unhbox0\relax\fi%
    \ifnum\wd1>0\relax\vskip0mm\hangindent4.5mm\advance\hangindent by #1\hangafter1\hskip#1\hbox to \pindent{2.} \unhbox1\relax\fi%
    \ifnum\wd2>0\relax\vskip0mm\hangindent4.5mm\advance\hangindent by #1\hangafter1\hskip#1\hbox to \pindent{3.} \unhbox2\relax\fi%
    \ifnum\wd3>0\relax\vskip0mm\hangindent4.5mm\advance\hangindent by #1\hangafter1\hskip#1\hbox to \pindent{4.} \unhbox3\relax\fi%
    \ifnum\wd4>0\relax\vskip0mm\hangindent4.5mm\advance\hangindent by #1\hangafter1\hskip#1\hbox to \pindent{5.} \unhbox4\relax\fi%
    \ifnum\wd5>0\relax\vskip0mm\hangindent4.5mm\advance\hangindent by #1\hangafter1\hskip#1\hbox to \pindent{6.} \unhbox5\relax\fi%
    \ifnum\wd6>0\relax\vskip0mm\hangindent4.5mm\advance\hangindent by #1\hangafter1\hskip#1\hbox to \pindent{7.} \unhbox6\relax\fi%
    \ifnum\wd7>0\relax\vskip0mm\hangindent4.5mm\advance\hangindent by #1\hangafter1\hskip#1\hbox to \pindent{8.} \unhbox7\relax\fi%
}

\long\def\uzasadnijkropki #1 #2 #3 #4 #5 #6 #7 #8 #9 {%
\overfullrule=0pt%
      \SPR=\hsize%
      \advance\SPR by -\pindent%
      \advance\SPR by -#1%
      \nobreak%
      \mathsurround=0pt%
      \setbox0\hbox{#2}%
      \setbox1\hbox{#3}%
      \setbox2\hbox{#4}%
      \setbox3\hbox{#5}%
      \setbox4\hbox{#6}%
      \setbox5\hbox{#7}%
      \setbox6\hbox{#8}%
      \setbox7\hbox{#9}%
    \ifnum\wd0>0\relax\vskip1mm\hangindent\pindent\advance\hangindent by #1\hangafter1\hskip#1\hbox to \pindent{1.\hss}\unhbox0\vskip1.5mm\hfill\hbox to \SPR{\dotfill}\relax\fi%
    \ifnum\wd1>0\relax\vskip1mm\hangindent\pindent\advance\hangindent by #1\hangafter1\hskip#1\hbox to \pindent{2.\hss}\unhbox1\vskip1.5mm\hfill\hbox to \SPR{\dotfill}\relax\fi%
    \ifnum\wd2>0\relax\vskip1mm\hangindent\pindent\advance\hangindent by #1\hangafter1\hskip#1\hbox to \pindent{3.\hss}\unhbox2\vskip1.5mm\hfill\hbox to \SPR{\dotfill}\relax\fi%
    \ifnum\wd3>0\relax\vskip1mm\hangindent\pindent\advance\hangindent by #1\hangafter1\hskip#1\hbox to \pindent{4.\hss}\unhbox3\vskip1.5mm\hfill\hbox to \SPR{\dotfill}\relax\fi%
    \ifnum\wd4>0\relax\vskip1mm\hangindent\pindent\advance\hangindent by #1\hangafter1\hskip#1\hbox to \pindent{5.\hss}\unhbox4\vskip1.5mm\hfill\hbox to \SPR{\dotfill}\relax\fi%
    \ifnum\wd5>0\relax\vskip1mm\hangindent\pindent\advance\hangindent by #1\hangafter1\hskip#1\hbox to \pindent{6.\hss}\unhbox5\vskip1.5mm\hfill\hbox to \SPR{\dotfill}\relax\fi%
    \ifnum\wd6>0\relax\vskip1mm\hangindent\pindent\advance\hangindent by #1\hangafter1\hskip#1\hbox to \pindent{7.\hss}\unhbox6\vskip1.5mm\hfill\hbox to \SPR{\dotfill}\relax\fi%
    \ifnum\wd7>0\relax\vskip1mm\hangindent\pindent\advance\hangindent by #1\hangafter1\hskip#1\hbox to \pindent{8.\hss}\unhbox7\vskip1.5mm\hfill\hbox to \SPR{\dotfill}\relax\fi%
}

\long\def\uzasadnijkropkipp #1 #2 #3 #4 #5 #6 #7 #8 #9 {%
\overfullrule=0pt%
      \SPR=\hsize%
      \advance\SPR by -\pindent%
      \advance\SPR by -#1%
      \nobreak%
      \mathsurround=0pt%
      \setbox0\hbox{#2}%
      \setbox1\hbox{#3}%
      \setbox2\hbox{#4}%
      \setbox3\hbox{#5}%
      \setbox4\hbox{#6}%
      \setbox5\hbox{#7}%
      \setbox6\hbox{#8}%
      \setbox7\hbox{#9}%
    \ifnum\wd0>0\relax\vskip1mm\hangindent\pindent\advance\hangindent by #1\hangafter1\hskip#1\hbox to \pindent{a)\hss}\unhbox0\vskip1.5mm\hfill\hbox to \SPR{\dotfill}\relax\fi%
    \ifnum\wd1>0\relax\vskip1mm\hangindent\pindent\advance\hangindent by #1\hangafter1\hskip#1\hbox to \pindent{b)\hss}\unhbox1\vskip1.5mm\hfill\hbox to \SPR{\dotfill}\relax\fi%
    \ifnum\wd2>0\relax\vskip1mm\hangindent\pindent\advance\hangindent by #1\hangafter1\hskip#1\hbox to \pindent{c)\hss}\unhbox2\vskip1.5mm\hfill\hbox to \SPR{\dotfill}\relax\fi%
    \ifnum\wd3>0\relax\vskip1mm\hangindent\pindent\advance\hangindent by #1\hangafter1\hskip#1\hbox to \pindent{d)\hss}\unhbox3\vskip1.5mm\hfill\hbox to \SPR{\dotfill}\relax\fi%
    \ifnum\wd4>0\relax\vskip1mm\hangindent\pindent\advance\hangindent by #1\hangafter1\hskip#1\hbox to \pindent{e)\hss}\unhbox4\vskip1.5mm\hfill\hbox to \SPR{\dotfill}\relax\fi%
    \ifnum\wd5>0\relax\vskip1mm\hangindent\pindent\advance\hangindent by #1\hangafter1\hskip#1\hbox to \pindent{f)\hss}\unhbox5\vskip1.5mm\hfill\hbox to \SPR{\dotfill}\relax\fi%
    \ifnum\wd6>0\relax\vskip1mm\hangindent\pindent\advance\hangindent by #1\hangafter1\hskip#1\hbox to \pindent{g)\hss}\unhbox6\vskip1.5mm\hfill\hbox to \SPR{\dotfill}\relax\fi%
    \ifnum\wd7>0\relax\vskip1mm\hangindent\pindent\advance\hangindent by #1\hangafter1\hskip#1\hbox to \pindent{h)\hss}\unhbox7\vskip1.5mm\hfill\hbox to \SPR{\dotfill}\relax\fi%
}

\long\def\ulozZdanie #1 #2 #3 #4 #5 #6 #7 #8 #9 {
\overfullrule=0pt
      \vskip0mm
      \nobreak
      \mathsurround=0pt
      \setbox0\hbox{#1}
      \setbox1\hbox{#2}
      \setbox2\hbox{#3}
      \setbox3\hbox{#4}
      \setbox4\hbox{#5}
      \setbox5\hbox{#6}
      \setbox6\hbox{#7}
      \setbox7\hbox{#8}
      \setbox8\hbox{#9}
    \ifnum\wd0>0\relax\vskip4.5mm\st 4 0 \unhbox0\vskip2mm\Kr{16.5}\vskip2mm\Kr{16.5}\relax\fi%
    \ifnum\wd1>0\relax\vskip4.5mm\st 4 0 \unhbox1\vskip2mm\Kr{16.5}\vskip2mm\Kr{16.5}\relax\fi%
    \ifnum\wd2>0\relax\vskip4.5mm\st 4 0 \unhbox2\vskip2mm\Kr{16.5}\vskip2mm\Kr{16.5}\relax\fi%
    \ifnum\wd3>0\relax\vskip4.5mm\st 4 0 \unhbox3\vskip2mm\Kr{16.5}\vskip2mm\Kr{16.5}\relax\fi%
    \ifnum\wd4>0\relax\vskip4.5mm\st 4 0 \unhbox4\vskip2mm\Kr{16.5}\vskip2mm\Kr{16.5}\relax\fi%
    \ifnum\wd5>0\relax\vskip4.5mm\st 4 0 \unhbox5\vskip2mm\Kr{16.5}\vskip2mm\Kr{16.5}\relax\fi%
    \ifnum\wd6>0\relax\vskip4.5mm\st 4 0 \unhbox6\vskip2mm\Kr{16.5}\vskip2mm\Kr{16.5}\relax\fi%
    \ifnum\wd7>0\relax\vskip4.5mm\st 4 0 \unhbox7\vskip2mm\Kr{16.5}\vskip2mm\Kr{16.5}\relax\fi%
    \ifnum\wd8>0\relax\vskip4.5mm\st 4 0 \unhbox8\vskip2mm\Kr{16.5}\vskip2mm\Kr{16.5}\relax\fi%
}

\long\def\krotkiejOdpowiedzi #1 #2 #3 #4 #5 #6 #7 #8 #9 {
\overfullrule=0pt
      \SPR=\hsize
      \advance\SPR by -#1
      \vskip0mm
      \nobreak
      \mathsurround=0pt
      \setbox0\hbox{#2}
      \setbox1\hbox{#3}
      \setbox2\hbox{#4}
      \setbox3\hbox{#5}
      \setbox4\hbox{#6}
      \setbox5\hbox{#7}
      \setbox6\hbox{#8}
      \setbox7\hbox{#9}
    \ifnum\wd0>0\relax\hbox to \SPR{\unhbox0\dotfill}\relax\fi%
    \ifnum\wd1>0\relax\hbox to \SPR{\unhbox1\dotfill}\relax\fi%
    \ifnum\wd2>0\relax\hbox to \SPR{\unhbox2\dotfill}\relax\fi%
    \ifnum\wd3>0\relax\hbox to \SPR{\unhbox3\dotfill}\relax\fi%
    \ifnum\wd4>0\relax\hbox to \SPR{\unhbox4\dotfill}\relax\fi%
    \ifnum\wd5>0\relax\hbox to \SPR{\unhbox5\dotfill}\relax\fi%
    \ifnum\wd6>0\relax\hbox to \SPR{\unhbox6\dotfill}\relax\fi%
    \ifnum\wd7>0\relax\hbox to \SPR{\unhbox7\dotfill}\relax\fi%
}

\long\def\krotkiejOdpowiedzipp #1 #2 #3 #4 #5 #6 #7 #8 #9 {
\overfullrule=0pt
      \SPR=\hsize
      \advance\SPR by -#1
      \vskip0mm
      \nobreak
      \mathsurround=0pt
      \setbox0\hbox{#2}
      \setbox1\hbox{#3}
      \setbox2\hbox{#4}
      \setbox3\hbox{#5}
      \setbox4\hbox{#6}
      \setbox5\hbox{#7}
      \setbox6\hbox{#8}
      \setbox7\hbox{#9}
    \ifnum\wd0>0\relax\hbox to \SPR{\pp\unhbox0\dotfill}\relax\fi%
    \ifnum\wd1>0\relax\hbox to \SPR{\pp\unhbox1\dotfill}\relax\fi%
    \ifnum\wd2>0\relax\hbox to \SPR{\pp\unhbox2\dotfill}\relax\fi%
    \ifnum\wd3>0\relax\hbox to \SPR{\pp\unhbox3\dotfill}\relax\fi%
    \ifnum\wd4>0\relax\hbox to \SPR{\pp\unhbox4\dotfill}\relax\fi%
    \ifnum\wd5>0\relax\hbox to \SPR{\pp\unhbox5\dotfill}\relax\fi%
    \ifnum\wd6>0\relax\hbox to \SPR{\pp\unhbox6\dotfill}\relax\fi%
    \ifnum\wd7>0\relax\hbox to \SPR{\pp\unhbox7\dotfill}\relax\fi%
}

\long\def\cyfryNaKratkach #1 #2 #3 #4 #5 #6 #7 #8 #9 {
\overfullrule=0pt
      \nobreak
      \mathsurround=0pt
      \setbox0\hbox{#1}
      \setbox1\hbox{#2}
      \setbox2\hbox{#3}
      \setbox3\hbox{#4}
      \setbox4\hbox{#5}
      \setbox5\hbox{#6}
      \setbox6\hbox{#7}
      \setbox7\hbox{#8}
      \setbox8\hbox{#9}
    \ifnum\wd0>0\relax\vskip1mm\hangindent6.9mm\nkratkw~\unhbox0\relax\fi%
    \ifnum\wd1>0\relax\vskip1mm\hangindent6.9mm\nkratkw~\unhbox1\relax\fi%
    \ifnum\wd2>0\relax\vskip1mm\hangindent6.9mm\nkratkw~\unhbox2\relax\fi%
    \ifnum\wd3>0\relax\vskip1mm\hangindent6.9mm\nkratkw~\unhbox3\relax\fi%
    \ifnum\wd4>0\relax\vskip1mm\hangindent6.9mm\nkratkw~\unhbox4\relax\fi%
    \ifnum\wd5>0\relax\vskip1mm\hangindent6.9mm\nkratkw~\unhbox5\relax\fi%
    \ifnum\wd6>0\relax\vskip1mm\hangindent6.9mm\nkratkw~\unhbox6\relax\fi%
    \ifnum\wd7>0\relax\vskip1mm\hangindent6.9mm\nkratkw~\unhbox7\relax\fi%
    \ifnum\wd8>0\relax\vskip1mm\hangindent6.9mm\nkratkw~\unhbox8\relax\fi%
}


\long\def\Przypisy #1 #2 #3 #4 #5 #6 #7 #8 #9 {
\overfullrule=0pt
      \vskip0mm
      \global\Przypis=1
      \nobreak
      \mathsurround=0pt
      \setbox0\hbox{#1}
      \setbox1\hbox{#2}
      \setbox2\hbox{#3}
      \setbox3\hbox{#4}
      \setbox4\hbox{#5}
      \setbox5\hbox{#6}
      \setbox6\hbox{#7}
      \setbox7\hbox{#8}
      \setbox8\hbox{#9}
    \ifnum\wd0>0\relax\hangindent2.7mm\hangafter1\st 4 0 \hbox to2.7mm{\ppr\hss}\unhbox0\vskip0mm\relax\fi%
    \ifnum\wd1>0\relax\hangindent2.7mm\hangafter1\st 4 0 \hbox to2.7mm{\ppr\hss}\unhbox1\vskip0mm\relax\fi%
    \ifnum\wd2>0\relax\hangindent2.7mm\hangafter1\st 4 0 \hbox to2.7mm{\ppr\hss}\unhbox2\vskip0mm\relax\fi%
    \ifnum\wd3>0\relax\hangindent2.7mm\hangafter1\st 4 0 \hbox to2.7mm{\ppr\hss}\unhbox3\vskip0mm\relax\fi%
    \ifnum\wd4>0\relax\hangindent2.7mm\hangafter1\st 4 0 \hbox to2.7mm{\ppr\hss}\unhbox4\vskip0mm\relax\fi%
    \ifnum\wd5>0\relax\hangindent2.7mm\hangafter1\st 4 0 \hbox to2.7mm{\ppr\hss}\unhbox5\vskip0mm\relax\fi%
    \ifnum\wd6>0\relax\hangindent2.7mm\hangafter1\st 4 0 \hbox to2.7mm{\ppr\hss}\unhbox6\vskip0mm\relax\fi%
    \ifnum\wd7>0\relax\hangindent2.7mm\hangafter1\st 4 0 \hbox to2.7mm{\ppr\hss}\unhbox7\vskip0mm\relax\fi%
    \ifnum\wd8>0\relax\hangindent2.7mm\hangafter1\st 4 0 \hbox to2.7mm{\ppr\hss}\unhbox8\vskip0mm\relax\fi%
}

\long\def\odppodpunkty #1 #2 #3 #4 #5 #6 #7 #8 {%
\overfullrule=0pt%
      \SPR=0pt%
      \leavevmode%
      \mathsurround=0pt%
      \setbox0\hbox{#1}%
      \setbox1\hbox{#2}%
      \setbox2\hbox{#3}%
      \setbox3\hbox{#4}%
      \setbox4\hbox{#5}%
      \setbox5\hbox{#6}%
      \setbox6\hbox{#7}%
      \setbox7\hbox{#8}%
      \ifdim\wd0>23pt\relax\advance\SPR by \wd0\relax\else\advance\SPR by 0pt\fi%
      \ifdim\wd1>23pt\relax\advance\SPR by \wd1\relax\else\advance\SPR by 0pt\fi%
      \ifdim\wd2>23pt\relax\advance\SPR by \wd2\relax\else\advance\SPR by 0pt\fi%
      \ifdim\wd3>23pt\relax\advance\SPR by \wd3\relax\else\advance\SPR by 0pt\fi%
      \ifdim\wd4>23pt\relax\advance\SPR by \wd4\relax\else\advance\SPR by 0pt\fi%
      \ifdim\wd5>23pt\relax\advance\SPR by \wd5\relax\else\advance\SPR by 0pt\fi%
      \ifdim\wd6>23pt\relax\advance\SPR by \wd6\relax\else\advance\SPR by 0pt\fi%
      \ifdim\wd7>23pt\relax\advance\SPR by \wd7\relax\else\advance\SPR by 0pt\fi%
      \ifdim\SPR=0pt\relax%
        \ifnum\wd0>0\relax\global\Podpunkt=97\ppp \box0\relax\fi%
        \ifnum\wd1>0\relax,\hskip3mm plus.5mm minus.5mm\global\Podpunkt=98\pp\box1\relax\fi%
        \ifnum\wd2>0\relax,\hskip3mm plus.5mm minus.5mm\global\Podpunkt=99\pp\box2\relax\fi%
        \ifnum\wd3>0\relax,\hskip3mm plus.5mm minus.5mm\global\Podpunkt=100\pp\box3\relax\fi%
        \ifnum\wd4>0\relax,\hskip3mm plus.5mm minus.5mm\global\Podpunkt=101\pp\box4\relax\fi%
        \ifnum\wd5>0\relax,\hskip3mm plus.5mm minus.5mm\global\Podpunkt=102\pp\box5\relax\fi%
        \ifnum\wd6>0\relax,\hskip3mm plus.5mm minus.5mm\global\Podpunkt=103\pp\box6\relax\fi%
        \ifnum\wd7>0\relax,\hskip3mm plus.5mm minus.5mm\global\Podpunkt=104\pp\box7\relax\fi%
      \else\relax%
        \ifnum\wd0>0\relax\global\Podpunkt=97\ppp \unhbox0\relax\fi%
        \ifnum\wd1>0\relax\vskip0mm\global\Podpunkt=98\pp \unhbox1\relax\fi%
        \ifnum\wd2>0\relax\vskip0mm\global\Podpunkt=99\pp \unhbox2\relax\fi%
        \ifnum\wd3>0\relax\vskip0mm\global\Podpunkt=100\pp \unhbox3\relax\fi%
        \ifnum\wd4>0\relax\vskip0mm\global\Podpunkt=101\pp \unhbox4\relax\fi%
        \ifnum\wd5>0\relax\vskip0mm\global\Podpunkt=102\pp \unhbox5\relax\fi%
        \ifnum\wd6>0\relax\vskip0mm\global\Podpunkt=103\pp \unhbox6\relax\fi%
        \ifnum\wd7>0\relax\vskip0mm\global\Podpunkt=104\pp \unhbox7\relax\fi%
      \fi%
}

\long\def\odpnumery #1 #2 #3 #4 #5 #6 #7 #8 {%
\overfullrule=0pt%
      \SPR=0pt%
      \leavevmode%
      \mathsurround=0pt%
      \setbox0\hbox{#1}%
      \setbox1\hbox{#2}%
      \setbox2\hbox{#3}%
      \setbox3\hbox{#4}%
      \setbox4\hbox{#5}%
      \setbox5\hbox{#6}%
      \setbox6\hbox{#7}%
      \setbox7\hbox{#8}%
      \ifdim\wd0>23pt\relax\advance\SPR by \wd0\relax\else\advance\SPR by 0pt\fi%
      \ifdim\wd1>23pt\relax\advance\SPR by \wd1\relax\else\advance\SPR by 0pt\fi%
      \ifdim\wd2>23pt\relax\advance\SPR by \wd2\relax\else\advance\SPR by 0pt\fi%
      \ifdim\wd3>23pt\relax\advance\SPR by \wd3\relax\else\advance\SPR by 0pt\fi%
      \ifdim\wd4>23pt\relax\advance\SPR by \wd4\relax\else\advance\SPR by 0pt\fi%
      \ifdim\wd5>23pt\relax\advance\SPR by \wd5\relax\else\advance\SPR by 0pt\fi%
      \ifdim\wd6>23pt\relax\advance\SPR by \wd6\relax\else\advance\SPR by 0pt\fi%
      \ifdim\wd7>23pt\relax\advance\SPR by \wd7\relax\else\advance\SPR by 0pt\fi%
      \ifdim\SPR=0pt\relax%
        \ifnum\wd0>0\relax\global\Numer=1\Nr\box0\relax\fi%
        \ifnum\wd1>0\relax\hskip3mm plus.5mm minus.5mm\global\Numer=2\Nr\box1\relax\fi%
        \ifnum\wd2>0\relax\hskip3mm plus.5mm minus.5mm\global\Numer=3\Nr\box2\relax\fi%
        \ifnum\wd3>0\relax\hskip3mm plus.5mm minus.5mm\global\Numer=4\Nr\box3\relax\fi%
        \ifnum\wd4>0\relax\hskip3mm plus.5mm minus.5mm\global\Numer=5\Nr\box4\relax\fi%
        \ifnum\wd5>0\relax\hskip3mm plus.5mm minus.5mm\global\Numer=6\Nr\box5\relax\fi%
        \ifnum\wd6>0\relax\hskip3mm plus.5mm minus.5mm\global\Numer=7\Nr\box6\relax\fi%
        \ifnum\wd7>0\relax\hskip3mm plus.5mm minus.5mm\global\Numer=8\Nr\box7\relax\fi%
      \else\relax%
        \ifnum\wd0>0\relax\global\Numer=1\Nr \unhbox0\relax\fi%
        \ifnum\wd1>0\relax\vskip0mm\global\Numer=2\Nr \unhbox1\relax\fi%
        \ifnum\wd2>0\relax\vskip0mm\global\Numer=3\Nr \unhbox2\relax\fi%
        \ifnum\wd3>0\relax\vskip0mm\global\Numer=4\Nr \unhbox3\relax\fi%
        \ifnum\wd4>0\relax\vskip0mm\global\Numer=5\Nr \unhbox4\relax\fi%
        \ifnum\wd5>0\relax\vskip0mm\global\Numer=6\Nr \unhbox5\relax\fi%
        \ifnum\wd6>0\relax\vskip0mm\global\Numer=7\Nr \unhbox6\relax\fi%
        \ifnum\wd7>0\relax\vskip0mm\global\Numer=8\Nr \unhbox7\relax\fi%
      \fi%
}

\def\WWABCDtab #1 #2 #3 #4 #5 #6 #7 #8 #9 {%
\overfullrule=0pt%
      \vskip0mm%
      \nobreak%
      \mathsurround=0pt%
      \SPR=0pt%
      \setbox0\hbox{#1}%                          %pudełkowanie
      \setbox1\hbox{#2}%
      \setbox2\hbox{#3}%
      \setbox3\hbox{#4}%
      \setbox4\hbox{#5}%
      \setbox5\hbox{#6}%
      \setbox6\hbox{#7}%
      \setbox7\hbox{#8}%
      \setbox8\hbox{#9}%
      \ifdim\wd0>435pt\relax\advance\pfCount by 1\relax\fi%
      \ifdim\wd1>435pt\relax\advance\pfCount by 1\relax\fi%
      \ifdim\wd2>435pt\relax\advance\pfCount by 1\relax\fi%
      \ifdim\wd3>435pt\relax\advance\pfCount by 1\relax\fi%
      \ifdim\wd4>435pt\relax\advance\pfCount by 1\relax\fi%
      \ifdim\wd5>435pt\relax\advance\pfCount by 1\relax\fi%
      \ifdim\wd6>435pt\relax\advance\pfCount by 1\relax\fi%
      \ifdim\wd7>435pt\relax\advance\pfCount by 1\relax\fi%
      \ifdim\wd8>435pt\relax\advance\pfCount by 1\relax\fi%
      \ifnum\pfCount=0
          {\setbox\strutbox=\hbox{\vrule height12pt depth5pt width0pt}%
          \offinterlineskip%
          \halign{%
          \vrule\strut\hskip5pt##\hskip3pt&%
          \vrule\hskip5pt##\hskip5pt\hfill\vrule\cr
          \noalign{\hrule}%
          1.&\unhbox0\cr\noalign{\hrule}%
          \ifnum\wd1>0\relax2.&\unhbox1\cr\noalign{\hrule}\fi%
          \ifnum\wd2>0\relax3.&\unhbox2\cr\noalign{\hrule}\fi%
          \ifnum\wd3>0\relax4.&\unhbox3\cr\noalign{\hrule}\fi%
          \ifnum\wd4>0\relax5.&\unhbox4\cr\noalign{\hrule}\fi%
          \ifnum\wd5>0\relax6.&\unhbox5\cr\noalign{\hrule}\fi%
          \ifnum\wd6>0\relax7.&\unhbox6\cr\noalign{\hrule}\fi%
          \ifnum\wd7>0\relax8.&\unhbox7\cr\noalign{\hrule}\fi%
          \ifnum\wd8>0\relax9.&\unhbox8\cr\noalign{\hrule}\fi%
          }}%
      \else
          {\setbox\strutbox=\hbox{\vrule height12pt depth 5pt width0pt}%
          \offinterlineskip%
          \halign to \hsize{\strut%
          \vrule\hskip5pt\vtop{\hsize=9.6pt##\hfill}\hskip2pt&%
          \vrule\hskip5pt\vtop{\hsize=437.2pt\lineskip3pt##\vrule height0pt depth5pt width0pt\hfill}\quad\vrule\cr%
          \noalign{\hrule}%
          1.&\unhbox0\cr\noalign{\hrule}%
          \ifnum\wd1>0\relax2.&\unhbox1\cr\noalign{\hrule}\fi%
          \ifnum\wd2>0\relax3.&\unhbox2\cr\noalign{\hrule}\fi%
          \ifnum\wd3>0\relax4.&\unhbox3\cr\noalign{\hrule}\fi%
          \ifnum\wd4>0\relax5.&\unhbox4\cr\noalign{\hrule}\fi%
          \ifnum\wd5>0\relax6.&\unhbox5\cr\noalign{\hrule}\fi%
          \ifnum\wd6>0\relax7.&\unhbox6\cr\noalign{\hrule}\fi%
          \ifnum\wd7>0\relax8.&\unhbox7\cr\noalign{\hrule}\fi%
          \ifnum\wd8>0\relax9.&\unhbox8\cr\noalign{\hrule}\fi%
          }}\fi%
}

\def\WWABCDtabLit #1 #2 #3 #4 #5 #6 #7 #8 #9 {%
\overfullrule=0pt%
      \vskip0mm%
      \nobreak%
      \mathsurround=0pt%
      \SPR=0pt%
      \setbox0\hbox{#1}%                          %pudełkowanie
      \setbox1\hbox{#2}%
      \setbox2\hbox{#3}%
      \setbox3\hbox{#4}%
      \setbox4\hbox{#5}%
      \setbox5\hbox{#6}%
      \setbox6\hbox{#7}%
      \setbox7\hbox{#8}%
      \setbox8\hbox{#9}%
      \ifdim\wd0>435pt\relax\advance\pfCount by 1\relax\fi%
      \ifdim\wd1>435pt\relax\advance\pfCount by 1\relax\fi%
      \ifdim\wd2>435pt\relax\advance\pfCount by 1\relax\fi%
      \ifdim\wd3>435pt\relax\advance\pfCount by 1\relax\fi%
      \ifdim\wd4>435pt\relax\advance\pfCount by 1\relax\fi%
      \ifdim\wd5>435pt\relax\advance\pfCount by 1\relax\fi%
      \ifdim\wd6>435pt\relax\advance\pfCount by 1\relax\fi%
      \ifdim\wd7>435pt\relax\advance\pfCount by 1\relax\fi%
      \ifdim\wd8>435pt\relax\advance\pfCount by 1\relax\fi%
      \ifnum\pfCount=0
          {\setbox\strutbox=\hbox{\vrule height12pt depth5pt width0pt}%
          \offinterlineskip%
          \halign{%
          \vrule\strut\hskip5pt##\hskip3pt&%
          \vrule\hskip5pt##\hskip5pt\hfill\vrule\cr
          \noalign{\hrule}%
          A.&\unhbox0\cr\noalign{\hrule}%
          \ifnum\wd1>0\relax B.&\unhbox1\cr\noalign{\hrule}\fi%
          \ifnum\wd2>0\relax C.&\unhbox2\cr\noalign{\hrule}\fi%
          \ifnum\wd3>0\relax D.&\unhbox3\cr\noalign{\hrule}\fi%
          \ifnum\wd4>0\relax E.&\unhbox4\cr\noalign{\hrule}\fi%
          \ifnum\wd5>0\relax F.&\unhbox5\cr\noalign{\hrule}\fi%
          \ifnum\wd6>0\relax G.&\unhbox6\cr\noalign{\hrule}\fi%
          \ifnum\wd7>0\relax H.&\unhbox7\cr\noalign{\hrule}\fi%
          \ifnum\wd8>0\relax I.&\unhbox8\cr\noalign{\hrule}\fi%
          }}%
      \else
          {\setbox\strutbox=\hbox{\vrule height12pt depth 5pt width0pt}%
          \offinterlineskip%
          \halign to \hsize{\strut%
          \vrule\hskip5pt\vtop{\hsize=9.6pt##\hfill}\hskip2pt&%
          \vrule\hskip5pt\vtop{\hsize=437.2pt\lineskip3pt##\vrule height0pt depth5pt width0pt\hfill}\quad\vrule\cr%
          \noalign{\hrule}%
          1.&\unhbox0\cr\noalign{\hrule}%
          \ifnum\wd1>0\relax2.&\unhbox1\cr\noalign{\hrule}\fi%
          \ifnum\wd2>0\relax3.&\unhbox2\cr\noalign{\hrule}\fi%
          \ifnum\wd3>0\relax4.&\unhbox3\cr\noalign{\hrule}\fi%
          \ifnum\wd4>0\relax5.&\unhbox4\cr\noalign{\hrule}\fi%
          \ifnum\wd5>0\relax6.&\unhbox5\cr\noalign{\hrule}\fi%
          \ifnum\wd6>0\relax7.&\unhbox6\cr\noalign{\hrule}\fi%
          \ifnum\wd7>0\relax8.&\unhbox7\cr\noalign{\hrule}\fi%
          \ifnum\wd8>0\relax9.&\unhbox8\cr\noalign{\hrule}\fi%
          }}\fi%
}

\def\tabwypelniany #1 #2 #3 #4 #5 #6 #7 #8 {%
\overfullrule=0pt%
      \vskip0mm%
      \nobreak%
      \mathsurround=0pt%
      \wysI=0pt%
      \wysMAX=0pt%
      \pfCount=0%
      \setbox0\vbox{\hsize=429pt #1}%                         %pudełkowanie
      \setbox1\vbox{\hsize=429pt #2}%
      \setbox2\vbox{\hsize=429pt #3}%
      \setbox3\vbox{\hsize=429pt #4}%
      \setbox4\vbox{\hsize=429pt #5}%
      \setbox5\vbox{\hsize=429pt #6}%
      \setbox6\vbox{\hsize=429pt #7}%
      \setbox7\vbox{\hsize=429pt #8}%
      \ifnum\ht2>\ht0\relax\ht0=\ht2\relax\fi%   %spr wielkości par 1/3, 2/4
      \ifnum\ht3>\ht1\relax\ht1=\ht3\relax\fi%
      \ifnum\ht6>\ht4\relax\ht4=\ht6\relax\fi%   %spr wielkości par 5/7, 6/8
      \ifnum\ht7>\ht5\relax\ht5=\ht7\relax\fi%
      \ifnum\ht4>\ht0\relax\ht0=\ht4\relax\fi%   %spr wielkości par 1/5, 2/6
      \ifnum\ht5>\ht1\relax\ht1=\ht5\relax\fi%
      \ifnum\ht0>\ht1\relax\wysMAX=\ht0\relax%
            \else\wysMAX=\ht1\relax\fi %spr najwyższego pudełka
      \ifnum\wd0>0\relax\pfCount=0\relax\fi%
      \ifnum\wd1>0\relax\advance\pfCount by 1\relax\fi%
      \ifnum\wd2>0\relax\advance\pfCount by 1\relax\fi%
      \ifnum\wd3>0\relax\advance\pfCount by 1\relax\fi%
      \ifnum\wd4>0\relax\advance\pfCount by 1\relax\fi%
      \ifnum\wd5>0\relax\advance\pfCount by 1\relax\fi%
      \ifnum\wd6>0\relax\advance\pfCount by 1\relax\fi%
      \ifnum\wd7>0\relax\advance\pfCount by 1\relax\fi%
      \wysI=0.5\baselineskip%
      \wysII=\baselineskip\advance\wysII by \wysMAX%
      \multirput(0pt,-\wysII)(0,-\wysII){\pfCount}{\psline[linewidth=.425pt](0,0)(469.5pt,0)}%
      \ifnum\wd0>0\relax%
      \rput[tl](6pt,-\wysI){\box0}\relax\fi%
      \ifnum\wd1>0\relax%
      \advance\wysI by \wysMAX%
      \advance\wysI by 1\baselineskip%
      \rput[tl](6pt,-\wysI){\box1}\relax\fi%
      \ifnum\wd2>0\relax%
      \advance\wysI by \wysMAX%
      \advance\wysI by 1\baselineskip%
      \rput[tl](6pt,-\wysI){\box2}\relax\fi%
      \ifnum\wd3>0\relax%
      \advance\wysI by \wysMAX%
      \advance\wysI by 1\baselineskip%
      \rput[tl](6pt,-\wysI){\box3}\relax\fi%
      \ifnum\wd4>0\relax%
      \advance\wysI by \wysMAX%
      \advance\wysI by 1\baselineskip%
      \rput[tl](6pt,-\wysI){\box4}\relax\fi%
      \ifnum\wd5>0\relax%
      \advance\wysI by \wysMAX%
      \advance\wysI by 1\baselineskip%
      \rput[tl](6pt,-\wysI){\box5}\relax\fi%
      \ifnum\wd6>0\relax%
      \advance\wysI by \wysMAX%
      \advance\wysI by 1\baselineskip%
      \rput[tl](6pt,-\wysI){\box6}\relax\fi%
      \ifnum\wd7>0\relax%
      \advance\wysI by \wysMAX%
      \advance\wysI by 1\baselineskip%
      \rput[tl](6pt,-\wysI){\box7}\relax\fi%
      \advance\wysI by \wysMAX%
      \advance\wysI by 0.5\baselineskip%
      \rput[B](446pt,0pt){\psline[linewidth=.425pt](0,0)(0,-\wysI)}%
      \psframe[linewidth=.425pt](0,0pt)(469.5pt,-\wysI)%
      \vskip\wysI%
}

\def\pf #1 #2 #3 #4 #5 #6 #7 #8 {%
\overfullrule=0pt%
      \vskip0mm%
      \nobreak%
      \mathsurround=0pt%
      \wysI=0pt%
      \wysMAX=0pt%
      \pfCount=0%
      \setbox0\vbox{\hsize=406pt #1}%                          %pudełkowanie
      \setbox1\vbox{\hsize=406pt #2}%
      \setbox2\vbox{\hsize=406pt #3}%
      \setbox3\vbox{\hsize=406pt #4}%
      \setbox4\vbox{\hsize=406pt #5}%
      \setbox5\vbox{\hsize=406pt #6}%
      \setbox6\vbox{\hsize=406pt #7}%
      \setbox7\vbox{\hsize=406pt #8}%
      \ifnum\ht2>\ht0\relax\ht0=\ht2\relax\fi%   %spr wielkości par 1/3, 2/4
      \ifnum\ht3>\ht1\relax\ht1=\ht3\relax\fi%
      \ifnum\ht6>\ht4\relax\ht4=\ht6\relax\fi%   %spr wielkości par 5/7, 6/8
      \ifnum\ht7>\ht5\relax\ht5=\ht7\relax\fi%
      \ifnum\ht4>\ht0\relax\ht0=\ht4\relax\fi%   %spr wielkości par 1/5, 2/6
      \ifnum\ht5>\ht1\relax\ht1=\ht5\relax\fi%
      \ifnum\ht0>\ht1\relax\wysMAX=\ht0\relax%
            \else\wysMAX=\ht1\relax\fi%          %spr najwyższego pudełka
      \ifnum\wd0>0\relax\pfCount=0\relax\fi%
      \ifnum\wd1>0\relax\advance\pfCount by 1\relax\fi%
      \ifnum\wd2>0\relax\advance\pfCount by 1\relax\fi%
      \ifnum\wd3>0\relax\advance\pfCount by 1\relax\fi%
      \ifnum\wd4>0\relax\advance\pfCount by 1\relax\fi%
      \ifnum\wd5>0\relax\advance\pfCount by 1\relax\fi%
      \ifnum\wd6>0\relax\advance\pfCount by 1\relax\fi%
      \ifnum\wd7>0\relax\advance\pfCount by 1\relax\fi%
      \wysI=0.5\baselineskip%
      \wysII=\baselineskip\advance\wysII by \wysMAX%
      \multirput(0pt,-\wysII)(0,-\wysII){\pfCount}{\psline[linewidth=.425pt](0,0)(469.5pt,0)}%
      \ifnum\wd0>0\relax\rput[tl](6pt,-\wysI){\box0}%
      \rput[t](434.5pt,-\wysI){{\bf P}}%
      \rput[t](457.5pt,-\wysI){{\bf F}}\relax\fi%
      \ifnum\wd1>0\relax\advance\wysI by \wysMAX\advance\wysI by 1\baselineskip%
      \rput[tl](6pt,-\wysI){\box1}%
      \rput[t](434.5pt,-\wysI){{\bf P}}%
      \rput[t](457.5pt,-\wysI){{\bf F}}\relax\fi%
      \ifnum\wd2>0\relax\advance\wysI by \wysMAX\advance\wysI by 1\baselineskip%
      \rput[tl](6pt,-\wysI){\box2}%
      \rput[t](434.5pt,-\wysI){{\bf P}}%
      \rput[t](457.5pt,-\wysI){{\bf F}}\relax\fi%
      \ifnum\wd3>0\relax\advance\wysI by \wysMAX\advance\wysI by 1\baselineskip%
      \rput[tl](6pt,-\wysI){\box3}%
      \rput[t](434.5pt,-\wysI){{\bf P}}%
      \rput[t](457.5pt,-\wysI){{\bf F}}\relax\fi%
      \ifnum\wd4>0\relax\advance\wysI by \wysMAX\advance\wysI by 1\baselineskip%
      \rput[tl](6pt,-\wysI){\box4}%
      \rput[t](434.5pt,-\wysI){{\bf P}}%
      \rput[t](457.5pt,-\wysI){{\bf F}}\relax\fi%
      \ifnum\wd5>0\relax\advance\wysI by \wysMAX\advance\wysI by 1\baselineskip%
      \rput[tl](6pt,-\wysI){\box5}%
      \rput[t](434.5pt,-\wysI){{\bf P}}%
      \rput[t](457.5pt,-\wysI){{\bf F}}\relax\fi%
      \ifnum\wd6>0\relax\advance\wysI by \wysMAX\advance\wysI by 1\baselineskip%
      \rput[tl](6pt,-\wysI){\box6}%
      \rput[t](434.5pt,-\wysI){{\bf P}}%
      \rput[t](457.5pt,-\wysI){{\bf F}}\relax\fi%
      \ifnum\wd7>0\relax\advance\wysI by \wysMAX\advance\wysI by 1\baselineskip%
      \rput[tl](6pt,-\wysI){\box7}%
      \rput[t](434.5pt,-\wysI){{\bf P}}%
      \rput[t](457.5pt,-\wysI){{\bf F}}\relax\fi%
      \advance\wysI by \wysMAX\advance\wysI by 0.5\baselineskip%
      \rput[B](423pt,0pt){\psline[linewidth=.425pt](0,0)(0,-\wysI)}%
      \rput[B](446pt,0pt){\psline[linewidth=.425pt](0,0)(0,-\wysI)}%
      \psframe[linewidth=.425pt](0,0pt)(469.5pt,-\wysI)%
      \vskip\wysI%
}

\def\pn #1 #2 #3 #4 #5 #6 #7 #8 {%
\overfullrule=0pt%
      \vskip0mm%
      \nobreak%
      \mathsurround=0pt%
      \wysI=0pt%
      \wysMAX=0pt%
      \pfCount=0%
      \setbox0\vbox{\hsize=406pt #1}%                          %pudełkowanie
      \setbox1\vbox{\hsize=406pt #2}%
      \setbox2\vbox{\hsize=406pt #3}%
      \setbox3\vbox{\hsize=406pt #4}%
      \setbox4\vbox{\hsize=406pt #5}%
      \setbox5\vbox{\hsize=406pt #6}%
      \setbox6\vbox{\hsize=406pt #7}%
      \setbox7\vbox{\hsize=406pt #8}%
      \ifnum\ht2>\ht0\relax\ht0=\ht2\relax\fi%   %spr wielkości par 1/3, 2/4
      \ifnum\ht3>\ht1\relax\ht1=\ht3\relax\fi%
      \ifnum\ht6>\ht4\relax\ht4=\ht6\relax\fi%   %spr wielkości par 5/7, 6/8
      \ifnum\ht7>\ht5\relax\ht5=\ht7\relax\fi%
      \ifnum\ht4>\ht0\relax\ht0=\ht4\relax\fi%   %spr wielkości par 1/5, 2/6
      \ifnum\ht5>\ht1\relax\ht1=\ht5\relax\fi%
      \ifnum\ht0>\ht1\relax\wysMAX=\ht0\relax%
            \else\wysMAX=\ht1\relax\fi%          %spr najwyższego pudełka
      \ifnum\wd0>0\relax\pfCount=0\relax\fi%
      \ifnum\wd1>0\relax\advance\pfCount by 1\relax\fi%
      \ifnum\wd2>0\relax\advance\pfCount by 1\relax\fi%
      \ifnum\wd3>0\relax\advance\pfCount by 1\relax\fi%
      \ifnum\wd4>0\relax\advance\pfCount by 1\relax\fi%
      \ifnum\wd5>0\relax\advance\pfCount by 1\relax\fi%
      \ifnum\wd6>0\relax\advance\pfCount by 1\relax\fi%
      \ifnum\wd7>0\relax\advance\pfCount by 1\relax\fi%
      \wysI=0.5\baselineskip%
      \wysII=\baselineskip\advance\wysII by \wysMAX%
      \multirput(0pt,-\wysII)(0,-\wysII){\pfCount}{\psline[linewidth=.425pt](0,0)(469.5pt,0)}%
      \ifnum\wd0>0\relax\rput[tl](6pt,-\wysI){\box0}%
      \rput[t](434.5pt,-\wysI){{\bf P}}%
      \rput[t](457.5pt,-\wysI){{\bf N}}\relax\fi%
      \ifnum\wd1>0\relax\advance\wysI by \wysMAX\advance\wysI by 1\baselineskip%
      \rput[tl](6pt,-\wysI){\box1}%
      \rput[t](434.5pt,-\wysI){{\bf P}}%
      \rput[t](457.5pt,-\wysI){{\bf N}}\relax\fi%
      \ifnum\wd2>0\relax\advance\wysI by \wysMAX\advance\wysI by 1\baselineskip%
      \rput[tl](6pt,-\wysI){\box2}%
      \rput[t](434.5pt,-\wysI){{\bf P}}%
      \rput[t](457.5pt,-\wysI){{\bf N}}\relax\fi%
      \ifnum\wd3>0\relax\advance\wysI by \wysMAX\advance\wysI by 1\baselineskip%
      \rput[tl](6pt,-\wysI){\box3}%
      \rput[t](434.5pt,-\wysI){{\bf P}}%
      \rput[t](457.5pt,-\wysI){{\bf N}}\relax\fi%
      \ifnum\wd4>0\relax\advance\wysI by \wysMAX\advance\wysI by 1\baselineskip%
      \rput[tl](6pt,-\wysI){\box4}%
      \rput[t](434.5pt,-\wysI){{\bf P}}%
      \rput[t](457.5pt,-\wysI){{\bf N}}\relax\fi%
      \ifnum\wd5>0\relax\advance\wysI by \wysMAX\advance\wysI by 1\baselineskip%
      \rput[tl](6pt,-\wysI){\box5}%
      \rput[t](434.5pt,-\wysI){{\bf P}}%
      \rput[t](457.5pt,-\wysI){{\bf N}}\relax\fi%
      \ifnum\wd6>0\relax\advance\wysI by \wysMAX\advance\wysI by 1\baselineskip%
      \rput[tl](6pt,-\wysI){\box6}%
      \rput[t](434.5pt,-\wysI){{\bf P}}%
      \rput[t](457.5pt,-\wysI){{\bf N}}\relax\fi%
      \ifnum\wd7>0\relax\advance\wysI by \wysMAX\advance\wysI by 1\baselineskip%
      \rput[tl](6pt,-\wysI){\box7}%
      \rput[t](434.5pt,-\wysI){{\bf P}}%
      \rput[t](457.5pt,-\wysI){{\bf N}}\relax\fi%
      \advance\wysI by \wysMAX\advance\wysI by 0.5\baselineskip%
      \rput[B](423pt,0pt){\psline[linewidth=.425pt](0,0)(0,-\wysI)}%
      \rput[B](446pt,0pt){\psline[linewidth=.425pt](0,0)(0,-\wysI)}%
      \psframe[linewidth=.425pt](0,0pt)(469.5pt,-\wysI)%
      \vskip\wysI%
}

\def\wpTab #1 #2 #3 #4 #5 #6 #7 #8 {%
\overfullrule=0pt%
      \vskip0mm%
      \nobreak%
      \mathsurround=0pt%
      \wysI=0pt%
      \wysMAX=0pt%
      \pfCount=0%
      \setbox0\vbox{\hsize=406pt #1}%                          %pudełkowanie
      \setbox1\vbox{\hsize=406pt #2}%
      \setbox2\vbox{\hsize=406pt #3}%
      \setbox3\vbox{\hsize=406pt #4}%
      \setbox4\vbox{\hsize=406pt #5}%
      \setbox5\vbox{\hsize=406pt #6}%
      \setbox6\vbox{\hsize=406pt #7}%
      \setbox7\vbox{\hsize=406pt #8}%
      \ifnum\ht2>\ht0\relax\ht0=\ht2\relax\fi%   %spr wielkości par 1/3, 2/4
      \ifnum\ht3>\ht1\relax\ht1=\ht3\relax\fi%
      \ifnum\ht6>\ht4\relax\ht4=\ht6\relax\fi%   %spr wielkości par 5/7, 6/8
      \ifnum\ht7>\ht5\relax\ht5=\ht7\relax\fi%
      \ifnum\ht4>\ht0\relax\ht0=\ht4\relax\fi%   %spr wielkości par 1/5, 2/6
      \ifnum\ht5>\ht1\relax\ht1=\ht5\relax\fi%
      \ifnum\ht0>\ht1\relax\wysMAX=\ht0\relax%
            \else\wysMAX=\ht1\relax\fi%          %spr najwyższego pudełka
      \ifnum\wd0>0\relax\pfCount=0\relax\fi%
      \ifnum\wd1>0\relax\advance\pfCount by 1\relax\fi%
      \ifnum\wd2>0\relax\advance\pfCount by 1\relax\fi%
      \ifnum\wd3>0\relax\advance\pfCount by 1\relax\fi%
      \ifnum\wd4>0\relax\advance\pfCount by 1\relax\fi%
      \ifnum\wd5>0\relax\advance\pfCount by 1\relax\fi%
      \ifnum\wd6>0\relax\advance\pfCount by 1\relax\fi%
      \ifnum\wd7>0\relax\advance\pfCount by 1\relax\fi%
      \wysI=0.5\baselineskip%
      \wysII=\baselineskip\advance\wysII by \wysMAX%
      \multirput(0pt,-\wysII)(0,-\wysII){\pfCount}{\psline[linewidth=.425pt](0,0)(469.5pt,0)}%
      \ifnum\wd0>0\relax\rput[tl](6pt,-\wysI){\box0}%
      \rput[t](434.5pt,-\wysI){{\bf W}}%
      \rput[t](457.5pt,-\wysI){{\bf P}}\relax\fi%
      \ifnum\wd1>0\relax\advance\wysI by \wysMAX\advance\wysI by 1\baselineskip%
      \rput[tl](6pt,-\wysI){\box1}%
      \rput[t](434.5pt,-\wysI){{\bf W}}%
      \rput[t](457.5pt,-\wysI){{\bf P}}\relax\fi%
      \ifnum\wd2>0\relax\advance\wysI by \wysMAX\advance\wysI by 1\baselineskip%
      \rput[tl](6pt,-\wysI){\box2}%
      \rput[t](434.5pt,-\wysI){{\bf W}}%
      \rput[t](457.5pt,-\wysI){{\bf P}}\relax\fi%
      \ifnum\wd3>0\relax\advance\wysI by \wysMAX\advance\wysI by 1\baselineskip%
      \rput[tl](6pt,-\wysI){\box3}%
      \rput[t](434.5pt,-\wysI){{\bf W}}%
      \rput[t](457.5pt,-\wysI){{\bf P}}\relax\fi%
      \ifnum\wd4>0\relax\advance\wysI by \wysMAX\advance\wysI by 1\baselineskip%
      \rput[tl](6pt,-\wysI){\box4}%
      \rput[t](434.5pt,-\wysI){{\bf W}}%
      \rput[t](457.5pt,-\wysI){{\bf P}}\relax\fi%
      \ifnum\wd5>0\relax\advance\wysI by \wysMAX\advance\wysI by 1\baselineskip%
      \rput[tl](6pt,-\wysI){\box5}%
      \rput[t](434.5pt,-\wysI){{\bf W}}%
      \rput[t](457.5pt,-\wysI){{\bf P}}\relax\fi%
      \ifnum\wd6>0\relax\advance\wysI by \wysMAX\advance\wysI by 1\baselineskip%
      \rput[tl](6pt,-\wysI){\box6}%
      \rput[t](434.5pt,-\wysI){{\bf W}}%
      \rput[t](457.5pt,-\wysI){{\bf P}}\relax\fi%
      \ifnum\wd7>0\relax\advance\wysI by \wysMAX\advance\wysI by 1\baselineskip%
      \rput[tl](6pt,-\wysI){\box7}%
      \rput[t](434.5pt,-\wysI){{\bf W}}%
      \rput[t](457.5pt,-\wysI){{\bf P}}\relax\fi%
      \advance\wysI by \wysMAX\advance\wysI by 0.5\baselineskip%
      \rput[B](423pt,0pt){\psline[linewidth=.425pt](0,0)(0,-\wysI)}%
      \rput[B](446pt,0pt){\psline[linewidth=.425pt](0,0)(0,-\wysI)}%
      \psframe[linewidth=.425pt](0,0pt)(469.5pt,-\wysI)%
      \vskip\wysI%
}

\def\tn #1 #2 #3 #4 #5 #6 #7 #8 {
\overfullrule=0pt
      \vskip0mm
      \nobreak
      \mathsurround=0pt
      \wysI=0pt
      \wysMAX=0pt
      \pfCount=0
      \setbox0\vbox{\hsize=406pt #1}                          %pudełkowanie
      \setbox1\vbox{\hsize=406pt #2}
      \setbox2\vbox{\hsize=406pt #3}
      \setbox3\vbox{\hsize=406pt #4}
      \setbox4\vbox{\hsize=406pt #5}                          %pudełkowanie
      \setbox5\vbox{\hsize=406pt #6}
      \setbox6\vbox{\hsize=406pt #7}
      \setbox7\vbox{\hsize=406pt #8}
      \ifnum\ht2>\ht0\relax\ht0=\ht2\relax\fi   %spr wielkości par 1/3, 2/4
      \ifnum\ht3>\ht1\relax\ht1=\ht3\relax\fi
      \ifnum\ht6>\ht4\relax\ht4=\ht6\relax\fi   %spr wielkości par 5/7, 6/8
      \ifnum\ht7>\ht5\relax\ht5=\ht7\relax\fi
      \ifnum\ht4>\ht0\relax\ht0=\ht4\relax\fi   %spr wielkości par 1/5, 2/6
      \ifnum\ht5>\ht1\relax\ht1=\ht5\relax\fi
      \ifnum\ht0>\ht1\relax\wysMAX=\ht0\relax
            \else\wysMAX=\ht1\relax\fi %spr najwyższego pudełka
      \ifnum\wd0>0\relax\pfCount=0\relax\fi
      \ifnum\wd1>0\relax\advance\pfCount by 1\relax\fi
      \ifnum\wd2>0\relax\advance\pfCount by 1\relax\fi
      \ifnum\wd3>0\relax\advance\pfCount by 1\relax\fi
      \ifnum\wd4>0\relax\advance\pfCount by 1\relax\fi
      \ifnum\wd5>0\relax\advance\pfCount by 1\relax\fi
      \ifnum\wd6>0\relax\advance\pfCount by 1\relax\fi
      \ifnum\wd7>0\relax\advance\pfCount by 1\relax\fi
      \wysI=0.5\baselineskip
      \wysII=\baselineskip\advance\wysII by \wysMAX
      \multirput(0pt,-\wysII)(0,-\wysII){\pfCount}{\psline[linewidth=.425pt](0,0)(469.5pt,0)}%
      \ifnum\wd0>0\relax\rput[tl](6pt,-\wysI){\box0}%
      \rput[t](434.5pt,-\wysI){{\bf T}}%
      \rput[t](457.5pt,-\wysI){{\bf N}}\relax\fi
      \ifnum\wd1>0\relax\advance\wysI by \wysMAX\advance\wysI by 1\baselineskip
      \rput[tl](6pt,-\wysI){\box1}%
      \rput[t](434.5pt,-\wysI){{\bf T}}%
      \rput[t](457.5pt,-\wysI){{\bf N}}\relax\fi
      \ifnum\wd2>0\relax\advance\wysI by \wysMAX\advance\wysI by 1\baselineskip
      \rput[tl](6pt,-\wysI){\box2}%
      \rput[t](434.5pt,-\wysI){{\bf T}}%
      \rput[t](457.5pt,-\wysI){{\bf N}}\relax\fi
      \ifnum\wd3>0\relax\advance\wysI by \wysMAX\advance\wysI by 1\baselineskip
      \rput[tl](6pt,-\wysI){\box3}%
      \rput[t](434.5pt,-\wysI){{\bf T}}%
      \rput[t](457.5pt,-\wysI){{\bf N}}\relax\fi
      \ifnum\wd4>0\relax\advance\wysI by \wysMAX\advance\wysI by 1\baselineskip
      \rput[tl](6pt,-\wysI){\box4}%
      \rput[t](434.5pt,-\wysI){{\bf T}}%
      \rput[t](457.5pt,-\wysI){{\bf N}}\relax\fi
      \ifnum\wd5>0\relax\advance\wysI by \wysMAX\advance\wysI by 1\baselineskip
      \rput[tl](6pt,-\wysI){\box5}%
      \rput[t](434.5pt,-\wysI){{\bf T}}%
      \rput[t](457.5pt,-\wysI){{\bf N}}\relax\fi
      \ifnum\wd6>0\relax\advance\wysI by \wysMAX\advance\wysI by 1\baselineskip
      \rput[tl](6pt,-\wysI){\box6}%
      \rput[t](434.5pt,-\wysI){{\bf T}}%
      \rput[t](457.5pt,-\wysI){{\bf N}}\relax\fi
      \ifnum\wd7>0\relax\advance\wysI by \wysMAX\advance\wysI by 1\baselineskip
      \rput[tl](6pt,-\wysI){\box7}%
      \rput[t](434.5pt,-\wysI){{\bf T}}%
      \rput[t](457.5pt,-\wysI){{\bf N}}\relax\fi
      \advance\wysI by \wysMAX\advance\wysI by 0.5\baselineskip
      \rput[B](423pt,0pt){\psline[linewidth=.425pt](0,0)(0,-\wysI)}
      \rput[B](446pt,0pt){\psline[linewidth=.425pt](0,0)(0,-\wysI)}
      \psframe[linewidth=.425pt](0,0pt)(469.5pt,-\wysI)%
      \vskip\wysI%
}

\def\tnp #1 #2 #3 #4 #5 #6 #7 #8 {
\overfullrule=0pt
      \vskip0mm
      \nobreak
      \mathsurround=0pt
      \wysI=0pt
      \wysMAX=0pt
      \pfCount=0
      \setbox0\vbox{\hsize=352pt #1}                          %pudełkowanie
      \setbox1\vbox{\hsize=352pt #2}
      \setbox2\vbox{\hsize=352pt #3}
      \setbox3\vbox{\hsize=352pt #4}                          %pudełkowanie
      \setbox4\vbox{\hsize=352pt #5}
      \setbox5\vbox{\hsize=352pt #6}
      \setbox6\vbox{\hsize=352pt #7}                          %pudełkowanie
      \setbox7\vbox{\hsize=352pt #8}
      \ifnum\wd1>0\relax\advance\pfCount by 1\relax\fi
      \ifnum\wd2>0\relax\advance\pfCount by 1\relax\fi
      \ifnum\wd3>0\relax\advance\pfCount by 1\relax\fi
      \ifnum\wd4>0\relax\advance\pfCount by 1\relax\fi
      \ifnum\wd5>0\relax\advance\pfCount by 1\relax\fi
      \ifnum\wd6>0\relax\advance\pfCount by 1\relax\fi
      \ifnum\wd7>0\relax\advance\pfCount by 1\relax\fi
      \ifnum\ht2>\ht0\relax\ht0=\ht2\relax\fi   %spr wielkości par 1/3, 2/4
      \ifnum\ht3>\ht1\relax\ht1=\ht3\relax\fi
      \ifnum\ht6>\ht4\relax\ht4=\ht6\relax\fi   %spr wielkości par 5/7, 6/8
      \ifnum\ht7>\ht5\relax\ht5=\ht7\relax\fi
      \ifnum\ht4>\ht0\relax\ht0=\ht4\relax\fi   %spr wielkości par 1/5, 2/6
      \ifnum\ht5>\ht1\relax\ht1=\ht5\relax\fi
      \ifnum\ht0>\ht1\relax\wysMAX=\ht0\relax
            \else\wysMAX=\ht1\relax\fi
      \wysI=0.5\baselineskip
      \wysII=\baselineskip\advance\wysII by \wysMAX
      \multirput(80pt,-\wysII)(0,-\wysII){\pfCount}{\psline[linewidth=.425pt](0,0)(389.5pt,0)}%
      \rput[tl](110pt,-\wysI){\box0}
      \rput[t](92pt,-\wysI){{\bf A.}}
      \advance\wysI by \wysMAX\advance\wysI by 1\baselineskip
      \rput[tl](110pt,-\wysI){\box1}
      \rput[t](92pt,-\wysI){{\bf B.}}
      \ifnum\wd2>0\relax\advance\wysI by \wysMAX\advance\wysI by 1\baselineskip
      \rput[tl](110pt,-\wysI){\box2}
      \rput[t](92pt,-\wysI){{\bf C.}}\relax\fi
      \ifnum\wd3>0\relax\advance\wysI by \wysMAX\advance\wysI by 1\baselineskip
      \rput[tl](110pt,-\wysI){\box3}
      \rput[t](92pt,-\wysI){{\bf D.}}\relax\fi
      \ifnum\wd4>0\relax\advance\wysI by \wysMAX\advance\wysI by 1\baselineskip
      \rput[tl](110pt,-\wysI){\box4}
      \rput[t](92pt,-\wysI){{\bf E.}}\relax\fi
      \ifnum\wd5>0\relax\advance\wysI by \wysMAX\advance\wysI by 1\baselineskip
      \rput[tl](110pt,-\wysI){\box5}
      \rput[t](92pt,-\wysI){{\bf F.}}\relax\fi
      \ifnum\wd6>0\relax\advance\wysI by \wysMAX\advance\wysI by 1\baselineskip
      \rput[tl](110pt,-\wysI){\box6}
      \rput[t](92pt,-\wysI){{\bf G.}}\relax\fi
      \ifnum\wd7>0\relax\advance\wysI by \wysMAX\advance\wysI by 1\baselineskip
      \rput[tl](110pt,-\wysI){\box7}
      \rput[t](92pt,-\wysI){{\bf H.}}\relax\fi
      \advance\wysI by \wysMAX\advance\wysI by 0.5\baselineskip
      \psframe[linewidth=.425pt](0,0pt)(469.5pt,-\wysI)%
      \rput[B](24pt,0pt){\psline[linewidth=.425pt](0,0)(0,-\wysI)}
      \rput[B](80pt,0pt){\psline[linewidth=.425pt](0,0)(0,-\wysI)}
      \rput[B](104pt,0pt){\psline[linewidth=.425pt](0,0)(0,-\wysI)}
      \wysII=\wysI\divide\wysII by 2
      \rput[B](0pt,-\wysII){\psline[linewidth=.425pt](0,0)(24pt,0pt)}%
      \advance\wysII by 4pt
      \rput[B](52pt,-\wysII){ponieważ}
      \advance\wysII by -4pt\divide\wysII by 2\advance\wysII by 4pt
      \rput[B](12pt,-\wysII){{\bf T}}
      \advance\wysII by -4pt\multiply\wysII by 3\advance\wysII by 4pt
      \rput[B](12pt,-\wysII){{\bf N}}
      \vskip\wysI%
}


%-------------------------------------------------------------------
%           Makro tworzące zadania typu "tak/nie, ponieważ"
% Założenia: \hsize=469.5pt
% #1-#4:    W ogólnym przypadku nie musi być samo "tak/nie", mogą być 4 parametry, minimum dwa
% #5:       słowo "ponieważ" może być dowolnym tekstem o maksymalnej szerokości 110pt
% #6-#9:    Makro dopuszcza od dwóch do czterech zdań, ale również ich brak -- jeśli uczeń ma uzasadnić samodzielnie.
%-------------------------------------------------------------------

\def\uzasadnienie #1 #2 #3 #4 #5 #6 #7 #8 #9 {
\overfullrule=0pt
      \vskip0mm
      \nobreak
      \mathsurround=0pt
      \wysI=0pt
      \wysMAX=0pt
      \komMAX=0pt
      \temp=0pt
      \pfCount=1            %parametr #1 musi zawsze być
      \uzCount=0            %parametr #6 nie musi być obecny
      \setbox0\hbox{#1}                          %pudełkowanie
      \setbox1\hbox{#2}
      \setbox2\hbox{#3}
      \setbox3\hbox{#4}
      \setbox4\hbox{#5}
      \setbox5\hbox{#6}
      \setbox6\hbox{#7}
      \setbox7\hbox{#8}
      \setbox8\hbox{#9}
%------------sprawdzanie szerokości--------------------------------
      \ifnum \wd1>0 \advance\pfCount by 1\relax\fi%          %liczymy ile mamy parametrów
      \ifnum \wd2>0 \advance\pfCount by 1\relax\fi%
      \ifnum \wd3>0 \advance\pfCount by 1\relax\fi%
      \ifnum \wd5>0 \advance\uzCount by 1\relax\fi%
      \ifnum \wd6>0 \advance\uzCount by 1\relax\fi%
      \ifnum \wd7>0 \advance\uzCount by 1\relax\fi%
      \ifnum \wd8>0 \advance\uzCount by 1\relax\fi%
      \ifdim \wd4>103pt                                 %szerokość #5
        \setbox4\vbox{\hsize=103pt #5\raggedright}
        \szerII=117pt
      \else
        \szerII=\wd4
        \advance\szerII by 14pt
      \fi%
      \ifnum\wd1>\wd0\relax\wd0=\wd1\relax\fi   %szukamy najszerszego pudełka z lewej części
      \ifnum\wd3>\wd2\relax\wd2=\wd3\relax\fi
      \ifnum\wd2>\wd0\relax\wd0=\wd2\relax\fi
      \ifdim \wd0>134pt                         %jeśli przekracza wymiar chowamy do vboxów
          \setbox0\vbox{\hsize=134pt #1\raggedright}
          \setbox1\vbox{\hsize=134pt #2\raggedright}
          \setbox2\vbox{\hsize=134pt #3\raggedright}
          \setbox3\vbox{\hsize=134pt #4\raggedright}
          \szerI=164pt
      \else
        \szerI=\wd0
        \advance\szerI by 30pt
      \fi
      \szerIII=\hsize                           %na prawą część zostaje reszta wolnego miejsca, minimum 40% licząc ABCD
      \advance\szerIII by -\szerII
      \advance\szerIII by -\szerI
      \advance\szerIII by -30pt
      \setbox5\vbox{\hsize=\szerIII #6\raggedright}
      \setbox6\vbox{\hsize=\szerIII #7\raggedright}
      \setbox7\vbox{\hsize=\szerIII #8\raggedright}
      \setbox8\vbox{\hsize=\szerIII #9\raggedright}
%------------sprawdzanie wysokości i ewentualne poprawki-----------
      \advance\szerIII by 30pt
      \ifnum\ht1>\ht0\relax\ht0=\ht1\relax\fi   %szukamy najwyższego pudełka z lewej części
      \ifnum\ht3>\ht2\relax\ht2=\ht3\relax\fi
      \ifnum\ht2>\ht0\relax\ht0=\ht2\relax\fi
      \wysMAX=\ht0
      \ifnum \pfCount<\uzCount                      %przypadek, gdy więcej zdań jest po prawej i dodanie do lewej części poprawki z nim związanej
        \ifnum\ht6>\ht5\relax\ht5=\ht6\relax\fi     %szukamy najwyższego pudełka z prawej części
        \ifnum\ht8>\ht7\relax\ht7=\ht8\relax\fi
        \ifnum\ht7>\ht5\relax\ht5=\ht7\relax\fi
        \komMAX=\ht5
        \temp=\komMAX
        \wysMAX=0pt
        \advance\wysMAX by \uzCount\baselineskip    %wysMAX = ht5 * (uz/pf) + [(uz - pf) * /baselineskip]/pf
        \advance\wysMAX by -\pfCount\baselineskip   %komMAX = ht5
        \divide\wysMAX by \pfCount
        \divide\komMAX by \pfCount
        \multiply\komMAX by \uzCount
        \advance\komMAX by \wysMAX
        \wysMAX=\komMAX
        \komMAX=\temp
      \else
          \ifdim \ht5>0pt                               %jeśli istnieje #5, szukamy najwyższego pudełka z prawej części
              \ifnum\ht6>\ht5\relax\ht5=\ht6\relax\fi
              \ifnum\ht8>\ht7\relax\ht7=\ht8\relax\fi
              \ifnum\ht7>\ht5\relax\ht5=\ht7\relax\fi
              \komMAX=\ht5
          \else
            \komMAX=0pt
          \fi%
          \ifdim\komMAX>\wysMAX\relax\wysMAX=\komMAX\relax\fi   %w przypadku równości
      \fi%
      \wysI=0.5\baselineskip
      \wysII=\baselineskip\advance\wysII by \wysMAX
      \advance\pfCount by -1    %robimy linie między komórkami, musi być 1 mniej
      \multirput(0pt,-\wysII)(0,-\wysII){\pfCount}{\psline[linewidth=.425pt](0,0)(\szerI,0)}%
      \advance\pfCount by 1     %naprawiamy licznik
      \rput[tl](25pt,-\wysI){\box0}
      \rput[t](10pt,-\wysI){{\bf 1.}}
      \advance\wysI by \wysMAX\advance\wysI by 1\baselineskip
      \rput[tl](25pt,-\wysI){\box1}
      \rput[t](10pt,-\wysI){{\bf 2.}}
      \ifnum\wd2>0\relax\advance\wysI by \wysMAX\advance\wysI by 1\baselineskip
      \rput[tl](25pt,-\wysI){\box2}
      \rput[t](10pt,-\wysI){{\bf 3.}}\relax\fi
      \ifnum\wd3>0\relax\advance\wysI by \wysMAX\advance\wysI by 1\baselineskip
      \rput[tl](25pt,-\wysI){\box3}
      \rput[t](10pt,-\wysI){{\bf 4.}}\relax\fi
      \advance\wysI by \wysMAX\advance\wysI by 0.5\baselineskip%w tym momencie \wysI staje się stałą - wysokością całej tabeli, \wysII jest używana jako zmienna robocza
%------------rysowanie pionowych kresek oddzielających-------------
      \psframe[linewidth=.425pt](0,0pt)(469.5pt,-\wysI)%ramka całości
      \rput[B](20pt,0pt){\psline[linewidth=.425pt](0,0)(0,-\wysI)}%linia 123 | tak/nie
      \rput[B](\szerI,0pt){\psline[linewidth=.425pt](0,0)(0,-\wysI)}%linia tak/nie | ponieważ
      \szerIII=\szerI
      \advance\szerIII by \szerII
      \rput[B](\szerIII,0pt){\psline[linewidth=.425pt](0,0)(0,-\wysI)}%linia ponieważ | abc
      \ifnum \wd5>0
          \advance\szerIII by 20pt
          \rput[B](\szerIII,0pt){\psline[linewidth=.425pt](0,0)(0,-\wysI)}%linia abc | dopełnienia
      \else\fi
      \wysII=\wysI
      \advance\wysII by \ht4
      \divide\wysII by 2
      \advance\wysII by -2pt
      \szerIII=\szerII
      \advance\szerIII by 2\szerI
      \divide\szerIII by 2
      \rput[B](\szerIII,-\wysII){\box4}
%----------------------------------------------------------------
      \ifnum \wd5>0 %jesli jest parametr #5 tworzymy prawą część
          \szerIII=\szerII
          \advance\szerIII by \szerI
          \szerI=\hsize
          \advance\szerI by -\szerIII
          \advance\szerI by -1pt
          \ifnum \wd8>0 %przypadek 4 pól
                %4 kreski dla 1-4
                \wysII=\wysI\relax%
                \divide \wysII by 2\relax%
                \advance \wysII by -0.2pt\relax%
                \rput[B](\szerIII,-\wysII){\psline[linewidth=.425pt](0,0)(\szerI,0pt)}%środkowa kreska
                \divide \wysII by 2\relax%
                \rput[B](\szerIII,-\wysII){\psline[linewidth=.425pt](0,0)(\szerI,0pt)}%pierwsza kreska
                \multiply \wysII by 3\relax%
                \rput[B](\szerIII,-\wysII){\psline[linewidth=.425pt](0,0)(\szerI,0pt)}%trzecia kreska
          \else
              \ifnum \wd7>0 %przypadek 3 pól
                    %3 kreski dla 1-3
                    \wysII=\wysI\relax%
                    \divide \wysII by 3\relax%
                    \rput[B](\szerIII,-\wysII){\psline[linewidth=.425pt](0,0)(\szerI,0pt)}%
                    \multiply \wysII by 2\relax%
                    \advance \wysII by -0.2pt\relax%
                    \rput[B](\szerIII,-\wysII){\psline[linewidth=.425pt](0,0)(\szerI,0pt)} %
              \else%
                    %2 kreski dla 1-2
                    \wysII=\wysI\relax%
                    \divide \wysII by 2\relax%
                    \advance \wysII by -0.2pt\relax%
                    \rput[B](\szerIII,-\wysII){\psline[linewidth=.425pt](0,0)(\szerI,0pt)} %
              \fi%
          \fi%
          \szerII=\szerIII
          \advance\szerIII by 10pt
          \advance\szerII by 25pt
          \ifnum \pfCount<\uzCount
            \wysMAX=\komMAX
          \else
            \komMAX=0pt
            \advance\komMAX by \pfCount\baselineskip
            \advance\komMAX by -\uzCount\baselineskip
            \divide\komMAX by \uzCount
            \divide\wysMAX by \uzCount
            \multiply\wysMAX by \pfCount
            \advance\wysMAX by \komMAX
          \fi
          \wysI=0.5\baselineskip
          \wysII=\baselineskip\advance\wysII by \wysMAX
          \rput[tl](\szerII,-\wysI){\box5}
          \rput[t](\szerIII,-\wysI){{\bf A.}}
          \advance\wysI by \wysMAX\advance\wysI by 1\baselineskip
          \rput[tl](\szerII,-\wysI){\box6}
          \rput[t](\szerIII,-\wysI){{\bf B.}}
          \ifnum\wd7>0\relax\advance\wysI by \wysMAX\advance\wysI by 1\baselineskip
          \rput[tl](\szerII,-\wysI){\box7}
          \rput[t](\szerIII,-\wysI){{\bf C.}}\relax\fi
          \ifnum\wd8>0\relax\advance\wysI by \wysMAX\advance\wysI by 1\baselineskip
          \rput[tl](\szerII,-\wysI){\box8}
          \rput[t](\szerIII,-\wysI){{\bf D.}}\relax\fi
          \advance\wysI by \wysMAX\advance\wysI by 0.5\baselineskip
      \else\fi
      \vskip\wysI%odstęp pionowy, żeby zachować wymiar (tekst nie wjeżdżał na tabelę)
}

\def\dopelnienie #1 #2 #3 #4 #5 #6 #7 #8 #9 {
\overfullrule=0pt
      \vskip0mm
      \nobreak
      \mathsurround=0pt
      \SPR=0pt
      \setbox0\hbox{#1}                          %pudełkowanie
      \setbox1\hbox{#2}
      \setbox2\hbox{#3}
      \setbox3\hbox{#4}
      \setbox4\hbox{#5}
      \setbox5\hbox{#6}
      \setbox6\hbox{#7}
      \setbox7\hbox{#8}
      \setbox8\hbox{#9}
      \ifnum\wd4>\wd1\relax\wd1=\wd4\relax\fi       %sprawdzanie par #5 i #2, wynik przechodzi przez \wd1 (#2)
      \ifnum\wd7>\wd1\relax\wd1=\wd7\relax\fi       %sprawdzanie par #8 i #2, wynik przechodzi przez \wd1 (#2)
      \ifnum\wd5>\wd2\relax\wd2=\wd5\relax\fi       %sprawdzanie par #6 i #3, wynik przechodzi przez \wd2 (#3)
      \ifnum\wd8>\wd2\relax\wd2=\wd8\relax\fi       %sprawdzanie par #9 i #3, wynik przechodzi przez \wd2 (#3)
      \ifdim\wd1>130pt\relax\wd1=130pt\relax\fi
      \ifdim\wd2>130pt\relax\wd2=130pt\relax\fi
      \SPR=\hsize
      \advance\SPR by -20pt
      \advance\SPR by -\wd1
      \advance\SPR by -\wd2
      \hbox to \hsize{%
      \vtop{\hsize=\SPR#1}\hss%
      \vtop{\hsize=\wd1#2\raggedright}\hss%
      \vtop{\hsize=\wd2#3}}
      \hbox to \hsize{%
      \vtop{\hsize=\SPR#4}\hss%
      \vtop{\hsize=\wd1#5\raggedright}\hss%
      \vtop{\hsize=\wd2#6}}
      \hbox to \hsize{%
      \vtop{\hsize=\SPR#7}\hss%
      \vtop{\hsize=\wd1#8\raggedright}\hss%
      \vtop{\hsize=\wd2#9}}
}


\def\tabdokonczenie #1 #2 #3 #4 #5 {
\overfullrule=0pt
      \vskip0mm
      \nobreak
      \mathsurround=0pt
      \wysI=0pt
      \wysMAX=0pt
      \pfCount=0            %parametr #1 musi zawsze być
      \setbox0\hbox{#1}                          %pudełkowanie
      \setbox1\hbox{#2}
      \setbox2\hbox{#3}
      \setbox3\hbox{#4}
      \setbox4\hbox{#5}
%------------sprawdzanie szerokości--------------------------------
      \ifnum \wd2>0 \advance\pfCount by 1\relax\fi%     %liczymy ile mamy parametrów
      \ifnum \wd3>0 \advance\pfCount by 1\relax\fi%
      \ifnum \wd4>0 \advance\pfCount by 1\relax\fi%
      \ifdim \wd0>130pt                                 %szerokość #5
        \setbox0\vbox{\hsize=130pt #1\raggedright}
        \szerII=144pt
      \else
        \szerII=\wd0
        \advance\szerII by 14pt
      \fi%
      \szerIII=\hsize                           %na prawą część zostaje reszta wolnego miejsca, minimum 40% licząc ABCD
      \advance\szerIII by -\szerII
      \advance\szerIII by -30pt
      \setbox1\vbox{\hsize=\szerIII #2}
      \setbox2\vbox{\hsize=\szerIII #3}
      \setbox3\vbox{\hsize=\szerIII #4}
      \setbox4\vbox{\hsize=\szerIII #5}
      \advance\szerIII by 30pt
%------------sprawdzanie wysokości i ewentualne poprawki-----------
      \ifnum\ht2>\ht1\relax\ht1=\ht2\relax\fi   %szukamy najwyższego pudełka z lewej części
      \ifnum\ht4>\ht3\relax\ht3=\ht4\relax\fi
      \ifnum\ht3>\ht1\relax\ht1=\ht3\relax\fi
      \wysMAX=\ht1
      \wysI=0.5\baselineskip
      \wysII=\baselineskip\advance\wysII by \wysMAX
      \multirput(\szerII,-\wysII)(0,-\wysII){\pfCount}{\psline[linewidth=.425pt](0,0)(\szerIII,0)}%
      \szerIII=\szerII
      \advance\szerII by 25pt
      \advance\szerIII by 10pt
      \rput[tl](\szerII,-\wysI){\box1}
      \rput[t](\szerIII,-\wysI){{\bf A.}}
      \advance\wysI by \wysMAX\advance\wysI by 1\baselineskip
      \rput[tl](\szerII,-\wysI){\box2}
      \rput[t](\szerIII,-\wysI){{\bf B.}}
      \ifnum\wd3>0\relax\advance\wysI by \wysMAX\advance\wysI by 1\baselineskip
      \rput[tl](\szerII,-\wysI){\box3}
      \rput[t](\szerIII,-\wysI){{\bf C.}}\relax\fi
      \ifnum\wd4>0\relax\advance\wysI by \wysMAX\advance\wysI by 1\baselineskip
      \rput[tl](\szerII,-\wysI){\box4}
      \rput[t](\szerIII,-\wysI){{\bf D.}}\relax\fi
      \advance\wysI by \wysMAX\advance\wysI by 0.5\baselineskip%w tym momencie \wysI staje się stałą - wysokością całej tabeli, \wysII jest używana jako zmienna robocza
%------------rysowanie pionowych kresek oddzielających-------------
      \advance\szerII by -25pt
      \psframe[linewidth=.425pt](0,0pt)(469.5pt,-\wysI)%ramka całości
      \szerIII=0pt
      \advance\szerIII by \szerII
      \rput[B](\szerIII,0pt){\psline[linewidth=.425pt](0,0)(0,-\wysI)}%linia ponieważ | abc
      \advance\szerIII by 20pt
      \rput[B](\szerIII,0pt){\psline[linewidth=.425pt](0,0)(0,-\wysI)}%linia abc | dopełnienia
      \wysII=\wysI
      \advance\wysII by \ht0
      \divide\wysII by 2
      \advance\wysII by -2pt
      \szerIII=\szerII
      \advance\szerIII by 2\szerI
      \divide\szerIII by 2
      \rput[B](\szerIII,-\wysII){\box0}
%----------------------------------------------------------------
      \vskip\wysI%odstęp pionowy, żeby zachować wymiar (tekst nie wjeżdżał na tabelę)
}

\def\tabdokonczenieII #1 #2 #3 #4 #5 #6 #7 #8 {
\overfullrule=0pt
      \vskip0mm
      \nobreak
      \mathsurround=0pt
      \wysI=0pt
      \wysMAX=0pt
      \pfCount=0            %parametr #1 musi zawsze być
      \setbox0\hbox{#1}                          %pudełkowanie
      \setbox1\hbox{#2}
      \setbox2\hbox{#3}
      \setbox3\hbox{#4}
      \setbox4\hbox{#5}
      \setbox5\hbox{#6}
      \setbox6\hbox{#7}
      \setbox7\hbox{#8}
%------------sprawdzanie szerokości--------------------------------
      \ifnum \wd2>0 \advance\pfCount by 1\relax\fi%     %liczymy ile mamy parametrów
      \ifnum \wd3>0 \advance\pfCount by 1\relax\fi%
      \ifnum \wd4>0 \advance\pfCount by 1\relax\fi%
      \ifnum \wd5>0 \advance\pfCount by 1\relax\fi%
      \ifnum \wd6>0 \advance\pfCount by 1\relax\fi%
      \ifdim \wd0>130pt                                 %szerokość #5
        \setbox0\vbox{\hsize=130pt #1\raggedright}
        \szerII=144pt
      \else
        \szerII=\wd0
        \advance\szerII by 14pt
      \fi%
      \ifdim \wd7>130pt                                 %szerokość #5
        \setbox7\vbox{\hsize=130pt #8\raggedright}
        \szerI=144pt
      \else
        \szerI=\wd0
        \advance\szerI by 14pt
      \fi%
      \szerIII=\hsize                           %na prawą część zostaje reszta wolnego miejsca, minimum 40% licząc ABCD
      \advance\szerIII by -\szerII
      \advance\szerIII by -\szerI
      \advance\szerIII by -30pt
      \setbox1\vbox{\hsize=\szerIII #2}
      \setbox2\vbox{\hsize=\szerIII #3}
      \setbox3\vbox{\hsize=\szerIII #4}
      \setbox4\vbox{\hsize=\szerIII #5}
      \setbox5\vbox{\hsize=\szerIII #6}
      \setbox6\vbox{\hsize=\szerIII #7}
      \advance\szerIII by 30pt
%------------sprawdzanie wysokości i ewentualne poprawki-----------
      \ifnum\ht2>\ht1\relax\ht1=\ht2\relax\fi   %szukamy najwyższego pudełka z lewej części
      \ifnum\ht4>\ht3\relax\ht3=\ht4\relax\fi
      \ifnum\ht6>\ht5\relax\ht5=\ht6\relax\fi
      \ifnum\ht5>\ht3\relax\ht3=\ht5\relax\fi
      \ifnum\ht3>\ht1\relax\ht1=\ht3\relax\fi
      \wysMAX=\ht1
      \wysI=0.5\baselineskip
      \wysII=\baselineskip\advance\wysII by \wysMAX
      \multirput(\szerII,-\wysII)(0,-\wysII){\pfCount}{\psline[linewidth=.425pt](0,0)(\szerIII,0)}%
      \szerIII=\szerII
      \advance\szerII by 25pt
      \advance\szerIII by 10pt
      \rput[tl](\szerII,-\wysI){\box1}
      \rput[t](\szerIII,-\wysI){{\bf A.}}
      \advance\wysI by \wysMAX\advance\wysI by 1\baselineskip
      \rput[tl](\szerII,-\wysI){\box2}
      \rput[t](\szerIII,-\wysI){{\bf B.}}
      \ifnum\wd3>0\relax\advance\wysI by \wysMAX\advance\wysI by 1\baselineskip
      \rput[tl](\szerII,-\wysI){\box3}
      \rput[t](\szerIII,-\wysI){{\bf C.}}\relax\fi
      \ifnum\wd4>0\relax\advance\wysI by \wysMAX\advance\wysI by 1\baselineskip
      \rput[tl](\szerII,-\wysI){\box4}
      \rput[t](\szerIII,-\wysI){{\bf D.}}\relax\fi
      \ifnum\wd5>0\relax\advance\wysI by \wysMAX\advance\wysI by 1\baselineskip
      \rput[tl](\szerII,-\wysI){\box5}
      \rput[t](\szerIII,-\wysI){{\bf E.}}\relax\fi
      \ifnum\wd6>0\relax\advance\wysI by \wysMAX\advance\wysI by 1\baselineskip
      \rput[tl](\szerII,-\wysI){\box6}
      \rput[t](\szerIII,-\wysI){{\bf F.}}\relax\fi
      \advance\wysI by \wysMAX\advance\wysI by 0.5\baselineskip%w tym momencie \wysI staje się stałą - wysokością całej tabeli, \wysII jest używana jako zmienna robocza
%------------rysowanie pionowych kresek oddzielających-------------
      \advance\szerII by -25pt
      \psframe[linewidth=.425pt](0,0pt)(469.5pt,-\wysI)%ramka całości
      \szerIII=0pt
      \advance\szerIII by \szerII
      \rput[B](\szerIII,0pt){\psline[linewidth=.425pt](0,0)(0,-\wysI)}%linia ponieważ | abc
      \advance\szerIII by 20pt
      \rput[B](\szerIII,0pt){\psline[linewidth=.425pt](0,0)(0,-\wysI)}%linia abc | dopełnienia
      \wysII=\wysI
      \advance\wysII by \ht0
      \divide\wysII by 2
      \advance\wysII by -2pt
      \szerIII=\szerII
      \divide\szerIII by 2
      \rput[B](\szerIII,-\wysII){\box0}
%----------------------------------------------------------------
      \advance\szerII by 1pt
      \szerIII=\hsize
      \advance\szerIII by -\szerII
      \rput[B](\szerIII,0pt){\psline[linewidth=.425pt](0,0)(0,-\wysI)}%linia ponieważ | abc
      \wysII=\wysI
      \advance\wysII by \ht7
      \divide\wysII by 2
      \advance\wysII by -2pt
      \szerIII=\hsize
      \divide\szerII by 2
      \advance\szerIII by -\szerII
      \rput[B](\szerIII,-\wysII){\box7}
%----------------------------------------------------------------
      \vskip\wysI%odstęp pionowy, żeby zachować wymiar (tekst nie wjeżdżał na tabelę)
}

\def\tabdokonczenieIII #1 #2 #3 #4 #5 {
\overfullrule=0pt
      \vskip0mm
      \nobreak
      \mathsurround=0pt
      \wysI=0pt
      \wysMAX=0pt
      \pfCount=0            %parametr #1 musi zawsze być
      \setbox0\hbox{#1}                          %pudełkowanie
      \setbox1\hbox{#2}
      \setbox2\hbox{#3}
      \setbox3\hbox{#4}
      \setbox4\hbox{#5}
%------------sprawdzanie szerokości--------------------------------
      \ifnum \wd1>0 \advance\pfCount by 1\relax\fi%     %liczymy ile mamy parametrów
      \ifnum \wd2>0 \advance\pfCount by 1\relax\fi%
      \ifnum \wd3>0 \advance\pfCount by 1\relax\fi%
      \ifdim \wd4>130pt                                 %szerokość #5
        \setbox4\vbox{\hsize=130pt #5\raggedright}
        \szerII=144pt
      \else
        \szerII=\wd4
        \advance\szerII by 14pt
      \fi%
      \szerIII=\hsize                           %na prawą część zostaje reszta wolnego miejsca, minimum 40% licząc ABCD
      \advance\szerIII by -\szerII
      \advance\szerIII by -30pt
      \setbox0\vbox{\hsize=\szerIII #1}
      \setbox1\vbox{\hsize=\szerIII #2}
      \setbox2\vbox{\hsize=\szerIII #3}
      \setbox3\vbox{\hsize=\szerIII #4}
      \advance\szerIII by 30pt
%------------sprawdzanie wysokości i ewentualne poprawki-----------
      \ifnum\ht1>\ht0\relax\ht0=\ht1\relax\fi   %szukamy najwyższego pudełka z lewej części
      \ifnum\ht3>\ht2\relax\ht2=\ht3\relax\fi
      \ifnum\ht2>\ht0\relax\ht0=\ht2\relax\fi
      \wysMAX=\ht0
      \wysI=0.5\baselineskip
      \wysII=\baselineskip\advance\wysII by \wysMAX
      \multirput(0pt,-\wysII)(0,-\wysII){\pfCount}{\psline[linewidth=.425pt](0,0)(\szerIII,0)}%
      \szerIII=0pt
      \szerI=0pt
      \advance\szerI by 25pt
      \advance\szerIII by 10pt
      \rput[tl](\szerI,-\wysI){\box0}
      \rput[t](\szerIII,-\wysI){{\bf A.}}
      \advance\wysI by \wysMAX\advance\wysI by 1\baselineskip
      \rput[tl](\szerI,-\wysI){\box1}
      \rput[t](\szerIII,-\wysI){{\bf B.}}
      \ifnum\wd2>0\relax\advance\wysI by \wysMAX\advance\wysI by 1\baselineskip
      \rput[tl](\szerI,-\wysI){\box2}
      \rput[t](\szerIII,-\wysI){{\bf C.}}\relax\fi
      \ifnum\wd3>0\relax\advance\wysI by \wysMAX\advance\wysI by 1\baselineskip
      \rput[tl](\szerI,-\wysI){\box3}
      \rput[t](\szerIII,-\wysI){{\bf D.}}\relax\fi
      \advance\wysI by \wysMAX\advance\wysI by 0.5\baselineskip%w tym momencie \wysI staje się stałą - wysokością całej tabeli, \wysII jest używana jako zmienna robocza
%------------rysowanie pionowych kresek oddzielających-------------
      \psframe[linewidth=.425pt](0,0pt)(469.5pt,-\wysI)%ramka całości
      \szerIII=0pt
      \advance\szerIII by 20pt
      \rput[B](\szerIII,0pt){\psline[linewidth=.425pt](0,0)(0,-\wysI)}%linia ponieważ | abc
      \advance\szerII by 1pt
      \szerIII=\hsize
      \advance\szerIII by -\szerII
      \rput[B](\szerIII,0pt){\psline[linewidth=.425pt](0,0)(0,-\wysI)}%linia ponieważ | abc
      \wysII=\wysI
      \advance\wysII by \ht4
      \divide\wysII by 2
      \advance\wysII by -2pt
      \szerIII=\hsize
      \divide\szerII by 2
      \advance\szerIII by -\szerII
      \rput[B](\szerIII,-\wysII){\box4}
%----------------------------------------------------------------
      \vskip\wysI%odstęp pionowy, żeby zachować wymiar (tekst nie wjeżdżał na tabelę)
}

%------- działania pisemne


\newmuskip\defaultmuskip \defaultmuskip=6mu
\def\fifo #1{\ifx\ofif#1\ofif\fi\process#1\fifo}
\def\ofif #1\fifo{\fi}

\def\slupek#1#2#3#4% [+/-] x y wynik
   {\vtop{\def\process ##1{\hbox to4.7mm{\hfil ##1\hfil}}%
      \def\mhrulefill{\leaders\hrule height.3mm\hfill}
      \normalbaselines
      \ialign{\strut\hfil$##$&&\hfil$##$\hfil\crcr
              \strut\crcr\noalign{\vskip-\baselineskip}
                &\fifo#2\ofif\crcr\noalign{\vskip-.4mm}
              \hbox to4.7mm{\hfil$#1$\hfil}&\fifo#3\ofif\crcr\crcr\noalign{\vskip-4.05mm}
              \multispan2\mhrulefill\cr
                &\fifo#4\ofif\crcr
              \strut\crcr\noalign{\vskip-1\baselineskip}}}}%

\def\slupekII#1#2#3% [+/-] nad pod
   {\vtop{\def\process ##1{\hbox to4.7mm{\hfil ##1\hfil}}%
      \def\mhrulefill{\leaders\hrule height.3mm\hfill}
      \normalbaselines
      \ialign{\strut\hfil$##$&&\hfil$##$\hfil\crcr
              \hbox to4.7mm{\hfil$#1$\hfil}&\fifo#2\ofif\crcr\crcr\noalign{\vskip-4.05mm}
              \multispan2\mhrulefill\cr
                &\fifo#3\ofif\crcr
              \strut\crcr\noalign{\vskip-\baselineskip}}}}%

%------- działania z luką

\def\w#1{\leavevmode\kern0pt\raise4.7mm\hbox{$#1$}}
\def\dod{\vbox{\hsize1.5cm\hbox to\hsize{\h1 \dotfill\h1 }\v-1 \vskip-.8mm\centerline{{\scriptsize dodatnia}}}}
\def\uj{\vbox{\hsize1.5cm\hbox to\hsize{\h1 \dotfill\h1 }\v-1 \vskip-.8mm\centerline{{\scriptsize ujemna}}}}
\def\dodk{\vbox{\hsize1.5cm\hbox to\hsize{\h1 \dotfill\h1 }\v-1 \vskip-.8mm\centerline{{\scriptsize dodatnia}}}}
\def\ujk{\vbox{\hsize1.5cm\hbox to\hsize{\h1 \dotfill\h1 }\v-1 \vskip-.8mm\centerline{{\scriptsize ujemna}}}}
\def\ze{\vbox{\hsize1.5cm\hbox to\hsize{\h1 \dotfill\h1 }\v-1 \vskip-.8mm\centerline{{\scriptsize zero}}}}
\def\podpis #1 #2 {\vbox{\hsize=#1\noindent\lower0pt\hbox to\hsize{\h1 \dotfill\h1 } \noindent\raise7pt\centerline{{\scriptsize #2}}\vskip-18pt}}

\def\mlhr#1{\multispan#1\hrulefill}

			
%------- Rysunki


\def \NazwaEPS #1{%
     \ifsklad\rput[tl](-.7,0){{\NazwaEPSfont [#1]}}\else\fi}

\def \Lpic #1 #2 #3%
    {\vskip#1mm
        \leftline {\XeTeXpdffile '#3' }
     \vskip#2mm
    }
%-------------------
\def \Rpic #1 #2 #3%
    {\vskip#1mm
        \rightline {\XeTeXpdffile '#3'\hskip-.5mm }
     \vskip#2mm
    }
%-------------------
\def \Cpic #1 #2 #3%
    {\vskip#1mm
        \centerline {\XeTeXpdffile '#3' }
     \vskip#2mm
    }

\def \Lspic #1 #2 #3 #4%
    {\vskip#1mm
        \leftline {\zscale{#4}\vbox{\XeTeXpdffile '#3' }}
     \vskip#2mm
    }
%-------------------
\def \Rspic #1 #2 #3 #4%
    {\vskip#1mm
        \rightline {\zscale{#4}\vbox{\XeTeXpdffile '#3' }\hskip-.5mm}
     \vskip#2mm
    }
%-------------------
\def \Cspic #1 #2 #3 #4%
    {\vskip#1mm
        \centerline {\NazwaEPS {#3}\zscale{#4}\vbox{\XeTeXpdffile '#3' }}
     \vskip#2mm
    }

\def \lpic #1 #2 #3%
    {\rput[tl](#1,#2){\leftline{\XeTeXpdffile '#3' }}}

\def \lspic #1 #2 #3 #4%
    {\rput[tl](#1,#2){\leftline{\zscale{#4}\hbox{\XeTeXpdffile '#3' }}}}

\def \cpic #1 #2 #3%
    {\rput[tl](#1,#2){\centerline{\XeTeXpdffile '#3' }}}

\def \cspic #1 #2 #3 #4%
    {\rput[tl](#1,#2){\centerline{\zscale{#4}\hbox{\XeTeXpdffile '#3' }}}}

\def \rspic #1 #2 #3 #4%
    {\rput[tl](#1,#2){\rightline{\zscale{#4}\hbox{\XeTeXpdffile '#3' }}}}

\def \rpic #1 #2 #3%
    {\rput[tl](#1,#2){\rightline{\XeTeXpdffile '#3' }}}


%------- \naklejka
%---------------------------

\newcmykcolor{Raster}{0 0 0 .3}
%-------------------------------------Shadow--------------------------------------
% Parametry komendy BaseBlok
% #1 -- odci‘cie cienia od g˘ry (prawy r˘g) -- dimen
% #2 -- grubožŤ cienia z boku (prawego)     -- dimen
% #3 -- odci‘cie cienia z do’u (lewy r˘g)   -- dimen
% #4 -- grubožŤ cienia z do’u               -- dimen
% #5 -- tekst w ramce                       -- text
\long\def\BaseBlok#1#2#3#4#5{%
 \vbox{\setbox0=\hbox{#5} \offinterlineskip
 \hbox{\copy0\dimen0=\ht0 \advance\dimen0 by -#1
       {\Raster \vrule height\dimen0 width #2}}
 \hbox{\hskip#3\dimen0=\wd0
   \advance\dimen0 by -#3 \advance\dimen0 by #2
      {\Raster \vrule height #4 width\dimen0}} }}

\long\def\Shadow#1{\BaseBlok{1.5mm}{1mm}{1.5mm}{1mm}{#1}}

\long \def \SFrame #1 #2%
   {\Shadow{%
    \Frame{3.5mm}{0.08mm}{#1cm}{#2}%
    }}

% naklejka z ramk† TeX-ow†
\long \def \naklejka #1 #2 #3 #4 #5%
    {\setbox0 \vbox {\leftskip=0pt
                      \SFrame #3 {\vglue-1mm\relax%
                      #5%
                     }}
     \rput[br](#1,#2){\rotate{#4}\hbox{\box0}}
    }







%%%%%%%%%%%%%%%%%%%%%%%%%%%%%%%%%%%%
%------- Funkcje czasu i godziny


%----------------------------------
\newcount\hours
\newcount\minutes
\def\oktime{% format `hh:mm'
\hours=\time \divide \hours by 60 %
\minutes=-\hours \multiply \minutes by 60 \advance \minutes by \time
\ifnum\hours>9 \the\hours \else 0\the\hours \fi %
:%
\ifnum\minutes>9 \the\minutes \else 0\the\minutes \fi}
%----------------------------------
\def\monthnazwa{%
\ifcase \month%
\or stycznia\or lutego\or marca\or kwietnia\or maja\or czerwca\or lipca%
\or sierpnia\or września\or października\or listopada\or grudnia%
\fi}%

\newcount\licznik
\licznik=0
\def\kreska{\ifodd\licznik\vrule height1mm depth0mm width.4pt\rput[Bl](.2,0){\tiny\the\licznik}\newline\relax\else\vrule height1mm depth0mm width0pt\newline\relax\fi\advance\licznik by 1}

%--------------------------------------------------------------------
%%%%%%%%%%%%%%%%%%%%%%%%%%%%%%%%%%%%
%------- Makro tabelka i cała reszta

\long\def\odptab #1 #2 #3 #4 #5 #6%
{\global\Podpunkt=97\vtop{\halign to 735pt{%
\vrule\vtop{\hsize=25pt\strut\h2 ##\hfil}\vrule&%
\hskip6pt\vtop{\hsize=#1pt\lineskip=3pt##\hfil}\hskip6pt\vrule&%
\vtop{\hsize=41pt\strut\h3 ##\hfil}\vrule&%
\hskip6pt\vtop{\hsize=#2pt\lineskip=3pt##\hfil}\hskip6pt\vrule\cr
#3.&%
#5&%
0--#4&%
#6\cr%
\omit\vrule\vrule height4pt width0pt depth0pt\hfil\vrule&\omit\hfil\vrule&\omit\hfil\vrule&\omit\hfil\vrule\cr
\noalign{\hrule}
}}}

\long\def\odptabhead #1 #2%
{\vtop{\halign to 735pt{%
\vrule\vtop{\hsize=25pt\strut\hfil##\hfil}\vrule&%
\hskip6pt\vtop{\hsize=#1pt\lineskip=3pt\hfil##\hfil}\hskip6pt\vrule&%
\vtop{\hsize=41pt\strut\hfil##\hfil}\vrule&%
\hskip6pt\vtop{\hsize=#2pt\lineskip=3pt\hfil##\hfil}\hskip6pt\vrule\cr
\bf Lp.&%
\bf Odpowiedź&%
\bf Punkty&%
\bf Schemat oceniania\cr%
\omit\vrule\vrule height4pt width0pt depth0pt\hfil\vrule&\omit\hfil\vrule&\omit\hfil\vrule&\omit\hfil\vrule\cr
\noalign{\hrule}
}}}

\long\def\odptabgrupa #1%
{\vtop{\halign to 735pt{%
\hbox to 735pt{\strut\vrule\hss \textbf{GRUPA ## -- KLUCZ ODPOWIEDZI}\hss\vrule}\cr
#1\cr
\omit\vrule\hfil\vrule height4pt width0pt depth0pt\hfil\vrule\cr
\noalign{\hrule}
}}}

\newcount \dPodpunkt \dPodpunkt=65
\newdimen \dpindent \dpindent=5mm
\newdimen \hboxa \dpindent=33.5mm

\newcount\Nrboxa\Nrboxa=1
\newcount\Lboxow\Lboxow=0
\newcount\Nrkolumn\Nrkolumn=0
\newcount\Lkolumn\Lkolumn=0
\newcount\Nrwiersz\Nrwiersz=1
\newcount\Lwiersz\Lwiersz=0
\newcount\iloczyn\iloczyn=0

\newcount\Podstep\Podstep=0
\newdimen \Wys
\newdimen\Lbazowa\Lbazowa=1.5mm   %standartowy odstep pomiedzy wierszami tabeli
\newdimen\roznica\roznica=0mm
\newdimen\Wcinka \Wcinka=165mm
\newdimen\Podstep\Podstep=0mm

\gdef\pB #1 {{\global\setbox\Nrboxa\hbox to\hboxa{#1\hss}
              \global\advance\Nrboxa by1
             }}

\gdef\pBB #1 {{\global\setbox\Nrboxa\hbox to\hboxa{\h4.5 #1\hss}
              \global\advance\Nrboxa by1
             }}

\gdef\PB #1 {{\global\setbox\Nrboxa\hbox to33.5mm{#1\hss}
              \global\advance\Nrboxa by1
             }}


\gdef\p #1 {{\global\setbox\Nrboxa\hbox to\hboxa{\podpunkt#1\hss}
             \global\advance\Podpunkt by1
             \global\advance\Nrboxa by1
            }}

\def \tabelka #1 #2 #3
    {\global\Podpunkt=97
     \global\Nrboxa=0
     \ifnum#1=2\global\hboxa=67mm\fi%
     \ifnum#1=3\global\hboxa=43mm\fi%
     \ifnum#1=4\global\hboxa=33.5mm\fi%
     \global\Lkolumn=#1
     \global\Lwiersz=0
     \global\Nrkolumn=1          %kolumny numerowane są od jedynki
     \global\Nrwiersz=0          %wiersze od 0
     \global\Lbazowa=1.5mm
     \global\advance\Lbazowa by#2mm
     \global\Podstep=10mm
     \global\advance\Podstep by #3mm
     }


% makro przewiduje od 1 do 6 kolumn i maksymalnie 11 wierszy!
\def \tabelkaEND {{\vskip1.5mm
                  \global\Lboxow=\Nrboxa
%-----------------------------------
                  \global\iloczyn\Nrwiersz
                  \loop\ifnum\iloczyn<\Lboxow
                    \global\advance\Lwiersz by1
                    \global\iloczyn=\Lwiersz
                    \global\multiply\iloczyn by\Lkolumn
                  \repeat
%------------------------------------
\hfuzz30pt
% Ustalamy nr boxa w tabeli.
                  \def\UNB{{%
                      \global\Nrboxa=\Nrkolumn
                      \global\multiply\Nrboxa by\Lwiersz
                      \global\advance\Nrboxa by\Nrwiersz
                      \global\advance\Nrboxa by-\Lwiersz
                      \ifnum\Nrkolumn<\Lkolumn
                         \global\advance\Nrkolumn by 1
                      \else
                         \global\Nrkolumn=1
                         \global\advance\Nrwiersz by 1
                      \fi
                      }}
%------------------------------------
                  \global\iloczyn=11
                  \global\advance\iloczyn by -\Lwiersz
                  \h0 \setbox0\vbox{%
                  \ifcase\Lkolumn%
                  \or
                     \halign {\tabskip=\Podstep%10mm plus30mm minus3mm
                     \UNB\box\Nrboxa##\tabskip=0mm\cr
                       \cr\noalign{\vskip\Lbazowa}
                       \cr\noalign{\vskip\Lbazowa}
                       \cr\noalign{\vskip\Lbazowa}
                       \cr\noalign{\vskip\Lbazowa}
                       \cr\noalign{\vskip\Lbazowa}
                       \cr\noalign{\vskip\Lbazowa}
                       \cr\noalign{\vskip\Lbazowa}
                       \cr\noalign{\vskip\Lbazowa}
                       \cr\noalign{\vskip\Lbazowa}
                       \cr\noalign{\vskip\Lbazowa}
                       \cr}%
                  \or
                     \halign {\tabskip=\Podstep%10mm plus30mm minus3mm
                     \UNB\box\Nrboxa##&\UNB\box\Nrboxa##\tabskip0mm\cr
                       &\cr\noalign{\vskip\Lbazowa}
                       &\cr\noalign{\vskip\Lbazowa}
                       &\cr\noalign{\vskip\Lbazowa}
                       &\cr\noalign{\vskip\Lbazowa}
                       &\cr\noalign{\vskip\Lbazowa}
                       &\cr\noalign{\vskip\Lbazowa}
                       &\cr\noalign{\vskip\Lbazowa}
                       &\cr\noalign{\vskip\Lbazowa}
                       &\cr\noalign{\vskip\Lbazowa}
                       &\cr\noalign{\vskip\Lbazowa}
                       &\cr}%
                  \or
                     \halign {\tabskip=\Podstep%10mm plus30mm minus3mm
                     \UNB\box\Nrboxa##&\UNB\box\Nrboxa##&\UNB\box\Nrboxa##\tabskip0mm\cr
                       &&\cr\noalign{\vskip\Lbazowa}
                       &&\cr\noalign{\vskip\Lbazowa}
                       &&\cr\noalign{\vskip\Lbazowa}
                       &&\cr\noalign{\vskip\Lbazowa}
                       &&\cr\noalign{\vskip\Lbazowa}
                       &&\cr\noalign{\vskip\Lbazowa}
                       &&\cr\noalign{\vskip\Lbazowa}
                       &&\cr\noalign{\vskip\Lbazowa}
                       &&\cr\noalign{\vskip\Lbazowa}
                       &&\cr\noalign{\vskip\Lbazowa}
                       &&\cr}%
                  \or
                     \halign {\tabskip=\Podstep%10mm plus30mm minus3mm
                     \UNB\box\Nrboxa##&\UNB\box\Nrboxa##&%
                     \UNB\box\Nrboxa##&\UNB\box\Nrboxa##\tabskip0mm\cr
                       &&&\cr\noalign{\vskip\Lbazowa}
                       &&&\cr\noalign{\vskip\Lbazowa}
                       &&&\cr\noalign{\vskip\Lbazowa}
                       &&&\cr\noalign{\vskip\Lbazowa}
                       &&&\cr\noalign{\vskip\Lbazowa}
                       &&&\cr\noalign{\vskip\Lbazowa}
                       &&&\cr\noalign{\vskip\Lbazowa}
                       &&&\cr\noalign{\vskip\Lbazowa}
                       &&&\cr\noalign{\vskip\Lbazowa}
                       &&&\cr\noalign{\vskip\Lbazowa}
                       &&&\cr}%
                  \or
                     \halign {\tabskip=\Podstep%10mm plus30mm minus3mm
                     \UNB\box\Nrboxa##&\UNB\box\Nrboxa##&\UNB\box\Nrboxa##&%
                     \UNB\box\Nrboxa##&\UNB\box\Nrboxa##\tabskip0mm\cr
                       &&&&\cr\noalign{\vskip\Lbazowa}
                       &&&&\cr\noalign{\vskip\Lbazowa}
                       &&&&\cr\noalign{\vskip\Lbazowa}
                       &&&&\cr\noalign{\vskip\Lbazowa}
                       &&&&\cr\noalign{\vskip\Lbazowa}
                       &&&&\cr\noalign{\vskip\Lbazowa}
                       &&&&\cr\noalign{\vskip\Lbazowa}
                       &&&&\cr\noalign{\vskip\Lbazowa}
                       &&&&\cr\noalign{\vskip\Lbazowa}
                       &&&&\cr\noalign{\vskip\Lbazowa}
                       &&&&\cr}%
                  \or
                     \halign {\tabskip=\Podstep%10mm plus30mm minus3mm
                     \UNB\box\Nrboxa##&\UNB\box\Nrboxa##&%
                     \UNB\box\Nrboxa##&\UNB\box\Nrboxa##&%
                     \UNB\box\Nrboxa##&\UNB\box\Nrboxa##\tabskip0mm\cr
                       &&&&&\cr\noalign{\vskip\Lbazowa}
                       &&&&&\cr\noalign{\vskip\Lbazowa}
                       &&&&&\cr\noalign{\vskip\Lbazowa}
                       &&&&&\cr\noalign{\vskip\Lbazowa}
                       &&&&&\cr\noalign{\vskip\Lbazowa}
                       &&&&&\cr\noalign{\vskip\Lbazowa}
                       &&&&&\cr\noalign{\vskip\Lbazowa}
                       &&&&&\cr\noalign{\vskip\Lbazowa}
                       &&&&&\cr\noalign{\vskip\Lbazowa}
                       &&&&&\cr\noalign{\vskip\Lbazowa}
                       &&&&&\cr}%
                  \or
                     \halign {\tabskip=\Podstep%10mm plus30mm minus3mm
                     \UNB\box\Nrboxa##&\UNB\box\Nrboxa##&\UNB\box\Nrboxa##&%
                     \UNB\box\Nrboxa##&\UNB\box\Nrboxa##&%
                     \UNB\box\Nrboxa##&\UNB\box\Nrboxa##\tabskip0mm\cr
                       &&&&&&\cr\noalign{\vskip\Lbazowa}
                       &&&&&&\cr\noalign{\vskip\Lbazowa}
                       &&&&&&\cr\noalign{\vskip\Lbazowa}
                       &&&&&&\cr\noalign{\vskip\Lbazowa}
                       &&&&&&\cr\noalign{\vskip\Lbazowa}
                       &&&&&&\cr\noalign{\vskip\Lbazowa}
                       &&&&&&\cr\noalign{\vskip\Lbazowa}
                       &&&&&&\cr\noalign{\vskip\Lbazowa}
                       &&&&&&\cr\noalign{\vskip\Lbazowa}
                       &&&&&&\cr\noalign{\vskip\Lbazowa}
                       &&&&&&\cr}%
                  \else\fi}%
                  \Wys = \ht 0%
                  \global\roznica=\baselineskip
                  \multiply\roznica by\iloczyn
                  \advance\Wys by-\roznica
                  \global\roznica=\Lbazowa
                  \multiply\roznica by\iloczyn
                  \advance\Wys by-\roznica
                  \global\ht0=\Wys
                  \nobreak
                  \box0%
                  \vskip.5mm
                  }}

%%%%%%%%%%%%%%%%%%%%%%%%%%%%%%%%%%%%%%%%%%%%%%%%%%%%%%%%%%%%%%%%%%%%%%%%%%%%%%%%%%%%%%%%%%%
%                                   Język polski testowe
%%%%%%%%%%%%%%%%%%%%%%%%%%%%%%%%%%%%%%%%%%%%%%%%%%%%%%%%%%%%%%%%%%%%%%%%%%%%%%%%%%%%%%%%%%%

\def\tekstinfo #1 #2 #3 #4 #5 {%
%\rput[tl](\hsize,\baselineskip){\h5 \vbox{\scriptsize #3min\par #4 słów\par #5 zad.}}%
%
\centerline{{\large #1} \hskip2em {\bf\large #2}}
\vskip1\baselineskip
}

\endinput