%------- INPUT
\input trans
\input epsfx
\input ../input/bop-hax
\input map
\input split

%====================================================
%                  F O N T Y
%====================================================
\defaultfontfeatures{Ligatures=TeX,Scale=.92}
\setmainfont{Lucida Bright OT}
\setsansfont{Lucida Sans OT}
\setmonofont{Lucida Sans Typewriter OT}
\setmathfont{Lucida Bright Math OT}
\setmathfont[version=bold]{Lucida Bright Math OT Demibold}

\newfontface\LucidaBlackletter{Lucida Blackletter OT}       %% a nuż się kiedyś przydadzą
\newfontface\LucidaCalligraphy{Lucida Calligraphy OT}
\newfontface\LucidaHandwriting{Lucida Handwriting OT}

\makeatletter
  \newcommand\lokofont{\@setfontsize\lokofont{15pt}{19}}
  \newcommand\espe{\@setfontsize\espe{11.5pt}{15}}
  \newcommand\espemat{\@setfontsize\espemat{13pt}{12.5}}
  \newcommand\gimbaza{\@setfontsize\gimbaza{10pt}{14}}
  \newcommand\gimbazamat{\@setfontsize\gimbazamat{11.5pt}{11}}
  \newcommand\trialheader{\@setfontsize\trialheader{14pt}{12}}
  \newcommand\trialheaderT{\@setfontsize\trialheaderT{7pt}{7}}
  \newcommand\trialheaderR{\@setfontsize\trialheaderR{10pt}{10}}
  \newcommand\gimbazanorr{\@setfontsize\gimbazanorr{8pt}{10}}  %%dodałem, bo potrzebowałe mniejszych Ł.
\makeatother

% czerwony tekst (wyróżnienie tekstu)
\newcommand{\highlighted}[1]{{\color{red}#1}}

\def\nor{}
\def\norr{}   %%ale to coś mi nie działa :)
\let\nor\gimbaza
\let\norr\gimbazanorr



%------- Ułamki
\makeatletter                       %przerabianie fraca od AMStex
  \DeclareRobustCommand{\frac}{\new@ifnextchar[{\frfrac{}}{\frfrac{}[]}}
  \def\g@ra{\raise.125em}
  \def\rozp@rka{\vrule height.615em depth0pt width0pt}
  \def\frfrac#1[#2]#3#4{\def\next@{#2}%
    \ifx\next@\@empty \def\next@{\g@ra\hbox{$#1{#3\over\rozp@rka\relax#4}$}}%
    \else \def\next@{\g@ra\hbox{$#1{#3\above#2\relax\rozp@rka\relax#4}$}}%
    \fi
    \next@}
  \def\dfrac{\protect\frfrac\displaystyle[]}
  \def\tfrac{\protect\frfrac\textstyle[]}
\makeatother

\def\jednostki#1#2{\frac{\mathrm{#1}}{\mathrm{#2}}}         % km na godzine itepe

\catcode`\=\active
\def \Imat
  {\begingroup
   \def{$\egroup\endgroup}%
   \lower0.03em\hbox\bgroup\gimbazamat$%%%%%%%%%%%
  }
\let=\Imat%

\catcode`\=\active
\def \IImat
  {\begingroup
   \def{$\egroup\endgroup}%
   \lower0.06em\hbox\bgroup\espemat$%%%%%%%%%%%
  }
\let=\IImat%

\def\Frac #1#2{\bgroup \frac{#1}{#2} \egroup}
\def\FracII #1#2{\bgroup \frac{#1}{#2} \egroup}

%------- Rozmiary strony definiujemy na początku main.tex
\topskip = 0mm

\parindent=0cm
\parskip0pt
\hfuzz=1pt
\thinmuskip=3mu
\medmuskip=4mu
\thickmuskip=5mu
\renewcommand{\line}[1]{\hbox to\hsize{#1}}
\def \v #1 {\vskip#1mm\relax}
\def \h #1 {\hskip#1mm\relax}
\def \a #1 {\noalign{\vskip#1mm}}

%------- county dimeny fontowe bzdety itepe
\def\kompozytorFont[1#1pt]{%
\ifcase #1\gimbaza%0
      \or\espe%1
      \else Błąd. Niepoprawny rozmiar czcionki.\fi%error
}

\newcount\Podpunkt \Podpunkt=97     %\pp
\newdimen\pindent \pindent=4.5mm    %\pp
\newdimen\SzerOka  %\odp
\newdimen\Szeroka  %\odp
\newdimen\SPR      %\odp
\newdimen\tmpa      %kratki
\newdimen\temp      %podręczna
\newif\ifsklad
%\skladfalse
\newdimen\kol       %wcinka
\newdimen\szer      %wcinka
\newdimen\dszer     %wcinka
\newdimen\Wcinka    %wcinka
\global\Wcinka\hsize%wcinka

\def\NazwaEPSfont{\tt\tiny}

%------- Kropeczki
\def\dotfill{\xleaders\hbox to 3.219pt{\hss\tiny.\hss}\hfill}
\def\Dotfill{\xleaders\hbox to 3.219pt{\hss\tiny.\hss}\hfill}
\def\ddotfill#1{\xleaders\hbox to 3.219pt{\hss\lower#1\hbox{\tiny.}\hss}\hfill}
\def\Kr#1{\hbox to #1cm{\Dotfill}}%

\def\Krl#1{\kern0pt\lower.68mm\hbox{\Kr{#1}}}%1mm
\def\Krd#1{\kern0pt\lower1.3mm\hbox{\Kr{#1}}}%1mm
\def\Krdd#1{\kern0pt\lower1.6mm\hbox{\Kr{#1}}}%1mm
\def\kr{\hbox{$\ldots$}}%

\def\Kropky#1{\bgroup\hskip0mm\hbox to #1mm{\small
\setlength{\unitlength}{1mm}
\multido{\n=0+1}{#1}{%
\put(\n,0){.}}
\hss}\egroup}
\def\kropkiON{\rput(0,0){\phantom{\DF{  }}}}

\def\DF#1{%
\setbox1=\hbox{#1}
\tmpa=\hsize\advance\tmpa by -\wd1 \advance\tmpa by -.5em
\line{\box1\hfill\lower.68mm\hbox to\tmpa{\Dotfill}}}

\def\DFI#1{%
\setbox1=\hbox{#1}
\tmpa=\hsize\advance\tmpa by -\wd1 \advance\tmpa by -.5em
\line{\box1\hfill\lower.68mm\hbox to\tmpa{\Dotfill}}}

\def\DFII#1{%
\setbox1=\hbox{#1}
\tmpa=\hsize\advance\tmpa by -\wd1 \advance\tmpa by -.5em
\line{\box1\hfill\lower.68mm\hbox to\tmpa{\Dotfill}}}

\def\DFD#1{%
\setbox1=\hbox{#1}
\tmpa=\hsize\advance\tmpa by -\wd1 \advance\tmpa by -.5em
\line{\box1\hfil\lower1.5mm\hbox to\tmpa{\Dotfill}}}

\def\DFDD#1{%
\setbox1=\hbox{#1}
\tmpa=\hsize\advance\tmpa by -\wd1 \advance\tmpa by -.5em
\line{\box1\hfil\lower2.5mm\hbox to\tmpa{\Dotfill}}}


%------- Podmiana znaków matematycznych
\XeTeXmathchardef\xleq = 1 0 "02A7D
\XeTeXmathchardef\xgeq = 1 0 "02A7E
\let\leq\xleq
\let\geq\xgeq
\let\le\xleq
\let\ge\xgeq
\XeTeXmathchardef\kat = 1 0 "02222
\XeTeXmathchardef\permil = 1 0 "02030
\def\promil{\hbox{\kern1pt$\permil$}}
\XeTeXmathcode`\: = 1 0 "0003A
\def\proc{\kern1pt\%}
\def\Circ{\raise3pt\zscale{75}\hbox{$\circ$}}
\def\CircII{\raise3pt\zscale{90}\hbox{$\circ$}}
\catcode`\=\active
\catcode`\=\active
\catcode`\Á=\active
\gdef\POLSPACJA {\nobreak\hskip 1.55pt\relax}%
\gdef\PS {\nobreak\hskip 1.55pt\relax}%
\gdef\TWARDASPACJA {\nobreak\hskip 3.1pt\relax}%
\gdef\CWSPACJA {\nobreak\hskip 0.75pt\relax}%
\let\POLSPACJA
\let=\TWARDASPACJA
\letÁ\CWSPACJA
\def\miesz #1#2#3{#1{\Frac{#2}{#3}}}

%------- Wcinki i dwaokna

\def \lwcinka #1 #2
    {\nointerlineskip
     \vskip1.8mm
     \vglue-1.8mm
%     \global \Podpunkt=97
     \global \Wcinka=#1mm
     \kol=6mm
     \dszer=\hsize
     \advance\dszer by -\Wcinka
     \szer=\dszer
     \advance\szer by -\kol
     \def \c ##1{\hbox to#1mm{\hskip2\leftskip\hss ##1\hss}}
     \def \cwc ##1{\hskip0pt \hbox to\szer{\hss ##1\hss}}
     \rightline{\llap{\smash{\hbox{\vtop{\hsize\szer
                \overfullrule=0mm
                \vskip1.8mm
                \rm #2}}}}}
     \begingroup\rightskip\dszer
    }

\def \rwcinka #1 #2
    {\nointerlineskip
     \vskip1.8mm
     \vglue-1.8mm
%     \global\Podpunkt=97
     \global\Wcinka=#1mm
     \kol=6mm
     \dszer=\hsize
     \advance\dszer by -\Wcinka
     \szer=\dszer
     \advance\szer by -\kol
     \def \c ##1{\hskip0mm \hbox to#1mm{\hfil ##1 \hfil}}
     \def \cwc ##1{\noindent\hskip0pt \hbox to\szer{\hfil ##1 \hfil}}
     \rlap{\smash{\hbox{\vtop{\hsize\szer
           \vskip1.8mm
           \rm #2}}}}
     \begingroup\leftskip\dszer
     }

\def \wcinkaEND
    {\par
     \endgroup
     \global\Wcinka\hsize
    }

\long\def\dwaokna #1 #2 #3 #4 #5{\noindent\hbox to#1mm{%
    \vtop{\hsize=#2mm #4}\hfil\vtop{\hsize=#3mm #5}}}

\long\def\trzyokna #1 #2 #3 #4 #5 #6{\noindent\hbox to 161mm{%
    \vtop{\hsize=#1mm #4}\hfil\vtop{\hsize=#2mm #5}\hfil\vtop{\hsize=#3mm #6}\hfil}}
    
\long\def\Trzyokna #1 #2 #3 #4 #5 #6{\noindent\hbox to 161mm{%
    \vtop{\hsize=#1mm #4}\hfill\vtop{\hsize=#2mm #5}\hfill\vtop{\hsize=#3mm #6}}}

\def\Metka #1 {%
\h0
\setbox0\hbox{#1}
\setbox1\hbox{\psframebox[linecolor=black, linewidth=.35mm, fillstyle=none, framesep=2mm, framearc=.4]{\hbox to\wd0{#1}}}
\temp=\wd1
\divide\temp by 2
\hbox{%
\vtop{\hsize=\wd1%
\rput[b](\temp,-\dp1){\psframe[linecolor=black, linewidth=.15mm](-.25,.025)(.25,-.4975)}\hfil\copy1\hfil%
\vskip5mm}\hss}
}


\newcmykcolor{yllwV}{0 0 .018 0}
%------- format zadań, odpowiedzi, podpunktów i inne makra pozycjonujące

\long\def\zad#1\ezad{\vskip2pt\global\Podpunkt=97\psframebox[linecolor=yllwV,linewidth=.5pt,framesep=1mm]{\hskip5mm\vtop{\hsize=165mm\vrule height0pt depth0mm width0pt#1}}}%

\long\def\zadII#1\ezad{\vskip2pt\global\Podpunkt=97\psframebox[linecolor=red,linewidth=.5pt,framesep=1mm]{\hskip5mm\vtop{\hsize=165mm\vrule height0pt depth0mm width0pt#1}}}%
\long\def\ODP#1\ezad{\vskip3mm\global\Podpunkt=97\psframebox[linecolor=blue,linewidth=.5pt,framesep=1mm]{\hskip5mm\vtop{\hsize=165mm\vrule height12pt depth0mm width0pt#1}}}%
\long\def\ODPII#1\ezad{\vskip0pt\global\Podpunkt=97\psframebox[linecolor=blue,linewidth=.5pt,framesep=1mm]{\hskip5mm\vtop{\hsize=165mm\vrule height12pt depth0mm width0pt#1}}}%


\def \ppp
    {\leavevmode%
     \podpunkt%
     \global \advance \Podpunkt by 1\relax
     \ignorespaces}

\def\podpunkt {\hbox to \pindent{\char\Podpunkt)\hss}}

\def \pp
    {\ifvmode\vskip.5mm \ppp\else\ppp\fi}

\def \st #1 #2 {\vrule width0pt height #1mm depth #2mm}
\def \stpt #1 #2 {\vrule width0pt height #1pt depth #2pt}
%\def \strut {\vrule width0pt height 8pt depth 3pt}

\def\ph#1{\phantom{#1}}

\def\doc{{\parfillskip0pt\endgraf}}

\def\hh #1 #2% -#1: od prawej, +#1: od lewej, -#2: od zera do #2 indent, +#2: po #2 linijkach indent
{\hangindent#1cm
\hangafter#2
}

%------- Kratki, \Frame i kółka
\def\koleczko #1%
    {\setbox0=\hbox{%
    \pscircle[linewidth=.3pt](.22,.12){.22}
    {\hbox to 4.4mm{\hss#1\hss}}}%
    \box0\hskip.2mm }



\def\kolkogim{}
\def\kolkosp{}

\def\pkolko{%
    \pscircle[linewidth=0pt,linecolor=Raster,fillstyle=solid,fillcolor=Raster](-.3,.15){.2}}

\def\szkratka{\hbox to 4mm{\psframe[linewidth=.25mm,linecolor=Raster,fillstyle=solid,fillcolor=Raster](0,0)(.4,.4)\hfil}}

\def\szkr{\hbox to 3mm{\psframe[linewidth=.25mm,linecolor=Raster,fillstyle=solid,fillcolor=Raster](0,0)(.3,.3)\hfil}}


\def\mkolko #1%
    {\setbox0=\hbox{%
    \pscircle[linewidth=.3pt](.2,.135){.25}
    {\hbox to 4mm{\hss#1\hss}}}%
    \box0\hskip.73mm }

\let\kolkogim\koleczko
\let\kolkosp\mkolko

\long\def\Frame #1#2#3#4{%
 \vbox{\hrule height#2 \hbox{\vrule width#2
 \hskip#1 \vbox{\vskip#1{}\hsize#3#4\vskip#1}\hskip#1
 \vrule width#2} \hrule height#2}}

\newcmykcolor{colBudek}{0 0 0 .35}
\def\Budka#1{\hbox{\psframe[linewidth=.18mm,framearc=.4, linecolor=colBudek](0,-.1)(.#1,.4)\vrule height4.2mm width0mm depth1.2mm\hskip#1mm}}
\def\bud#1{\hbox{\psframe[linewidth=.18mm,framearc=.4, linecolor=colBudek](0,-.15)(#1,.35)\vrule height3.2mm width0mm depth1.2mm\hskip#1cm}}

\def\graframka#1{\psframebox[framearc=.4,framesep=2mm]{\vrule height10.5pt depth3.5pt width0pt\hbox to 15mm{\hss#1\hss}}}
\def\kleks{\lower4.5pt\hbox{\XeTeXpdffile 'pdf/kleks.pdf' }}
\def\kratka{\zscale{100}\Frame{0pt}{.3pt}{4mm}{\vbox to 5.2mm{\hsize=4mm\hfil\vfil}}}
\def\kratkaSz{\zscale{100}\Frame{0pt}{.3pt}{4mm}{\vbox to 5.2mm{\hsize=6mm\hfil\vfil}}}
\def\krat{\lower2mm\hbox{\Frame{0pt}{.3pt}{4mm}{\vbox to 5.5mm{\hsize=4mm\hfil\vfil}}}}
\def\kratkw{\lower1mm\hbox{\Frame{0pt}{.3pt}{4mm}{\vbox to 4mm{\hsize=4mm\hfil\vfil}}}\kern1.5mm\ignorespaces} % kratka do makr NIE RUSZAC!!!
\def\kratsc{\kern.3pt\lower.5mm\hbox{\Frame{0pt}{.3pt}{2.5mm}{\vbox to 2.5mm{\hsize=2.5mm\hfil\vfil}}}}
\def\kratt{\lower1.2mm\hbox{\Frame{0pt}{.3pt}{8mm}{\vbox to 4.5mm{\hsize=4.5mm\hfil\vfil}}}}
\def\kratul{\lower2mm\hbox{\Frame{0pt}{.3pt}{4mm}{\vbox to 8.5mm{\hsize=5.5mm\hfil\vfil}}}}
\def\kratKW{\lower1mm\hbox{\Frame{0pt}{.3pt}{4mm}{\vbox to 4mm{\hsize=4mm\hfil\vfil}}}}
\def\kratKWTAB{\psframe[linecolor=black, linewidth=.3pt](-2mm,-.7mm)(2mm,3.5mm)}
\def\KR{\lower2mm\hbox{\Frame{0pt}{.3pt}{7mm}{\vbox to 6mm{\hsize=7mm\hfil\vfil}}}}
\def\KRR{\lower1.25mm\hbox{\Frame{0pt}{.3pt}{9mm}{\vbox to 4.5mm{\hsize=9mm\hfil\vfil}}}}           % 2 cyfry luzem SP
\def\frejm{\lower3pt\hbox{\Frame{0pt}{0.3pt}{4mm}{\vbox to 12pt{\hsize=10mm\hfil\vfil}}}} %221 janowicz 15
\def\arrow#1#2#3{\raise3pt\hbox to 12mm{\rput(.5,.25){$#1{#2}^{#3}$}\psline{->}(1,0)}} %214 janowicz 8
\def\strzalkado#1{\raise3pt\hbox to 14mm{\rput[B](.7,.25){#1}\psline{->}(.1,0)(1.3,0)}}
\def\Kwadiiv{\lower1.5mm\hbox{\Frame{2.5mm}{.3pt}{6mm}{}}}  %214 janowicz 8
\def\mkw{\vbox{\hrule %
               \hbox{\vrule \vrule depth1mm height1mm width0mm \hskip2mm \vrule}%
               \hrule}}

\def\kwadracik{\noindent\lower.75mm\vbox{\hrule %
               \hbox{\vrule \vrule depth3mm height2mm width0mm \hskip4mm \vrule}%
               \hrule}}

\def\Kwadracik{\noindent\lower.75mm\vbox{\hrule %
               \hbox{\vrule \vrule depth3mm height2mm width0mm \hskip6mm \vrule}%
               \hrule}}


%------- makra do zadań: odp, pf, tn, tnp
\def\forAga{{\sf X\ }}


\long\def\odp #1 #2 #3 #4
     {{\nobreak\overfullrule=0pt
      \vskip2mm
      \nobreak
      \mathsurround=0pt
      \setbox0\hbox{{\sf A.}~#1}
      \setbox1\hbox{{\sf B.}~#2}
      \setbox2\hbox{{\sf C.}~#3}
      \setbox3\hbox{{\sf D.}~#4}
      \setbox4\hbox{#1}
      \setbox5\hbox{#3}
      \ht4=0mm\ht5=0mm\wd5=0mm%
      \ifnum\wd0>\wd1\relax\SzerOka=\wd0\else\SzerOka=\wd1\relax\fi
      \ifnum\wd2>\wd3\relax\Szeroka=\wd2\else\Szeroka=\wd3\relax\fi
\SPR=\wd1\advance\SPR by\wd2\advance\SPR by\wd3\advance\SPR by\wd4\advance\SPR by30mm
\ifdim\SPR<\hsize
\hskip0pt \hbox to\hsize{%
      {\sf {A.}}~#1\hskip9.3mm plus.5mm minus5mm
      {\sf {B.}}~#2\hskip9.3mm plus.5mm minus5mm
      {\sf {C.}}~#3\hskip9.3mm plus.5mm minus5mm
      {\sf {D.}}~#4\hfil\hss}\vskip0pt
\else
\SPR=\SzerOka\advance\SPR by \Szeroka \advance\SPR by 15mm
\ifdim\SPR<\hsize
      \hskip0pt \hbox to\hsize{\leftskip=0mm%
              \vbox{\hsize=\SzerOka
      \line{{\sf A.}~\box4\hfil}%
      \vskip1mm
      \line{{\sf B.}~\hbox{#2}\hfil}}\hskip20mm plus.5mm minus1mm
                  \vbox{\hsize=\Szeroka
      \line{{\sf C.}~\box5\hfil}%
      \vskip1mm
      \line{{\sf D.}~\hbox{#4}\hfil}%
               }\hss}%
\else
\ifdim\SzerOka>\Szeroka \SPR=\SzerOka\else\SPR=\Szeroka\fi
\hangindent4.5mm\hangafter1
{\sf A.}~#1
\vskip1mm
\hangindent4.5mm\hangafter1
{\sf B.}~#2
\vskip1mm
\hangindent4.5mm\hangafter1
{\sf C.}~#3
\vskip1mm
\hangindent4.5mm\hangafter1
{\sf D.}~#4
\vskip1mm
\fi\fi
}}



\long\def\pf #1 #2 #3 #4 #5 #6                  %makro do zadań prawda/fałsz w kompozytorze dla gimnazjum. Minimum 2 zdania, maksimum 6.
     {{\nobreak\overfullrule=0pt                %wstępne skopiowane z \odp
      \vskip2mm
      \nobreak
      \mathsurround=0pt
      \setbox0\hbox{#1}                         %pudełkowanie
      \setbox1\hbox{#2}
      \setbox2\hbox{#3}
      \setbox3\hbox{#4}
      \setbox4\hbox{#5}
      \setbox5\hbox{#6}
      \setbox6\hbox to 164pt{\hss\kratkw prawda \hskip5pt \kratkw fałsz}
      \setbox7\vtop{\hsize=164pt\hskip2cm\kratkw prawda \hskip5pt \kratkw fałsz}
      \ifnum\wd3>\wd0\relax\wd0=\wd3\relax\fi   %spr wielkośći par 1/4, 2/5, 3/6
      \ifnum\wd4>\wd1\relax\wd1=\wd4\relax\fi
      \ifnum\wd5>\wd2\relax\wd2=\wd5\relax\fi
      \ifnum\wd0>\wd1\relax\SzerOka=\wd0\relax
            \else\SzerOka=\wd1\relax\fi %spr największego pudełka z trójki 123
      \ifnum\wd2>\SzerOka\relax\SzerOka=\wd2\relax\fi
      \SPR=\hsize\advance\SPR by -164pt         %granica to \hsize pomniejszony o pudełko prawda/fałsz i skipy
      \ifdim\SzerOka<\SPR                       %jak się mieści to hboxy
            \vskip.4\baselineskip\hskip0pt\hbox to \SzerOka{#1\hss}\copy6
            \vskip.4\baselineskip\hskip0pt\hbox to \SzerOka{#2\hss}\copy6
            \ifnum\wd2>0\relax\vskip.4\baselineskip\hskip0pt\hbox to \SzerOka{#3\hss}\copy6\relax\fi %nie wypisuj zerowych 3456
            \ifnum\wd3>0\relax\vskip.4\baselineskip\hskip0pt\hbox to \SzerOka{#4\hss}\copy6\relax\fi
            \ifnum\wd4>0\relax\vskip.4\baselineskip\hskip0pt\hbox to \SzerOka{#5\hss}\copy6\relax\fi
            \ifnum\wd5>0\relax\vskip.4\baselineskip\hskip0pt\hbox to \SzerOka{#6\hss}\copy6\relax\fi
            \vskip0pt
        \else \advance\SPR by -.1pt                %jak nie to vtopy
            \vskip.4\baselineskip\hskip0pt\vtop{\hsize=\SPR #1}\copy7
            \vskip.4\baselineskip\hskip0pt\vtop{\hsize=\SPR #2}\copy7
            \ifnum\wd2>0\relax\vskip.4\baselineskip\hskip0pt\vtop{\hsize=\SPR #3}\copy7\relax\fi
            \ifnum\wd3>0\relax\vskip.4\baselineskip\hskip0pt\vtop{\hsize=\SPR #4}\copy7\relax\fi
            \ifnum\wd4>0\relax\vskip.4\baselineskip\hskip0pt\vtop{\hsize=\SPR #5}\copy7\relax\fi
            \ifnum\wd5>0\relax\vskip.4\baselineskip\hskip0pt\vtop{\hsize=\SPR #6}\copy7\relax\fi
            \vskip0pt
      \fi}}

\long\def\tn #1 #2 #3 #4 #5 #6                  %makro do zadań tak/nie w kompozytorze dla gimnazjum. Minimum 3 zdania, maksimum 6. Różni się od \pf tylko pudełkami z prawej strony
     {{\nobreak\overfullrule=0pt                %wstępne skopiowane z \odp
      \vskip2mm
      \nobreak
      \mathsurround=0pt
      \setbox0\hbox{#1}                         %pudełkowanie
      \setbox1\hbox{#2}
      \setbox2\hbox{#3}
      \setbox3\hbox{#4}
      \setbox4\hbox{#5}
      \setbox5\hbox{#6}
      \setbox6\hbox to 144pt{\hskip2cm\kratkw TAK\hskip5mm\kratkw NIE}
      \setbox7\vtop{\hsize=144pt\hskip2cm\kratkw TAK\hskip5mm\kratkw NIE}
      \ifnum\wd3>\wd0\relax\wd0=\wd3\relax\fi   %spr wielkośći par 1/4, 2/5, 3/6
      \ifnum\wd4>\wd1\relax\wd1=\wd4\relax\fi
      \ifnum\wd5>\wd2\relax\wd2=\wd5\relax\fi
      \ifnum\wd0>\wd1\relax\SzerOka=\wd0\relax\else\SzerOka=\wd1\relax\fi %spr największego pudełka z trójki 123
      \ifnum\wd2>\SzerOka\relax\SzerOka=\wd2\relax\fi
      \SPR=\hsize\advance\SPR by -144pt         %granica to \hsize pomniejszony o pudełko tak/nie
      \ifdim\SzerOka<\SPR                       %jak się mieści to hboxy
            \vskip.4\baselineskip\hskip0pt\hbox to \SzerOka{#1\hss}\copy6
            \vskip.4\baselineskip\hskip0pt\hbox to \SzerOka{#2\hss}\copy6
            \ifnum\wd2>0\relax\vskip.4\baselineskip\hskip0pt\hbox to \SzerOka{#3\hss}\copy6\relax\fi %nie wypisuj zerowych 3456
            \ifnum\wd3>0\relax\vskip.4\baselineskip\hskip0pt\hbox to \SzerOka{#4\hss}\copy6\relax\fi
            \ifnum\wd4>0\relax\vskip.4\baselineskip\hskip0pt\hbox to \SzerOka{#5\hss}\copy6\relax\fi
            \ifnum\wd5>0\relax\vskip.4\baselineskip\hskip0pt\hbox to \SzerOka{#6\hss}\copy6\relax\fi
            \vskip0pt
      \else \advance\SPR by -0.1pt                %jak nie to vtopy
            \vskip.4\baselineskip\hskip0pt\vtop{\hsize=\SPR #1}\copy7
            \vskip.4\baselineskip\hskip0pt\vtop{\hsize=\SPR #2}\copy7
            \ifnum\wd2>0\relax\vskip.4\baselineskip\hskip0pt\vtop{\hsize=\SPR #3}\copy7\relax\fi
            \ifnum\wd3>0\relax\vskip.4\baselineskip\hskip0pt\vtop{\hsize=\SPR #4}\copy7\relax\fi
            \ifnum\wd4>0\relax\vskip.4\baselineskip\hskip0pt\vtop{\hsize=\SPR #5}\copy7\relax\fi
            \ifnum\wd5>0\relax\vskip.4\baselineskip\hskip0pt\vtop{\hsize=\SPR #6}\copy7\relax\fi
            \vskip0pt
      \fi}}
	
\long\def\pfII #1 #2 #3 #4 #5 #6 #7             %makro do zadań prawda/fałsz w kompozytorze dla gimnazjum. Minimum 2 zdania, maksimum 6.
     {{\nobreak\overfullrule=0pt                %wstępne skopiowane z \odp
      \vskip2mm
      \nobreak
      \mathsurround=0pt
      \setbox0\hbox{#2}                         %pudełkowanie
      \setbox1\hbox{#3}
      \setbox2\hbox{#4}
      \setbox3\hbox{#5}
      \setbox4\hbox{#6}
      \setbox5\hbox{#7}
      \setbox6\hbox to #1pt{\hss\kratkw prawda \hskip5pt \kratkw fałsz}
      \setbox7\vtop{\hsize=#1pt\hfill\kratkw prawda \hskip5pt \kratkw fałsz}
      \ifnum\wd3>\wd0\relax\wd0=\wd3\relax\fi   %spr wielkośći par 1/4, 2/5, 3/6
      \ifnum\wd4>\wd1\relax\wd1=\wd4\relax\fi
      \ifnum\wd5>\wd2\relax\wd2=\wd5\relax\fi
      \ifnum\wd0>\wd1\relax\SzerOka=\wd0\relax
            \else\SzerOka=\wd1\relax\fi %spr największego pudełka z trójki 123
      \ifnum\wd2>\SzerOka\relax\SzerOka=\wd2\relax\fi
      \SPR=\hsize\advance\SPR by -#1pt         %granica to \hsize pomniejszony o pudełko prawda/fałsz i skipy
      \ifdim\SzerOka<\SPR                       %jak się mieści to hboxy
            \vskip.4\baselineskip\hskip0pt\hbox to \SzerOka{#2\hss}\copy6
            \vskip.4\baselineskip\hskip0pt\hbox to \SzerOka{#3\hss}\copy6
            \ifnum\wd2>0\relax\vskip.4\baselineskip\hskip0pt\hbox to \SzerOka{#4\hss}\copy6\relax\fi %nie wypisuj zerowych 3456
            \ifnum\wd3>0\relax\vskip.4\baselineskip\hskip0pt\hbox to \SzerOka{#5\hss}\copy6\relax\fi
            \ifnum\wd4>0\relax\vskip.4\baselineskip\hskip0pt\hbox to \SzerOka{#6\hss}\copy6\relax\fi
            \ifnum\wd5>0\relax\vskip.4\baselineskip\hskip0pt\hbox to \SzerOka{#7\hss}\copy6\relax\fi
            \vskip0pt
        \else \advance\SPR by -.1pt                %jak nie to vtopy
            \vskip.4\baselineskip\hskip0pt\vtop{\hsize=\SPR #2}\copy7
            \vskip.4\baselineskip\hskip0pt\vtop{\hsize=\SPR #3}\copy7
            \ifnum\wd2>0\relax\vskip.4\baselineskip\hskip0pt\vtop{\hsize=\SPR #4}\copy7\relax\fi
            \ifnum\wd3>0\relax\vskip.4\baselineskip\hskip0pt\vtop{\hsize=\SPR #5}\copy7\relax\fi
            \ifnum\wd4>0\relax\vskip.4\baselineskip\hskip0pt\vtop{\hsize=\SPR #6}\copy7\relax\fi
            \ifnum\wd5>0\relax\vskip.4\baselineskip\hskip0pt\vtop{\hsize=\SPR #7}\copy7\relax\fi
            \vskip0pt
      \fi}}

\long\def\tnII #1 #2 #3 #4 #5 #6 #7             %makro do zadań tak/nie w kompozytorze dla gimnazjum. Minimum 3 zdania, maksimum 6. Różni się od \pf tylko pudełkami z prawej strony
     {{\nobreak\overfullrule=0pt                %wstępne skopiowane z \odp
      \vskip2mm
      \nobreak
      \mathsurround=0pt
      \setbox0\hbox{#2}                         %pudełkowanie
      \setbox1\hbox{#3}
      \setbox2\hbox{#4}
      \setbox3\hbox{#5}
      \setbox4\hbox{#6}
      \setbox5\hbox{#7}
      \setbox6\hbox to #1pt{\hss\kratkw TAK\hskip5mm\kratkw NIE}
      \setbox7\vtop{\hsize=#1pt\hfill\kratkw TAK\hskip5mm\kratkw NIE}
      \ifnum\wd3>\wd0\relax\wd0=\wd3\relax\fi   %spr wielkośći par 1/4, 2/5, 3/6
      \ifnum\wd4>\wd1\relax\wd1=\wd4\relax\fi
      \ifnum\wd5>\wd2\relax\wd2=\wd5\relax\fi
      \ifnum\wd0>\wd1\relax\SzerOka=\wd0\relax\else\SzerOka=\wd1\relax\fi %spr największego pudełka z trójki 123
      \ifnum\wd2>\SzerOka\relax\SzerOka=\wd2\relax\fi
      \SPR=\hsize\advance\SPR by -#1pt         %granica to \hsize pomniejszony o pudełko tak/nie
      \ifdim\SzerOka<\SPR                       %jak się mieści to hboxy
            \vskip.4\baselineskip\hskip0pt\hbox to \SzerOka{#2\hss}\copy6
            \vskip.4\baselineskip\hskip0pt\hbox to \SzerOka{#3\hss}\copy6
            \ifnum\wd2>0\relax\vskip.4\baselineskip\hskip0pt\hbox to \SzerOka{#4\hss}\copy6\relax\fi %nie wypisuj zerowych 3456
            \ifnum\wd3>0\relax\vskip.4\baselineskip\hskip0pt\hbox to \SzerOka{#5\hss}\copy6\relax\fi
            \ifnum\wd4>0\relax\vskip.4\baselineskip\hskip0pt\hbox to \SzerOka{#6\hss}\copy6\relax\fi
            \ifnum\wd5>0\relax\vskip.4\baselineskip\hskip0pt\hbox to \SzerOka{#7\hss}\copy6\relax\fi
            \vskip0pt
      \else \advance\SPR by -0.1pt                %jak nie to vtopy
            \vskip.4\baselineskip\hskip0pt\vtop{\hsize=\SPR #2}\copy7
            \vskip.4\baselineskip\hskip0pt\vtop{\hsize=\SPR #3}\copy7
            \ifnum\wd2>0\relax\vskip.4\baselineskip\hskip0pt\vtop{\hsize=\SPR #4}\copy7\relax\fi
            \ifnum\wd3>0\relax\vskip.4\baselineskip\hskip0pt\vtop{\hsize=\SPR #5}\copy7\relax\fi
            \ifnum\wd4>0\relax\vskip.4\baselineskip\hskip0pt\vtop{\hsize=\SPR #6}\copy7\relax\fi
            \ifnum\wd5>0\relax\vskip.4\baselineskip\hskip0pt\vtop{\hsize=\SPR #7}\copy7\relax\fi
            \vskip0pt
      \fi}}

\def\tnbox{\centerline{\hbox{\kratkw TAK\hskip5mm\kratkw NIE}}}

\def\tnpon{\hbox to 127pt{\noindent \vbox{\hbox{\koleczko{}\hskip2mm TAK,}%
\vskip3mm
\hbox{\koleczko{} \hskip2mm NIE,}} \rput(36pt,12.3pt){ponieważ}\hss}}

\long\def\tnp #1 #2 #3 #4 #5 #6                  %makro do zadań "tak,/nie, ponieważ" w kompozytorze dla gimnazjum. Minimum 2 zdania, maksimum 6.
     {\nobreak\overfullrule=0pt
      \vskip2mm
      \nobreak
      \mathsurround=0pt
      \SzerOka=0pt
      \setbox0\hbox{#1}                          %pudełkowanie
      \setbox1\hbox{#2}
      \setbox2\hbox{#3}
      \setbox3\hbox{#4}
      \setbox4\hbox{#5}
      \setbox5\hbox{#6}
      \setbox6\vtop{\hsize=352pt                  %tworzenie dużego pudełka, które będzie z prawej strony kartki
        \hskip0pt \kratkw \vtop{\hsize=320pt #1}
        \vskip2.5mm \hskip0pt \kratkw \vtop{\hsize=320pt #2}
        \ifnum\wd2>0\relax\vskip2.5mm\hskip0pt\kratkw\vtop{\hsize=320pt #3}\relax\fi
        \ifnum\wd3>0\relax\vskip2.5mm\hskip0pt\kratkw\vtop{\hsize=320pt #4}\relax\fi
        \ifnum\wd4>0\relax\vskip2.5mm\hskip0pt\kratkw\vtop{\hsize=320pt #5}\relax\fi
        \ifnum\wd5>0\relax\vskip2.5mm\hskip0pt\kratkw\vtop{\hsize=320pt #6}\relax\fi}
      \advance \SzerOka by \ht6               %poprawne wstawienie rputem lewego pudełka z "tak,/nie, ponieważ"
      \advance \SzerOka by \dp6
      \divide  \SzerOka by 2
      \advance\SzerOka by -13pt
      \vskip0pt\noindent
      \rput[l](0,-\SzerOka){\tnpon}
      \hskip117pt                             %umiejscowienie i wypisanie prawego pudełka
      \copy6
      \vskip2mm}

%------- działania pisemne


\newmuskip\defaultmuskip \defaultmuskip=6mu
\def\fifo #1{\ifx\ofif#1\ofif\fi\process#1\fifo}
\def\ofif #1\fifo{\fi}

\def\slupek#1#2#3#4% [+/-] x y wynik
   {\vtop{\def\process ##1{\hbox to4.7mm{\hfil ##1\hfil}}%
      \def\mhrulefill{\leaders\hrule height.3mm\hfill}
      \normalbaselines
      \ialign{\strut\hfil$##$&&\hfil$##$\hfil\crcr
              \strut\crcr\noalign{\vskip-\baselineskip}
                &\fifo#2\ofif\crcr\noalign{\vskip-.4mm}
              \hbox to4.7mm{\hfil$#1$\hfil}&\fifo#3\ofif\crcr\crcr\noalign{\vskip-4.05mm}
              \multispan2\mhrulefill\cr
                &\fifo#4\ofif\crcr
              \strut\crcr\noalign{\vskip-1\baselineskip}}}}%

\def\slupekII#1#2#3% [+/-] nad pod
   {\vtop{\def\process ##1{\hbox to4.7mm{\hfil ##1\hfil}}%
      \def\mhrulefill{\leaders\hrule height.3mm\hfill}
      \normalbaselines
      \ialign{\strut\hfil$##$&&\hfil$##$\hfil\crcr
              \hbox to4.7mm{\hfil$#1$\hfil}&\fifo#2\ofif\crcr\crcr\noalign{\vskip-4.05mm}
              \multispan2\mhrulefill\cr
                &\fifo#3\ofif\crcr
              \strut\crcr\noalign{\vskip-\baselineskip}}}}%

%------- działania z luką

\def\w#1{\leavevmode\kern0pt\raise4.7mm\hbox{$#1$}}
\def\dod{\vbox{\hsize1.5cm\hbox to\hsize{\h1 \dotfill\h1 }\v-1 \vskip-.8mm\centerline{{\scriptsize dodatnia}}}}
\def\uj{\vbox{\hsize1.5cm\hbox to\hsize{\h1 \dotfill\h1 }\v-1 \vskip-.8mm\centerline{{\scriptsize ujemna}}}}
\def\dodk{\vbox{\hsize1.5cm\hbox to\hsize{\h1 \dotfill\h1 }\v-1 \vskip-.8mm\centerline{{\scriptsize dodatnia}}}}
\def\ujk{\vbox{\hsize1.5cm\hbox to\hsize{\h1 \dotfill\h1 }\v-1 \vskip-.8mm\centerline{{\scriptsize ujemna}}}}
\def\ze{\vbox{\hsize1.5cm\hbox to\hsize{\h1 \dotfill\h1 }\v-1 \vskip-.8mm\centerline{{\scriptsize zero}}}}
\def\podpis #1 #2 {\vbox{\hsize=#1\noindent\lower0pt\hbox to\hsize{\h1 \dotfill\h1 } \noindent\raise7pt\centerline{{\scriptsize #2}}\vskip-18pt}}

\def\mlhr#1{\multispan#1\hrulefill}

			
%------- Rysunki


\def \NazwaEPS #1{%
     \ifsklad\rput[tl](-.7,0){{\NazwaEPSfont [#1]}}\else\fi}

\def \Lpic #1 #2 #3%
    {\vskip#1mm
        \leftline {\NazwaEPS {#3}\XeTeXpdffile '#3' }
     \vskip#2mm
    }
%-------------------
\def \Rpic #1 #2 #3%
    {\vskip#1mm
        \rightline {\NazwaEPS {#3}\XeTeXpdffile '#3'\hskip-.5mm }
     \vskip#2mm
    }
%-------------------
\def \Cpic #1 #2 #3%
    {\vskip#1mm
        \centerline {\NazwaEPS {#3}\XeTeXpdffile '#3' }
     \vskip#2mm
    }

\def \Lspic #1 #2 #3 #4%
    {\vskip#1mm
        \leftline {\NazwaEPS {#3}\zscale{#4}\vbox{\XeTeXpdffile '#3' } }
     \vskip#2mm
    }
%-------------------
\def \Rspic #1 #2 #3 #4%
    {\vskip#1mm
        \rightline {\NazwaEPS {#3}\zscale{#4}\vbox{\XeTeXpdffile '#3' }\hskip-.5mm }
     \vskip#2mm
    }
%-------------------
\def \Cspic #1 #2 #3 #4%
    {\vskip#1mm
        \centerline {\NazwaEPS {#3}\zscale{#4}\vbox{\XeTeXpdffile '#3' } }
     \vskip#2mm
    }

\def\RysK#1{\v3 \NazwaEPS{#1}\leftline{\h-.75 \XeTeXpdffile '#1' \hss}\v3 }

\def \lpic #1 #2 #3%
    {\rput[tl](#1,#2){\leftline{\NazwaEPS {#3}\XeTeXpdffile '#3' }}}

\def \lspic #1 #2 #3 #4%
    {\rput[tl](#1,#2){\leftline{\NazwaEPS {#3}\zscale{#4}\hbox{\XeTeXpdffile '#3' }}}}

\def \cpic #1 #2 #3%
    {\rput[tl](#1,#2){\centerline{\NazwaEPS {#3}\XeTeXpdffile '#3' }}}

\def \cspic #1 #2 #3 #4%
    {\rput[tl](#1,#2){\centerline{\NazwaEPS {#3}\zscale{#4}\hbox{\XeTeXpdffile '#3' }}}}

\def \rspic #1 #2 #3 #4%
    {\rput[tl](#1,#2){\rightline{\NazwaEPS {#3}\zscale{#4}\hbox{\XeTeXpdffile '#3' }}}}

\def \rpic #1 #2 #3%
    {\rput[tl](#1,#2){\rightline{\NazwaEPS {#3}\XeTeXpdffile '#3' }}}


%------- \naklejka
%---------------------------

\newcmykcolor{Raster}{0 0 0 .3}
%-------------------------------------Shadow--------------------------------------
% Parametry komendy BaseBlok
% #1 -- odci‘cie cienia od g˘ry (prawy r˘g) -- dimen
% #2 -- grubožŤ cienia z boku (prawego)     -- dimen
% #3 -- odci‘cie cienia z do’u (lewy r˘g)   -- dimen
% #4 -- grubožŤ cienia z do’u               -- dimen
% #5 -- tekst w ramce                       -- text
\long\def\BaseBlok#1#2#3#4#5{%
 \vbox{\setbox0=\hbox{#5} \offinterlineskip
 \hbox{\copy0\dimen0=\ht0 \advance\dimen0 by -#1
       {\Raster \vrule height\dimen0 width #2}}
 \hbox{\hskip#3\dimen0=\wd0
   \advance\dimen0 by -#3 \advance\dimen0 by #2
      {\Raster \vrule height #4 width\dimen0}} }}

\long\def\Shadow#1{\BaseBlok{1.5mm}{1mm}{1.5mm}{1mm}{#1}}

\long \def \SFrame #1 #2%
   {\Shadow{%
    \Frame{3.5mm}{0.08mm}{#1cm}{#2}%
    }}

% naklejka z ramk† TeX-ow†
\long \def \naklejka #1 #2 #3 #4 #5%
    {\setbox0 \vbox {\leftskip=0pt
                      \SFrame #3 {\vglue-1mm\relax%
                      #5%
                     }}
     \rput[br](#1,#2){\rotate{#4}\hbox{\box0}}
    }







%%%%%%%%%%%%%%%%%%%%%%%%%%%%%%%%%%%%
%------- Funkcje czasu i godziny


%----------------------------------
\newcount\hours
\newcount\minutes
\def\oktime{% format `hh:mm'
\hours=\time \divide \hours by 60 %
\minutes=-\hours \multiply \minutes by 60 \advance \minutes by \time
\ifnum\hours>9 \the\hours \else 0\the\hours \fi %
:%
\ifnum\minutes>9 \the\minutes \else 0\the\minutes \fi}
%----------------------------------
\def\monthnazwa{%
\ifcase \month%
\or stycznia\or lutego\or marca\or kwietnia\or maja\or czerwca\or lipca%
\or sierpnia\or września\or października\or listopada\or grudnia%
\fi}%

\newcount\licznik
\licznik=0
\def\kreska{\ifodd\licznik\vrule height1mm depth0mm width.4pt\rput[Bl](.2,0){\tiny\the\licznik}\newline\relax\else\vrule height1mm depth0mm width0pt\newline\relax\fi\advance\licznik by 1}

%--------------------------------------------------------------------
%%%%%%%%%%%%%%%%%%%%%%%%%%%%%%%%%%%%
%------- Makro tabelka i cała reszta



\newcount \dPodpunkt \dPodpunkt=65
\newdimen \dpindent \dpindent=5mm
\newdimen \hboxa \dpindent=33.5mm

\newcount\Nrboxa\Nrboxa=1
\newcount\Lboxow\Lboxow=0
\newcount\Nrkolumn\Nrkolumn=0
\newcount\Lkolumn\Lkolumn=0
\newcount\Nrwiersz\Nrwiersz=1
\newcount\Lwiersz\Lwiersz=0
\newcount\iloczyn\iloczyn=0

\newcount\Podstep\Podstep=0
\newdimen \Wys
\newdimen\Lbazowa\Lbazowa=1.5mm   %standartowy odstep pomiedzy wierszami tabeli
\newdimen\roznica\roznica=0mm
\newdimen\Wcinka \Wcinka=165mm
\newdimen\Podstep\Podstep=0mm

\gdef\pB #1 {{\global\setbox\Nrboxa\hbox to\hboxa{#1\hss}
              \global\advance\Nrboxa by1
             }}

\gdef\pBB #1 {{\global\setbox\Nrboxa\hbox to\hboxa{\h4.5 #1\hss}
              \global\advance\Nrboxa by1
             }}

\gdef\PB #1 {{\global\setbox\Nrboxa\hbox to33.5mm{#1\hss}
              \global\advance\Nrboxa by1
             }}


\gdef\p #1 {{\global\setbox\Nrboxa\hbox to\hboxa{\podpunkt#1\hss}
             \global\advance\Podpunkt by1
             \global\advance\Nrboxa by1
            }}

\def \tabelka #1 #2 #3
    {\global\Podpunkt=97
     \global\Nrboxa=0
     \ifnum#1=2\global\hboxa=67mm\fi%
     \ifnum#1=3\global\hboxa=43mm\fi%
     \ifnum#1=4\global\hboxa=33.5mm\fi%
     \global\Lkolumn=#1
     \global\Lwiersz=0
     \global\Nrkolumn=1          %kolumny numerowane są od jedynki
     \global\Nrwiersz=0          %wiersze od 0
     \global\Lbazowa=1.5mm
     \global\advance\Lbazowa by#2mm
     \global\Podstep=10mm
     \global\advance\Podstep by #3mm
     }


% makro przewiduje od 1 do 6 kolumn i maksymalnie 11 wierszy!
\def \tabelkaEND {{\vskip1.5mm
                  \global\Lboxow=\Nrboxa
%-----------------------------------
                  \global\iloczyn\Nrwiersz
                  \loop\ifnum\iloczyn<\Lboxow
                    \global\advance\Lwiersz by1
                    \global\iloczyn=\Lwiersz
                    \global\multiply\iloczyn by\Lkolumn
                  \repeat
%------------------------------------
\hfuzz30pt
% Ustalamy nr boxa w tabeli.
                  \def\UNB{{%
                      \global\Nrboxa=\Nrkolumn
                      \global\multiply\Nrboxa by\Lwiersz
                      \global\advance\Nrboxa by\Nrwiersz
                      \global\advance\Nrboxa by-\Lwiersz
                      \ifnum\Nrkolumn<\Lkolumn
                         \global\advance\Nrkolumn by 1
                      \else
                         \global\Nrkolumn=1
                         \global\advance\Nrwiersz by 1
                      \fi
                      }}
%------------------------------------
                  \global\iloczyn=11
                  \global\advance\iloczyn by -\Lwiersz
                  \h0 \setbox0\vbox{%
                  \ifcase\Lkolumn%
                  \or
                     \halign {\tabskip=\Podstep%10mm plus30mm minus3mm
                     \UNB\box\Nrboxa##\tabskip=0mm\cr
                       \cr\noalign{\vskip\Lbazowa}
                       \cr\noalign{\vskip\Lbazowa}
                       \cr\noalign{\vskip\Lbazowa}
                       \cr\noalign{\vskip\Lbazowa}
                       \cr\noalign{\vskip\Lbazowa}
                       \cr\noalign{\vskip\Lbazowa}
                       \cr\noalign{\vskip\Lbazowa}
                       \cr\noalign{\vskip\Lbazowa}
                       \cr\noalign{\vskip\Lbazowa}
                       \cr\noalign{\vskip\Lbazowa}
                       \cr}%
                  \or
                     \halign {\tabskip=\Podstep%10mm plus30mm minus3mm
                     \UNB\box\Nrboxa##&\UNB\box\Nrboxa##\tabskip0mm\cr
                       &\cr\noalign{\vskip\Lbazowa}
                       &\cr\noalign{\vskip\Lbazowa}
                       &\cr\noalign{\vskip\Lbazowa}
                       &\cr\noalign{\vskip\Lbazowa}
                       &\cr\noalign{\vskip\Lbazowa}
                       &\cr\noalign{\vskip\Lbazowa}
                       &\cr\noalign{\vskip\Lbazowa}
                       &\cr\noalign{\vskip\Lbazowa}
                       &\cr\noalign{\vskip\Lbazowa}
                       &\cr\noalign{\vskip\Lbazowa}
                       &\cr}%
                  \or
                     \halign {\tabskip=\Podstep%10mm plus30mm minus3mm
                     \UNB\box\Nrboxa##&\UNB\box\Nrboxa##&\UNB\box\Nrboxa##\tabskip0mm\cr
                       &&\cr\noalign{\vskip\Lbazowa}
                       &&\cr\noalign{\vskip\Lbazowa}
                       &&\cr\noalign{\vskip\Lbazowa}
                       &&\cr\noalign{\vskip\Lbazowa}
                       &&\cr\noalign{\vskip\Lbazowa}
                       &&\cr\noalign{\vskip\Lbazowa}
                       &&\cr\noalign{\vskip\Lbazowa}
                       &&\cr\noalign{\vskip\Lbazowa}
                       &&\cr\noalign{\vskip\Lbazowa}
                       &&\cr\noalign{\vskip\Lbazowa}
                       &&\cr}%
                  \or
                     \halign {\tabskip=\Podstep%10mm plus30mm minus3mm
                     \UNB\box\Nrboxa##&\UNB\box\Nrboxa##&%
                     \UNB\box\Nrboxa##&\UNB\box\Nrboxa##\tabskip0mm\cr
                       &&&\cr\noalign{\vskip\Lbazowa}
                       &&&\cr\noalign{\vskip\Lbazowa}
                       &&&\cr\noalign{\vskip\Lbazowa}
                       &&&\cr\noalign{\vskip\Lbazowa}
                       &&&\cr\noalign{\vskip\Lbazowa}
                       &&&\cr\noalign{\vskip\Lbazowa}
                       &&&\cr\noalign{\vskip\Lbazowa}
                       &&&\cr\noalign{\vskip\Lbazowa}
                       &&&\cr\noalign{\vskip\Lbazowa}
                       &&&\cr\noalign{\vskip\Lbazowa}
                       &&&\cr}%
                  \or
                     \halign {\tabskip=\Podstep%10mm plus30mm minus3mm
                     \UNB\box\Nrboxa##&\UNB\box\Nrboxa##&\UNB\box\Nrboxa##&%
                     \UNB\box\Nrboxa##&\UNB\box\Nrboxa##\tabskip0mm\cr
                       &&&&\cr\noalign{\vskip\Lbazowa}
                       &&&&\cr\noalign{\vskip\Lbazowa}
                       &&&&\cr\noalign{\vskip\Lbazowa}
                       &&&&\cr\noalign{\vskip\Lbazowa}
                       &&&&\cr\noalign{\vskip\Lbazowa}
                       &&&&\cr\noalign{\vskip\Lbazowa}
                       &&&&\cr\noalign{\vskip\Lbazowa}
                       &&&&\cr\noalign{\vskip\Lbazowa}
                       &&&&\cr\noalign{\vskip\Lbazowa}
                       &&&&\cr\noalign{\vskip\Lbazowa}
                       &&&&\cr}%
                  \or
                     \halign {\tabskip=\Podstep%10mm plus30mm minus3mm
                     \UNB\box\Nrboxa##&\UNB\box\Nrboxa##&%
                     \UNB\box\Nrboxa##&\UNB\box\Nrboxa##&%
                     \UNB\box\Nrboxa##&\UNB\box\Nrboxa##\tabskip0mm\cr
                       &&&&&\cr\noalign{\vskip\Lbazowa}
                       &&&&&\cr\noalign{\vskip\Lbazowa}
                       &&&&&\cr\noalign{\vskip\Lbazowa}
                       &&&&&\cr\noalign{\vskip\Lbazowa}
                       &&&&&\cr\noalign{\vskip\Lbazowa}
                       &&&&&\cr\noalign{\vskip\Lbazowa}
                       &&&&&\cr\noalign{\vskip\Lbazowa}
                       &&&&&\cr\noalign{\vskip\Lbazowa}
                       &&&&&\cr\noalign{\vskip\Lbazowa}
                       &&&&&\cr\noalign{\vskip\Lbazowa}
                       &&&&&\cr}%
                  \or
                     \halign {\tabskip=\Podstep%10mm plus30mm minus3mm
                     \UNB\box\Nrboxa##&\UNB\box\Nrboxa##&\UNB\box\Nrboxa##&%
                     \UNB\box\Nrboxa##&\UNB\box\Nrboxa##&%
                     \UNB\box\Nrboxa##&\UNB\box\Nrboxa##\tabskip0mm\cr
                       &&&&&&\cr\noalign{\vskip\Lbazowa}
                       &&&&&&\cr\noalign{\vskip\Lbazowa}
                       &&&&&&\cr\noalign{\vskip\Lbazowa}
                       &&&&&&\cr\noalign{\vskip\Lbazowa}
                       &&&&&&\cr\noalign{\vskip\Lbazowa}
                       &&&&&&\cr\noalign{\vskip\Lbazowa}
                       &&&&&&\cr\noalign{\vskip\Lbazowa}
                       &&&&&&\cr\noalign{\vskip\Lbazowa}
                       &&&&&&\cr\noalign{\vskip\Lbazowa}
                       &&&&&&\cr\noalign{\vskip\Lbazowa}
                       &&&&&&\cr}%
                  \else\fi}%
                  \Wys = \ht 0%
                  \global\roznica=\baselineskip
                  \multiply\roznica by\iloczyn
                  \advance\Wys by-\roznica
                  \global\roznica=\Lbazowa
                  \multiply\roznica by\iloczyn
                  \advance\Wys by-\roznica
                  \global\ht0=\Wys
                  \nobreak
                  \box0%
                  \vskip.5mm
                  }}

\endinput
