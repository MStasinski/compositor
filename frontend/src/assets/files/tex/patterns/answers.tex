\documentclass[10pt,a4paper]{article}

\usepackage{amsmath}[2000/07/18] %% Displayed equations
\usepackage{amssymb}[2002/01/22] %% and additional symbols
\usepackage{unicode-math}
\usepackage{pstricks-add}            %% pstricksy i ościenne pakiety. XeTeX sam znajduje kompatybilne
\usepackage{pst-node}
\usepackage{pst-plot}
\usepackage{fancyhdr}            %% do zamiany nagłówków
\usepackage{fontspec}            %% do fontów, patrz: tutorial
\usepackage{polski}             %%polskie łamanie
\usepackage{layout}



%====================================================
%         U S T A W I E N I A    S T R O N Y
%====================================================
\fancyhf{}                              %czyszczenie nagłówków i stopek
\renewcommand{\headrulewidth}{0pt}
\renewcommand{\footrulewidth}{0.4pt}
\setlength{\paperwidth}{595pt}
\setlength{\paperheight}{842pt}
\setlength{\topmargin}{0in}
\setlength{\hoffset}{-30pt}
\setlength{\voffset}{-40pt}
\setlength{\evensidemargin}{0in}
\setlength{\oddsidemargin}{0in}
\setlength{\marginparwidth}{0pt}
\setlength{\marginparsep}{0pt}
\setlength{\headheight}{0pt}
\setlength{\footskip}{14pt}
\setlength{\headsep}{0in}
\setlength{\textwidth}{496.5pt}
\setlength{\textheight}{770pt}
\setlength{\headwidth}{\textwidth}

\special{pdf:docinfo<</Title(Odpowiedzi)%
/Keywords(PDF-1.5[[user-id]])>>}

\begin{document}
\pagestyle{fancy}
%\chead{\rput[B](0,-3.5pt){\vrule height0.4pt depth0pt width735pt}}
\vbadness=10000   %żeby log był‚ czystszy
\hbadness=10000   %żeby log był‚ czystszy

\fancyfoot[FL]{Wybór zadań: [[user-name]] {\color{white}[[user-id]]}}
\fancyfoot[FR]{Copyright {\textcopyright} Gdańskie Wydawnictwo Oświatowe}

\input ../input/macros_[[compile-type]]   %tu siedzą makra nasze
%\kompozytorFont[10pt]
\skladfalse

{\setbox\strutbox=\hbox{\vrule height12pt depth0pt width0pt}
\offinterlineskip

{
\psframebox[linecolor=white,linewidth=.5pt,framesep=0mm]{
\parbox[t][8mm]{175mm}{
\vbox{
\hsize=175mm
\trialheader [[project-title]] \trialheaderR -- odpowiedzi
}
}
}
}

[[[outlet]]]


\end{document}